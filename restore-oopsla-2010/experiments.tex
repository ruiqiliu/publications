% \newcommand{\Y}{\ding{51}}
% \newcommand{\N}{\ding{53}}
\newcommand{\Y}{yes}
\newcommand{\N}{no}

\begin{table*}[t]
\begin{small}
\begin{center}
\begin{threeparttable}
\begin{tabular}{@{}lrlrrrrlrrr@{}} 
\textsf{\textbf{Application}}     & \textsf{\textbf{SLOC}}     & 
\textsf{\textbf{Version}} & \textsf{\textbf{\# Snaps}} &
  \textsf{\textbf{Classes}} & \textsf{\textbf{Heap Obj.}} & \textsf{\textbf{Target Obj.}} & 
% Tar. Flds & \% N-to-N &
  \textsf{\textbf{Match}} & \textsf{\textbf{Synthesis}} & \textsf{\textbf{Inferred}}   & \textsf{\textbf{Type}} \\ \hline

Azureus         & 250K      & r2514  & 11 &
  1616 & 1315420 & 97 & 
% 745 & 42.2 &
  0.842 s & 0.120 s & \Y  &  conditional \\

                &          & r120        & 22 &
  1634 & 1117463 & 275 & 
% 404 & 39.9 &
  0.002 s & 0.000 s & \N &          \\ \hline

JEdit           & 154K  & r14027  & 5 &
  3044 & 703360 & 30 & 
% 271 & 43.7 &
  0.041 s & 0.008 s & \Y &  constant \\

%                &  90K  & r5178\tnote{1}   & -- &
%  -- & -- & -- & 
%  -- & -- & \Y &  constant  \\ 

                & 150K  & r13413  & 5 &
  3221 & 747849 & 0 & 
% -- & -- &
  0.000 s & 0.000 s & \N &   \\ \hline

JES & 2.3K    & 1.3.2   & 1 &
  911 & 51802 & 1 &
% 49 & 67.8 &
  0.020 s & 0.022 s & \Y & collection \\

 & 2.4k & 1.3.3 & 1 & 902 & 52210 & 1
 & 0.001 s & 0.007 s & \Y & constant \\
 \hline

% -- & -- &

CrossFTP & 13.9K & 1.06 & 1 & 953 & 38907 & 1 & 0.001 s & 0.009 s & \Y & copy$^1$ \\
& 13.9K & 1.06 & 1 &  953 &  38907 & 1 & 0.002 s & 0.007 s & \Y & copy$^1$ \\
& 13.9K & 1.06 & 1 &  953 &  38907 & 1 & 0.002 s & 0.009 s & \Y & constant \\
& 14K & 1.07 & 1 & 2417 & 152531 & 1 & 0.044 s & 0.013 s & \Y & constant \\
& 18.1K & 1.08 & 3 &  954 & 116833 & 1 & 0.002 s & 0.011 s & \Y & conditional \\

\end{tabular}
%\begin{tablenotes}
%\footnotesize{\item [1] Full data for this case could not
%                  be obtained.} 
%\end{tablenotes}
\caption{Summary of updates and inferred transformers.\label{tab:evaluation} Each example is the result of a single test case. ${}^1$Transformer that would also have been produced automatically by Jvolve.}
\end{threeparttable}
\end{center}
\end{small}
\vspace*{-1em}
\end{table*}

\subsection{Results}

% This section presents our experience using \TOS to generate
% state transformers for updates to  open-source Java applications that
% we found in the wild. 
In our experiments, we consider two updates to Azureus (a bittorrent client); three updates to jEdit 
(a text editor); one update  to JavaEmailServer (a.k.a., JES, an SMTP and
POP mail server); and one update to CrossFTP (an FTP server).  These
applications range from 2.3k to 250k source lines of code. These are
actively developed and maintained programs not built with dynamic
software updating in mind.

We chose these applications and updates for the following reasons.
First, dynamic updates for these programs would be useful.  For
Azureus, JES, and CrossFTP, dynamic updates would avoid disruption due
to shutdown/restart, e.g., they would preserve active connections
involving e-mail sending or file transfer, and they would preserve
important in-memory state, such as file/metadata caches.  Dynamic
updates to JEdit are less important (you could save in-progress work
and restart), but would be convenient, e.g., to preserve the content
and layout of
onscreen windows.  Second, the updates to JES and CrossFTP were also
considered in the original Jvolve work~\cite{jvolve}.  We wanted to
see if we could use \TOS to generate transformers we previously
wrote by hand. Finally, we found updates to these programs that are
relatively interesting, such as updates that do not affect all objects
uniformly (thus requiring conditional transformers) or updates that
fix memory leaks.  We focus on the challenging updates that
prior DSU systems cannot handle automatically. We also tested \TOS 
on many simple updates, and include two of these---the ``copy'' transformers
for CrossFTP v1.06---as exemplars.
% \mwh{Reviewer
%   question:  And when you did use a version, are your results about the entire
% upgrade or just a particular class of interest? Without such information, it's
% hard to get a sense of how well the current set of heuristics actually works on
% real systems.}

Table~\ref{tab:evaluation} summarizes information about the updates
and the inferred transformers.  We list the version for which we
synthesize an update and the number of snapshots generated by our test
case. In all but one example, \TOS needs only a single test case.  
Version 1.08 of CrossFTP required three test cases.
We list the number of classes used by the program, the number of object
instances present (total across all snapshots), and the total number of
instances of the target object (target objects changed or transitively
referred to by changed objects). Finally, we list execution times in
seconds for
matching and synthesis, and indicate if synthesis succeeded.

Matching and synthesis times are negligible for all examples.  Matching
times and effectiveness benefit from focusing on a single class at a
time, since it
significantly reduces the number of objects that \TOS considers.
\TOS restricts itself to changed objects and objects reachable from changed
objects by a bounded number of field dereferences.  This bound is the
same one mentioned in Section \ref{sec:synth-discuss}, which discusses
bounding the depth of field paths.
Synthesis benefits from the fact that our language of state
transformers is designed to be small and succinct.

The synthesized functions involve
constant updates, conditional updates, string transformations, and
collection updates.  For JES and CrossFTP, we used the generated
update functions in Jvolve~\cite{jvolve} and verified that the system
continued running correctly following the update.  We were unable to
get Azureus and jEdit to run reliably using the Jvolve VM, but this
was not due to Jvolve: the Jikes RVM, on which Jvolve is based, does
not run them properly either.  Since we could not run the Azureus
and jEdit updates, we performed a manual code
review of the transformers produced by our synthesis algorithm to check that
they matched what we would have written by hand.

\TOS fails in two cases to synthesize update functions because we
could not generate snapshots that captured the changed behavior;
we discuss these situations, and all of the updates, in detail below.

\paragraph*{Azureus update SVN r2514}

Azureus is a widely used BitTorrent server/client. Version~r2514
contains about 250k lines of code. This update changes two classes and
methods.  Figure~\ref{fig:azureus-r2514}
shows a portion of the update.  The added call to
\code{clearServerAdapter} nulls the \code{adapter} field of the
\code{_server} object when a peer server is stopped; doing so ensures
the adapter is garbage collected.  When we apply this update at
run-time, we would like it to
retroactively null \code{adapter} fields that the buggy version
neglected to.  \TOS facilities doing so by synthesizing a transformer for 
\code{PEPeerControlImpl} objects.

To generate the snapshots, we ran both program versions in a
controlled setting and had them download the same set of files.
\TOS performs matching on
\code{PEPeerControlImpl} objects.  The program contains one such
object for each file it downloads.  Matching chooses key
field \code{_nbPieces}, which is a value proportional to the file size
that is highly likely to be unique across different files.

Synthesis then observes that for
these matching objects the new versions' \code{_server.adapter} field
is \code{null} when \code{_server.bContinue} is false, but the two
versions match on the \code{_server.adapter} field otherwise.  
It then infers the following conditional transformer, which has the effect
of freeing the leaked objects:

\begin{samepage}
\begin{quote}
\vspace*{.5em}
\begin{lstlisting}
if (_bContinue == false)
  _server.adapter = null;
else
  _server.adapter = _server_old.adapter;
\end{lstlisting}
\vspace*{-.5em}
\end{quote}
\end{samepage}

\noindent 
Note that to generate this transformation function requires using
field paths instead of single fields (cf. Section~\ref{sec:synth-discuss}).

\paragraph*{Azureus update SVN r120}
Figure~\ref{fig:azureus-r120} shows update r120 to Azureus. This update modifies the condition under which
Azureus releases the read buffer between a client and its peer.  
There are two issues with using \TOS to infer this update.  The first
is that the read buffer does not contain a natural key field.  Each
buffer is identical and does not have a pointer back to the object
using it.  This problem manifests as a failure in the matching
process.
In such a case, one might consider adding a ghost field that records
the allocation order and using this as a key field to match objects.
Matching succeeds with this addition. However, the assignment of buffers
to clients varies across runs (and is unrelated to allocation order),
which causes synthesis to fail, since the $i^{th}$ buffer is not
associated with the same client in each run. 
This transformation is thus not inferable from the objects
themselves, which is a limitation of our approach. However, it is not
surprising that, on occasion, sufficient information for transformation
is not available from the heap.

\begin{figure}[t]
\vspace*{-1em}
\begin{quote}
\begin{lstlisting}
class PEPeerControlImpl {
  PESharedPortServerImpl _server; 
  boolean _bContinue; ...
  void stopAll() { ...
    //3. Stop the server
    _server.stopServer();
+   _server.clearServerAdapter();
    ... }
}
class PESharedPortServerImpl {
  void clearServerAdapter() {
    adapter = null;
  }
}
\end{lstlisting}
\vspace*{-.5em}
\caption{Azureus r2514 update\label{fig:azureus-r2514}}
\end{quote}
\vspace*{-1em}

\begin{quote}
\begin{lstlisting}
public class PeerSocket ... { ...
     //4. release the read Buffer
-    if (readBuffer != null && !readingLength)
+    if (readBuffer != null)
       ByteBufferPool.getInstance().freeBuffer(readBuffer);
... }
\end{lstlisting}
\vspace*{-.5em}
\caption{Azureus update r120\label{fig:azureus-r120}}
\end{quote}
\end{figure}

\begin{figure}[t]
\vspace*{-1em}
\begin{quote}
\begin{lstlisting}
class TokenMarker {
  public LineContext markTokens(...) {
        ...
        tokenHandler.setLineContext(context);

        /* for GC. */
+       this.tokenHandler = null;
        this.line = null;
        return context;
    }
}
\end{lstlisting}
\vspace*{-.5em}
\caption{Update r14027 to jEdit\label{fig:jedit-r14027}}
\end{quote}
\vspace*{-1em}
\begin{quote}
\begin{lstlisting}
class HistoryText {
 showPopupMenu() {
   if(popup != null && popup.isVisible())
   {
       popup.setVisible(false);
+      popup = null;
       return;
   }
 
-  popup = new JPopupMenu();
+  popup = new JPopupMenu() {
+      @Override
+      public void setVisible(boolean b) {
+          if (!b) {
+              popup = null;
+          }
+          super.setVisible(b);
+      }
+  };
   JMenuItem caption = new JMenuItem(jEdit.getProperty(
       "history.caption"));
   caption.addActionListener(new ActionListener()
 }
}
\end{lstlisting}
\vspace*{-.5em}
\caption{Update to jEdit: SVN revision r13413\label{fig:jedit-r13413}}
\end{quote}
\vspace*{-1em}
\begin{quote}
\begin{lstlisting}
class DataConnectionConfig {
   ...
+  boolean enableBonjour = true;
+  String listEncoding = "utf-8";
}
\end{lstlisting}
\vspace*{-.5em}
\caption{Update to CrossFTP in version 1.07\label{fig:crossftp}}
\vspace*{-1.5em}
\end{quote}
\end{figure}

\paragraph*{jEdit}
JEdit is a text editor for programmers that provides common and
advanced features,
such as syntax highlighting, folding, automatic indentation, a  built-in macro language, macro recording,
and plugin support. Here we consider two similar memory leaks fixed in
jEdit versions r5178 and r14027.  Figure~\ref{fig:jedit-r14027} shows
the source patch for the leak fixed in r14027.  jEdit calls the function
\code{markTokens} when it needs to split a string into tokens based on
the type of the file being edited. Each file type (C, Java, Verilog,
etc.) has special logic to split text in a line and embed it in an object
of type \code{TokenHandler}. The leaky jEdit version fails to set the
field \code{TokenMarker.tokenHandler} to \code{null}.

The inferred object transformer % shown in Figure~\ref{fig:jedit-r14027}~(b)
is simple. It sets the leaky field to \code{null}. It is safe to execute the
transformer as long as the \code{markTokens} function is not active on
stack.

% To cause a leak, we need multiple TokenMarker objects. TokenMarker objects
% are pointed to by field Mode.marker in class Mode. A Mode object is created
% for each file mode (corresponding to syntax) such as c, java, verilog, xml,
% etc. We can create multiple Mode objects, and correspondingly TokenMarker
% objects, by opening buffers of various syntax types.
% 
% Calling TokenMarker.markTokens() would leave in place reachable, but dead
% references to TokenHandler objects. There is a call to markTokens() in
% TextUtilities.findMatchingBracket(). This function is called even as you
% move the cursor around a Java file. Moving the cursor next to a bracket
% character calls markTokens().


Figure~\ref{fig:jedit-r13413} shows a change to jEdit in SVN revision r13413.
The leak is in function \code{HistoryText.showPopupMenu()}. jEdit calls
\code{showPopupMenu()} when the user performs certain actions, such as right
clicking a text field with history.
%HistoryText objects occur in two types
%of classes HistoryTextArea and HistoryTextField. HistoryTextArea and
%HistoryTextField objects are found in various types of dialogs: find and
%replace, file browser, etc.
In the old version, some instances of \code{HistoryText} have the
boolean field \code{popup.visible} set to true and others set it to
false. In the new version, the field \code{popup} is not null only when
an object also has its \code{popup.visible} field set to true. \TOS can
infer this property and generate a transformer that nulls the
\code{popup} field of instances that have \code{popup.visible} set to
false.  However, while creating snapshots we were unable to create a situation where
\code{popup.isVisible()} was true during a snapshot, 
which would execute
instructions controlled by the condition and exercise the leak.  This
update is an example that \TOS is capable of handling,
but for which we fail due to test coverage problems.

\paragraph*{JavaEmailServer (JES)}  The collection update is the one
discussed in Section \ref{sec:overview} and presented in Figure
\ref{fig:email-example}. We automatically generate a correct update
function for it.  The constant update involves the addition of a new
field \code{deliveryAttemptThreshold}.  This value controls how many
times JES should try to deliver a message before discarding the
message.  We automatically generate an initializer that sets this to
the default value.  This default value is not present in the code, but
rather in a configuration file that is read during JES start-up.
Thus it would not have been discovered by other approaches to
transformer generation.

\paragraph*{CrossFTP}
In this Java-based FTP server, the \textsf{\small{DataConnectionConfig}} class
maintains configuration information about the connection between the client
and the server. Its fields control what response the server sends to
various commands a client issues. Version 1.07 of CrossFTP adds two new
fields to this class. The boolean field \code{enableBonjour} controls whether the
server should support the Bonjour protocol and the String field
\code{listEncoding} specifies what encoding the server should use when responding to
the \code{LIST} command. There is always only one instance of this object in the
heap. From the heap snapshots, \TOS identifies the value of these fields in the
new version and generates the transformation function that sets these fields accordingly.

For version 1.06, we consider three field updates, two of which
involve copying the old version to the new and one of which involves
a constant initializer.  The copy cases involve fields whose access
modifiers have changed.  In such cases, simply copying the old value
to the new is a good heuristic and this transformer would be generated
automatically by Jvolve.  Our system also generates the transformer
but provides the added assurance that this update is consistent with
the states observed when running the old and new versions.

In version 1.08, the default port used by a \code{SocketFactory}
object is changed.  We provide \TOS with three snapshots at each
version---two in which non-default ports are configured and one that
has no port configured and thus falls back on the default port.  \TOS
is able to synthesize the conditional transformer that changes the
port to the new default value if it was set to the old default value
and leaves it unchanged
otherwise.

\paragraph*{Discussion} These first experiments with \TOS show that
good results can be obtained when matching and synthesis work in harmony.
If either step fails then \TOS fails for the targeted class.  But if
matching identifies key fields then synthesis-based matching can
be avoided or reduced and low \TOS run times are observed.
For the
examples we considered, most objects do have key fields that matching identifies and
uses to produce good examples for synthesis. Although it seems
intuitive that most programs will encode sufficient state in changed
classes to make matching practical, future work should explore this
question more thoroughly. Our synthesis language is relatively simple,
which eases synthesis, yet it includes common string and data
functions.  Future work should explore if the current synthesis
language has sufficient coverage on a wider range of programs.

% \mwh{Could we speculate about other dynamic updates we'd expect to be
%   able to handle, e.g., from Ginseng?}

% \mwh{reviewer comments we might address:}

% - Right now your experiments demonstrate mostly constant
% transformers. Is it fair to say that these are the ones that would
% also be easy for a human to come up with?

% You don't give any run-time measurements for the experiments described in
% Section 5. I assume that the implementation is rather slow, because it has to
% consider all fields (or actually all field combinations) of all heap objects in
% all snapshots for both the old and the new version of the updated program. I
% strongly suggest that the evaluation should contain run-time measurements.
