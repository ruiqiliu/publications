\documentclass[natbib]{sigplanconf}

\usepackage{color}
\usepackage{mathptmx}
\usepackage[scaled=.92]{helvet} % see www.ctan.org/get/macros/latex/required/psnfss/psnfss2e.pdf
\usepackage{graphicx}
\usepackage{xspace}
\usepackage{listings}
\lstset{language=Java,basicstyle=\small\sffamily,columns=flexible,numberstyle=\em\scriptsize,frame=none,mathescape=true}
\newcommand{\code}[1]{\lstinline|#1|\xspace}
\newcommand{\javac}{javac\xspace} % This package lets you punctuate \javac normally and get good spacing, e.g., \javac.  gives you: javac.
\usepackage{url}  % particularly useful for URLs in bib entries
% packages
\usepackage{slashed}

\usepackage{array}
\usepackage{longtable}
\usepackage{amsmath}
\usepackage{amssymb}
\usepackage{amsthm}


\usepackage{stmaryrd}
\usepackage{ifthen}
\usepackage{marvosym}
\usepackage{graphicx}
\usepackage{calc}
\usepackage{wasysym}
\usepackage{rotating}

% \usepackage{pdfsync}
% \usepackage{proof}
% \usepackage{bussproofs}
% \usepackage{mathpartir}
% \usepackage{varwidth}

% \usepackage{tikz}
% \usetikzlibrary{calc}
% \usetikzlibrary{arrows}
% \usetikzlibrary{positioning}
% \usetikzlibrary{chains}
% \usetikzlibrary{decorations.pathmorphing}
% \usetikzlibrary{decorations.pathreplacing}
% \usetikzlibrary{shapes.misc}
% \usetikzlibrary{shapes.multipart}
% \usetikzlibrary{shapes.geometric}
% \usetikzlibrary{backgrounds}

% Select symbols from bbding
\newcommand{\dingfamily}{\fontencoding{U}\fontfamily{ding}\selectfont}
\newcommand{\dingSymbol}[1]{{\dingfamily\symbol{#1}}}
\newcommand{\RectangleThin}{\dingSymbol{'164}}
\newcommand{\Rectangle}{\dingSymbol{'165}}
\newcommand{\RectangleBold}{\dingSymbol{'166}}

% semantic styles
% \usepackage[ligature,shorthand,reserved]{semantic}
% \newcommand{\semanticdomainstyle}[1]{%
%   \ensuremath{\mathchoice%
%     {\mbox{\textit{#1}}}%
%     {\mbox{\textit{#1}}}%
%     {\mbox{\textit{\scriptsize#1}}}%
%     {\mbox{\textit{\tiny#1}}}}}
% \reservestyle{\semanticdomain}{\semanticdomainstyle}
% \semanticdomain{Int[\ensuremath{\mathbb{Z}}]}

% \newcommand{\semanticvaluestyle}[1]{\ensuremath{\bf{\mathrm{#1}}}}
% \reservestyle{\semanticvalue}{\semanticvaluestyle}

% Rotating characters
\newcommand{\rotxc}[1]{\begin{sideways}#1\end{sideways}}
\newcommand{\rotrotxc}[1]{\rotxc{\rotxc{#1}}}


% Funky underline types
\def\dotuline{\bgroup
  \ifdim\ULdepth=\maxdimen  % Set depth based on font, if not set already
   \settodepth\ULdepth{(j}\advance\ULdepth.4pt\fi
  \markoverwith{\begingroup
  \advance\ULdepth0.08ex
  \lower\ULdepth\hbox{\kern.15em .\kern.1em}%
  \endgroup}\ULon}

\def\dashuline{\bgroup
  \ifdim\ULdepth=\maxdimen  % Set depth based on font, if not set already
   \settodepth\ULdepth{(j}\advance\ULdepth.4pt\fi
  \markoverwith{\kern.15em
  \vtop{\kern\ULdepth \hrule width .1em}%
  \kern.01em}\ULon}

% Common constructs
\renewcommand{\iff}{\text{ iff }}
\newcommand{\hole}{\Box}
\newcommand{\true}{\mathsf{true}}
\newcommand{\fail}{\mathsf{fail}}
\newcommand{\false}{\mathsf{false}}
\newcommand{\strue}{\mathrm{T}}
\newcommand{\sfalse}{\mathrm{F}}

% math ligatures
% \mathlig{--`}{\rightharpoonup} % partial function
% \mathlig{|->}{\mapsto} % tight points-to
% \mathlig{->}{\rightarrow} % remap for functions
% \mathlig{-->}{\longrightarrow} % remap for functions
% \mathlig{[[}{\llbracket} % open double bracket
% \mathlig{]]}{\rrbracket} % close double bracket
% \mathlig{<=>}{\Leftrightarrow}

% quantifiers
\newcommand{\A}[1][]{\forall \ifthenelse{\equal{#1}{}}{}{\mbox{$#1.$}\ }}
\newcommand{\E}[1][]{\exists \ifthenelse{\equal{#1}{}}{}{\mbox{$#1.$}\ }}
\newcommand{\Lam}[1][]{\lambda \ifthenelse{\equal{#1}{}}{}{\mbox{$#1.$}\ }}
\newcommand{\Am}[1][]{\forall_{\!*} \ifthenelse{\equal{#1}{}}{}{#1.\,}}
\newcommand{\Em}[1][]{\exists_{*} \ifthenelse{\equal{#1}{}}{}{#1.\,}}
\newcommand{\nA}[1][]{\not\forall \ifthenelse{\equal{#1}{}}{}{#1.\,}}
\newcommand{\nE}[1][]{\nexists \ifthenelse{\equal{#1}{}}{}{#1.\,}}

% Common functions
\newcommand{\dom}{\mathit{dom}}

% Common Decorations
\newcommand{\defemph}[1]{\textbf{#1}}
\newcommand{\estate}[1]{\langle #1 \rangle}

% make eqnarrays have more space
\renewcommand\arraystretch{1.2}

% Theorem Environments
\newtheorem{lem}{Lemma}
\newtheorem*{lemstar}{Lemma}
\newtheorem{thm}{Theorem}
\newtheorem{corr}{Corollary}
\newtheorem{defn}{Definition}
\newtheorem{claim}{Claim}
\newtheorem{obs}{Observation}
\newtheorem{prob}{Problem}

% comment blocks
\newcommand{\deleted}[1]{}
\newcommand{\draftonly}[1]{}
\newcommand{\stephencomment}[1]{
\begin{quote}
{\bf *** Stephen says:} {\it #1}
{\bf ***}
\end{quote}}

% indentation
\newcommand{\ind}{\hspace*{0.5cm}}

% \usepackage[ruled]{algorithm2e}
% Algorithm package options
% \SetInd{1ex}{1ex}

% extra algorithm constructs
% \SetKwIf{Let}{Failed}{let}{in}{match failed $=>$}{end}
% \SetKwSwitch{Match}{MCase}{MOther}{match}{with}{}{\_}{end}
% \SetKwInput{Type}{Type}
% \SetKwBlock{Ind}{\vspace{-3ex}}{}
% \SetKwBlock{IndEnd}{\vspace{-3ex}}{end}
% \SetKwIf{DefFn}{}{fun}{=}{}{in}
% \SetKwFor{uForEach}{foreach}{do}{\vspace{-3ex}}

\newcommand{\none}{\mathsf{None}}
\newcommand{\some}[1]{\mathsf{Some}\bigl(#1\bigr)}
\newcommand{\somenop}[1]{\mathsf{Some}#1}

% Bnf definitions
\newcommand{\GoesTo}{\mathrel{::=}}
\newcommand{\bnfdef}{\GoesTo}
\newcommand{\bnfalt}{\Or}
\newcommand{\bnfas}{\bnfdef}

% Math constructs
\newcommand{\sembrack}[1]{\llbracket#1\rrbracket\,}
\newcommand{\imp}{=>}

% Transition systems
\newcommand{\steps}[1]{\stackrel{#1}{\rightarrow}}
\newcommand{\stepst}[1]{\stackrel{#1}{\rightarrow^{*}}}
\newcommand{\stepsp}[1]{\stackrel{#1}{\rightarrow^{+}}}
\newcommand{\gentrans}{\dashrightarrow}
\newcommand{\gentransnum}[1]{\stackrel{#1}{\dashrightarrow}}
\newcommand{\trans}[1][]{\stackrel{#1}{\leadsto}}
\newcommand{\transt}[1][]{\mathrel{\trans[#1]{\!\!}^{*}}}
\newcommand{\transp}[1][]{\mathrel{\trans[#1]{\!\!}^{+}}}

%% derivation rules
% note that horizontal space can be hacked with:
%% \def \mpr@andskip {1em plus 0.25fil minus 0.25em}
% \renewcommand{\TirName}[1]{$#1$}
% \renewcommand{\RefTirName}[1]{$#1$}

\newcommand{\infrulelabel}[1]{\mbox{\normalfont\small\textsc{#1}}}
\newcommand{\drule}[4][]
  {\inferrule*[lab=\<#2>,right=\ensuremath{#1}]{#3}{\textstyle#4}}
\newlength{\templength}
\newcommand{\userule}[4][leftskip=0em]
  {\inferrule*[right=\textsc{#2},before=\setlength{\templength}{\widthof{\textsc{#2}}},rightskip=\templength,#1]{#3}{\textstyle#4}}
\newcommand{\daxiom}[3][]{\drule[#1]{#2}{ }{#3}} % derivation axiom
\newcommand{\daxiombc}[3]{\drulebc{#1}{#2}{ }{#3}}
\newcommand{\usingrule}[3]{\prooftree[#1]\using#2\leadsto#3\endprooftree}
\newcommand{\belowcondition}[1]{\rule{0pt}{3ex} {}^\dagger #1}
\newcommand{\drulebc}[4]{\drule[{}^\dagger]{#2}{#3}{#4\\\\\belowcondition{#1}}}
\newcommand{\eqjudg}[2]{\stackrel{\textstyle#1}{#2}}
%% \newcommand{\eqjudg}[2]{\begin{eqnalign}[c][b]#1\\[-2.5ex]#2\end{eqnalign}}
\newcommand{\pfdots}[1]{\stackrel{\textstyle\vdots}{#1}}

% \reservestyle{\infrule}{\infrulelabel}
% \infrule{}

% Stacking
\newcommand{\stacklabel}[1]{\stackrel{\smash{\scriptscriptstyle \mathrm{#1}}}}
\newcommand{\Def}{\stacklabel{def}}
\newcommand{\Fin}{\stacklabel{fin}}

\newcommand{\spec}[3]{\{#1\}\;#2\;\{#3\}} % Hoare triple

% Separation Logic
\newcommand{\emp}{\mathbf{emp}}
\newcommand{\pto}{\mapsto}
\newcommand{\lbl}[1]{\mathsf{#1}}
\newcommand{\dll}[1]{\mathrm{dll}(#1)}

% Program commands
\newcommand{\assume}{\mathsf{assume}}
\newcommand{\nondet}[1]{#1 :=\ ?}


\newcommand{\TOSAcronym}{\emph Targeted Object Synthesis (TOS)\xspace} 
\newcommand{\TOS}{TOS\xspace} 

\newcommand{\stephen}[1]{\textcolor{blue}{Stephen: #1}}
\newcommand{\sbm}[1]{\textcolor{blue}{Stephen: #1}}
\newcommand{\suriya}[1]{\textcolor{blue}{Suriya: #1}}
\newcommand{\kathryn}[1]{\textcolor{blue}{Kathryn: #1}}
\newcommand{\mike}[1]{\textcolor{blue}{Mike: #1}}

\begin{document}

\conferenceinfo{Mystery'11,} {January 1, 2011, Paris, TX.}
\CopyrightYear{2011}
\copyrightdata{2011}


\title{Automating Object Transformations for Dynamic Software Updating
%   \thanks{COMMENT OUT FOR SUBMISSION, UNCOMMENT FOR final version. Some of these may or may not apply to your work.  Make Kathryn check.
%     This work is supported by NSF SHF-0910818, NSF CSR-0917191, NSF
%     CCF-0811524, NSF CNS-0719966, NSF CCF-0429859, Intel, IBM, CISCO,
%     Google, and Microsoft.  Any opinions, findings and conclusions
%     expressed herein are the authors' and do not necessarily reflect
%     those of the sponsors.}
    }

% 2008: NSF EIA-0303609, DARPA F33615-03-C-4106,


\authorinfo{Anonymous}{}{}
%% \authorinfo{Stephen Magill, Michael Hicks}
%%            {University of Maryland, College Park}
%%            {smagill@cs.umd.edu}

%% \authorinfo{Suriya Subramanian}
%%            {Intel Corporation}
%%            {suriya@gmail.com}

%% \authorinfo{Kathryn S. McKinley}
%%            {Microsoft Research}
%%            {mckinley@cs.utexas.edu}

\maketitle

\subsection*{Abstract}
Software evolves to fix bugs and add features. Stopping and restarting
programs to apply changes is inconvenient and often costly.  Dynamic
software updating (DSU) addresses this problem by updating programs
while they execute, but existing DSU systems for managed languages do
not support many updates that occur in practice and are inefficient.
This paper presents the design and implementation of \DSU, a
DSU-enhanced Java VM.  Updated programs may add, delete, and replace fields
and methods anywhere within the class hierarchy.  \DSU{} implements
these updates by adding to and coordinating VM classloading,
just-in-time compilation, scheduling, return barriers, on-stack
replacement, and garbage collection.  \DSU{} is \emph{safe}: its use
of bytecode verification and VM thread synchronization ensures that an
% ISSUE: no bytecode verification in JikesRVM
update will always produce type-correct executions. \DSU{} is
\emph{flexible}: it can support 20 of 22 updates to three
open-source programs---Jetty web server, JavaEmailServer, and CrossFTP
server---based on actual releases occurring over 1 to 2 years.
%REFACTOR: though slight refactorings would allow the other updates.
\DSU{} is \emph{efficient}: performance experiments show that
\DSU{} incurs \emph{no overhead} during steady-state execution.  These
results demonstrate that this work is a significant step towards
practical support for 
dynamic updates in virtual machines for managed languages.

%%% Local Variables: 
%%% mode: latex
%%% TeX-master: "pldi64"
%%% End: 


% {\scriptsize
% \category{D.3.4}{Programming Languages}{Processors}[Memory management (garbage collection); Optimization] 
% \terms
% Experimentation, Languages, Performance, Measurement
% \keywords
% Heap
% }

\newcommand{\eqf}[1]{=_{#1}}
\newcommand{\eqfv}[2]{=_{#1:#2}}
\newcommand{\eqfvclass}[2]{[#2]_{#1}}
\newcommand{\eqfclass}[2]{[#2]_{#1}}
\newcommand{\fset}{F}
\newcommand{\objset}{O}
\newcommand{\setsize}[1]{\lvert #1 \rvert}
\newcommand{\no}{\hat{o}}
\newcommand{\nobjset}{\hat{O}}
\newcommand{\nL}{\hat{L}}

\newcommand{\se}{\textit{se}}
\newcommand{\oldvar}{\textrm{o}}
\newcommand{\newvar}{\textrm{n}}
\newcommand{\casestart}{\textsf{case }}
\newcommand{\caseend}{\textsf{ end}}
\newcommand{\carrow}{\Rightarrow}
\newcommand{\sconcat}{\textsf{concat}}
\newcommand{\filtermap}{\textsf{filtermap}}
\newcommand{\map}{\textsf{map}}
\newcommand{\delim}{\textit{delim}}
\newcommand{\substr}{\textsf{substr}}
\newcommand{\op}{\mathrel{\diamond}}

\newcommand{\degree}[2]{\textit{degree}_{#1}(#2)}
\newcommand{\maxdegree}[1]{\textit{degree}_\textrm{max}(#1)}
\newcommand{\atrace}{\alpha}
\newcommand{\env}{\epsilon}
\newcommand{\runprog}[3]{\textit{runprog}(#1,#2,#3)}
\newcommand{\values}[2]{\textit{values}_{#1}(#2)}
\newcommand{\set}{\sigma}
\newcommand{\setlst}{\vec{\set}}
\newcommand{\concat}{\mathrel{\textit{::}}}
\newcommand{\restrictval}[3]{#1 \mathord\downharpoonright_{#2 = #3}}
\newcommand{\lstlength}[1]{\textit{length}(#1)}
\newcommand{\equivcount}{\stackrel{\#}{=}}
\newcommand{\updset}{\Delta}
\newcommand{\matching}[1]{\sim_{#1}}
\newcommand{\return}{\textbf{ret}}

\newcommand{\inset}{\sigma_{\text{in}}}
\newcommand{\outset}{\sigma_{\text{out}}}
\newcommand{\rank}[3]{\mathit{rank}(#1,#2,#3)}
\newcommand{\anew}[1]{\mathsf{new}\ #1}

\newcommand{\slold}{\setlst_{\mathit{old}}}
\newcommand{\slnew}{\setlst_{\mathit{new}}}


\input intro

\input overview

\input related

\section{Example}

\input matching

\section{Synthesis Language}

\section{Using Synthesis to Guide Matching}

This is something Mike and I talked about at POPL.  It seems like a
good idea that could help us match objects even when there is not a
clear set of ``key fields''.  The issue is complexity.  Suppose we
have $M$ old-version objects and $N$ new-version objects (both of the
same class).  There are $M \times N$ possible pairings.  Of course we
are already doing $M \times N$ edit distance computations in the
matching phase, but these comparisons are likely much faster than
synthesis will be.

One hybrid approach would be to use edit-distance-based matching and
then use synthesis to fine-tune the matching in cases that are not
clear-cut.  So if $o_1$ is edit distance 4 from $o_2$ and edit
distance 5 from $o_3$, then maybe use synthesis to help determine
which pairing should be chosen.

We could also use synthesis to check whether it is a good idea to
introduce a key field.  If there are no natural key fields, but we
think that allocation order might be a good matching heuristic, we
could try it and see if synthesis succeeds.

It seems that when using synthesis as a ``sanity check'' for the
matching step, it is better to be doing synthesis for a less
expressive language.  The more expressive the language is, the less
likely synthesis is to fail, making it less useful as a filter for
matching.  This is just intuition though and could be totally wrong.

\section{Experiments}

Examples we have:
\begin{enumerate}
\item Azureus (at least two leaks.  need to find some non-leak examples as well.)
\item jEdit
\item Jetty
\item Derby
\item could potentially look at Java version of BerkelyDB---though only major releases are available (no fine-grained SVN-level updates).
\end{enumerate}

For each of these, we can do a few experiments.

\begin{enumerate}
\item The simplest is just to find a change and check that we can
  synthesize the transformer when presented with the versions on
  either side (as given by SVN).  That is, the only thing that has
  changed is the code that directly relates to the change we are
  interested in.
\item Next we can check that we still find this transformer when we
  look at the major versions on either side of the targeted change.
  So now more code has been modified.  We still give the matching tool
  the name of the class we are targeting though.
\item The highest goal would be to give the system two major versions
  and see what percentage of classes we can synthesize updates for.
\end{enumerate}


% \input{beef}
% \input{results.tex}

\section{Future Work}
\ShowTOC
\begin{frame}{Looking ahead}%{A Sub-title is optional}
\begin{center}
\begin{Huge}
What's in the path towards mainstream adoption?
\end{Huge}
\end{center}
\end{frame}

\begin{frame}{Safety}%{A Sub-title is optional}
\begin{itemize}
\item Currently, we guarantee type safety. Can we do more?
\item Safe behavior in a general context is undecidable \cite{Gupta94}
\item Divide program into transactions, ensure they are version consistent
\cite{neamtiu08context}
\item Exhaustively testing updates
\end{itemize}
\end{frame}

\begin{frame}{Flexibility}%{A Sub-title is optional}
\begin{itemize}
\item Supporting every possible update is not a goal
\item Support most of the updates that occur in practice
\item Large and complex applications and updates
\item Understand update semantics from refactorings
\end{itemize}
\end{frame}

\begin{frame}{Efficiency}%{A Sub-title is optional}
\begin{itemize}
\item Update time roughly proportional to GC time
\item Semi-space GC is not practical
\item We have two requirements
  \begin{itemize}
  \item Copying objects
  \item Full-heap collection
  \end{itemize}
\item Other collectors? Concurrent collectors?
\end{itemize}
\end{frame}

\section{Conclusion}
% \ShowTOC

% \begin{frame}{Future work}%{A Sub-title is optional}
% \begin{itemize}
% \item Baby steps towards using static analysis towards asserting safety of
% updates
% \item Restricting update points based on semantics of the update
% \item How can we show that these update points are indeed safe?
% \end{itemize}
% \end{frame}

\begin{frame}{Conclusion}%{A Sub-title is optional}
\begin{itemize}
\item \DSU{}, a Java VM with support for Dynamic Software Updating
\item Most-featured, best-performing DSU system for Java
\item Naturally extends existing VM services
\item Supports about two years worth of updates
\end{itemize}
\begin{block}{}
\emph{Dynamic software updating in managed languages can be achieved in a
safe, flexible and efficient manner.}
\end{block}
\begin{center}
Source code and other information:
\url{http://www.cs.utexas.edu/~suriya/jvolve}
% Also available in the Jikes
% RVM research archive.
\end{center}
\end{frame}

% \begin{frame}{Research plan}%{A Sub-title is optional}
% \begin{center}
% \begin{large}
% Bridge the gap between static and dynamic languages
% \end{large}
% \end{center}
% \begin{itemize}
% \item Static Languages: Java, C\#
%   \begin{itemize}
%   \item Compile-time type safety
%   \item Static analysis
%   \item High performance VMs
%   \end{itemize}
% \item Dynamic Languages: Python, Ruby
%   \begin{itemize}
%   \item Expressive
%   \item Concise programs, easy to write/understand
%   \item No compile-time guarantees
%   \end{itemize}
% \end{itemize}
% \begin{center}
% \begin{large}
% Improved safety and performance of Dynamic languages
% \end{large}
% \end{center}
% \end{frame}
% 
% \begin{frame}{Solutions}%{A Sub-title is optional}
% \begin{itemize}
% \item Diamondback Ruby: Static type checking for Ruby
% \item Languages targeting the JVM: Clojure, Groovy, etc.
% \item Dynamic Language VMs: Unladen Swallow, PyPy
% \item Gradual typing
% \item DaVinci project: Dynamic language support in JVM
% \item Static Typing Where Possible, Dynamic Typing When Needed
% \end{itemize}
% \end{frame}
% 
% \begin{frame}{Conclusion}%{A Sub-title is optional}
% \begin{itemize}
% \item Work towards improving the robustness of software systems
% \item Current work: A VM-based Dynamic Software Updating solution
% \item In the future: Robust Dynamic Languages
% \end{itemize}
% \end{frame}

\appendix
\begin{frame}[allowframebreaks]{References}%{A Sub-title is optional}
\bibliographystyle{apalike}
\bibliography{paper.bib}
\end{frame}



\bibliographystyle{abbrvnat}
% \renewcommand{\bibfont}{\footnotesize} % <--- change bib font size here
% \setlength{\bibsep}{0.5ex}             % <--- change space between bib entries here
\bibliography{paper}

\input oldtext 

\input conflicts

\end{document}
