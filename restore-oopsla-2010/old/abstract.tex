
\begin{abstract}
 
  Dynamic software updating (DSU) systems eliminate costly downtime by
  dynamically fixing bugs and adding features in the code and data of
  executing programs.  Given a static \emph{code} patch, most systems
  can construct a corresponding run-time code change completely
  automatically.  However, a dynamic update must also specify how to
  change the running program's execution \emph{state}, e.g., its stack
  and heap, to be compatible with the new code.  Constructing such
  \emph{state transformations} correctly and automatically remains an
  open problem.

  This paper presents a solution to this problem called \TOSAcronym.
  \TOS works by running the same tests on the old and new program
  versions separately, observing the program state at relevant points.
  Given pairs of corresponding states, \TOS infers functions that
  produce new-version objects given their old-version counterparts.
  This two-part process consists of (1) \emph{matching} corresponding
  objects between the two versions, and (2) \emph{synthesizing} the
  simplest-possible function that transforms the old version of each
  matched object to its corresponding new version.  Synthesis infers
  simple functions, such as copying and arithmetic operations,
%(new.f =  old.g + old.h)
  and more complex functions, such as partitioning
  strings,
%(new.s = prefix(k, old.s)), 
  partitioning collections, and filtering objects from a
  collection to fix leaks. 

  We show the efficacy of \TOS by inferring updates for X versions of
  Y different programs from open source repositories.  This experience
  shows that \TOS handles many updates that occur in practice and can
  thereby further reduce programmer effort when preparing a dynamic
  update.

\end{abstract}

%%% Local Variables: 
%%% mode: latex
%%% TeX-master: t
%%% End: 
