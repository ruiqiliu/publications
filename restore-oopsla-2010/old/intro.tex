
\section{Introduction}

Suppose you are running an on-line service and a memory leak in your
server software causes it to regularly run out of memory and
crash.  Eventually you discover the one-line fix: the
\code{Connection} class's \code{close} method should
unlink some metadata when a connection closes.  To apply this
fix in a standard deployment you would stop your server and restart
the patched version, but doing so would disrupt active users.  If your
server has \emph{dynamic software updating} (DSU) support,
e.g., by running on an extended VM like JRebel~\cite{javarebel}, you
can do better: apply a \emph{dynamic} patch 
to the \code{Connection} class of your \emph{running}
system to prevent further leaks and without disrupting current users.
In some DSU systems, such as Jvolve~\cite{jvolve}, you can do better
still: you can include \emph{state 
  transformation} code in your dynamic patch that will traverse the
heap and unlink the metadata that was left alive by the bug.

% Prior work shows how to implement code and
% data updates with special compilation support~\cite{update-streaks},
% run-time binary rewriting~\cite{}, and extended virtual machine (VM)
% functionality~\cite{jvolve}.    They further demonstrate successful
% application of years' worth of changes to applications and servers,
% such as the OpenSSH daemon, Very Secure FTP daemon (vsftpd), and
% memcached, to running systems to bring them up to date, even when they
% are actively performing work~\cite{jvolve,update-streaks}.

An important goal toward furthering the adoption of DSU systems is to
make them easy to use, i.e., to minimize the effort required to
produce a correct dynamic patch from two versions of a system.  As a
step in this direction, many DSU systems employ simple
\emph{syntactic}, \emph{type-based} tool support for constructing a
dynamic patch from the old and new program
versions~\cite{jvolve,ksplice,neamtiu06dsu, HicksNettles03}.  For
example, if class \code{C}'s method \code{m}'s bytecode has changed, Jvolve
will include \code{C.m} in the dynamic patch.  If \code{C}'s field
definitions have changed, in type or number, Jvolve will include a an
\emph{object transformation function} that will be applied to all
\code{C} objects when the patch is applied; the function will retain
the values of unchanged fields and initialize the rest with a default
value, e.g., \code{null}.

While tool support for identifying changed code is highly effective,
existing support for constructing state transformation code is rarely
sufficient, meaning the programmer must modify or add to the
generated code.  Indeed, the programmer would have to add the code that
unlinks the leaked metadata in our example.  Unfortunately, the cases
that require manual intervention are often challenging to get right.
For our example, the state transformer cannot simply unlink all
connection metadata unconditionally; instead, it must use appropriate
context, garnered from the running program's heap and stack, to
identify and unlink only the metadata that is logically dead.
Transformations that require moving objects between collections or
partitioning single objects into several objects---examples we have
observed in practice---require similar care in their construction.
Thus, writing state transformation code for DSU systems can be a
time-consuming, error-prone process.

Toward improving this state of affairs, this paper presents a
general-purpose approach for sythesizing state transformation code for
dynamic patches.  Our approach, which we call \emph{targeted object
  synthesis} (\TOS), works by comparing execution traces gathered from
similar runs of the old and new program versions.  Examining objects
from the two traces that play a similar role, we synthesize code that
could produce the new objects given the old ones (and objects related
to them).  We developed \TOS for the Java-based DSU system Jvolve, but
our techniques should adapt readily to other DSU systems that support state
transformation.

% shows how to automatically synthesize object state
% transformers and thus in many cases, significantly reduce or eliminate
% developer effort due to DSU.  

In more detail: \TOS has two phases, \emph{matching} and \emph{synthesis}.
Matching produces a 1 to $n$ mapping of old to new objects,
where $n \geq 0$ and $n=1$ is frequent.  \emph{Synthesis} infers an
object transformation function for this mapping.  \TOS matching
analyzes the code to target heap observations to the subset of objects
in the heap that correspond to changed classes. It executes the old
and new version of the program on the same input and observes the old
and new target objects. Matching finds anchor fields that are the
same in old and new objects, exploits type information on references
for inexact matching, and allocation order to avoid an exponential
search.  The result is a proposed 1 to $n$ mapping of old to new
objects.

To ease \TOS \emph{synthesis}, we define a simple \TOS object
transformation language and then search for functions in this language
consistent with the examples.  The choice of language is crucial to
accuracy and efficiency\textemdash the simpler the language the more
efficient, but the less expressive.  Since object references refer to
other objects, we recursively interleave matching and synthesis.
Object transformation synthesis is similar to recent work on
synthesizing string, Excel table, and code transformation functions
from input and output
examples~\cite{Gulwani:popl:11,Gulwani:pldi:11,MKM:11}.  Compared to
the prior work, \TOS object transformation functions have a richer
domain, require matching, and require recursive synthesis. For example, object transformers  are a
superset of string transformations, more general than tables, and
orthogonal to inserting and deleting code.  The result of the \TOS
synthesis step is a class transformation function that the DSU system
executes on heap objects in the old version executable that transforms
them to objects in the new class that are type correct and
semantically consistent with the new version executable.

We demonstrate our approach on a variety of updates from
widely used open-source Java programs and derived microbenchmarks. We show
that \TOS produces correct functions for a large variety of updates,
e.g., simple object field additions, string partitioning, partitioning
a collection based on a predicate, and deleting objects that leak
based on a predicate. Furthermore, this work is the first to show that
DSU systems can correct errors such as memory leaks at update time
using class-based transformation functions.

In summary, our contributions are as follows:
\begin{enumerate}
\item An object matching method to produce input-output examples
  that act as the specification for object transformation synthesis.
\item A method for synthesizing object transformers from input-output
  examples produced by matching.
\item A method of recursively interleaving the matching and synthesis
  steps to find composite object (e.g., collections) transformations.
\item The application of these techniques to actual dynamic updates,
  including those to undo the effects memory leaks, an application not
  previously explored in the literature.
\end{enumerate}

This paper thus shows a significant step towards completely automating
DSU systems by automatically generating data transformation functions,
in addition to the code transformation functions, both of which are
required to keep programs executing while fixing bugs and add features
to them.

%%% Local Variables: 
%%% mode: latex
%%% TeX-master: "paper"
%%% End: 
