% $Id: intro.tex 9934 2009-03-31 09:49:43Z mwh657 $

\section{Introduction}

Software is imperfect.  To fix bugs and adapt software to user
demands, developers must modify deployed systems.  However, halting a
software system to apply updates creates new problems: safety concerns
for mission-critical and transportation systems; substantial revenue
losses for businesses~\cite{gartner98downtime,roc1}; maintenance
costs~\cite{zorn05}; and at the least, inconvenience for users. These
problems may translate into serious security risks if patches are not
applied promptly~\cite{altekar05opus,ksplice}.  

Dynamic software updating (DSU) is a general-purpose mechanism that
solves these problems by updating programs while they run without a
special software architecture or redundant
hardware~\cite{kspliceslashdot08}. A practical DSU system must be safe,
flexible, and efficient.  \emph{Safe} updates insure the program is as
correct as deploying it from scratch.  The update model would ideally
be \emph{flexible} enough to support all software changes, but DSU is
still useful if the system supports most software updates.  Since
updates should be rare events, an \emph{efficient} DSU system would
ideally impose no 
runtime overhead during steady-state program execution.

Researchers have made significant strides toward making DSU practical
for systems written in C or C++, supporting server feature
upgrades~\cite{neamtiu06dsu,chen:icse07,upstare}, security
patches~\cite{altekar05opus}, and operating systems
upgrades~\cite{K42reconfig,k42usenix,baumann07reboots,dynamos_eurosys_07,chen06vee,ksplice}.
Enterprise systems and embedded systems---including safety-critical
applications---are increasingly written in managed languages, such as
Java, Ruby, and C\#.  Unfortunately, work on DSU for managed languages
lags behind work for C and C++.  For example, while the HotSpot
JVM~\cite{JVMhotswap} and some .NET languages~\cite{VSEnC} support
on-the-fly method body updates, this support is too inflexible for all
but the simplest updates---limiting changes to method bodies would
support less than half of the updates
to our three Java benchmark programs.  Academic
approaches~\cite{ritzau00dynamic,Mala00a,orso:java,bierman08upgradej} offer more
flexibility, but have not been proven on realistic applications.
Furthermore, they employ method
and object indirection to make applications DSU capable, imposing substantial space
and time overheads on steady-state execution.

This paper presents the design of \DSU, a DSU system for Java, which
we implement in \JikesRVM, a Java Virtual Machine. \DSU's combination
of flexibility, safety, and efficiency is a clear advance over prior
approaches.  The 
paper's key contribution is to show how to extend and integrate
existing VM services to support DSU that is flexible enough to support
a large class of updates, guarantees type-safety, and imposes no space
or time overheads on steady-state execution.
% \DSU{} is built on 
% \JikesRVM, a Java research virtual machine.
% \suriya{Say that Jvolve is built on top of JikesRVM. Reviewer 4 wanted us
% to say this earlier}

\DSU{} DSU supports many common updates. Users can add, delete, and
change existing classes.  Changes may add or remove fields and methods,
replace existing ones, and change type signatures.  Changes may occur
at any level of the class hierarchy.  To initialize new fields and
update existing ones, \DSU{} applies \emph{class} and \emph{object
  transformer} functions, the former for static fields and the latter
for object instance fields.  The system automatically generates
default transformers, which initialize new and changed fields to
default values and retain values of unchanged fields.  If needed,
programmers may customize the default transformers.

%MWH: Suriya asserted that we don't rely on bytecode verification.
%Are you saying that we don't actually verify the classes we load in?
%If we do, we get assurances of type safety, right?
% bytecode verification is a specific term to describe a technology that
% JVMs use to assure the safety/correctness of the loaded bytecode.
% JikesRVM does not have a bytecode verifier.
\DSU{} relies on bytecode verification to statically type-check
% ISSUE: no bytecode verification in JikesRVM
updated classes.  To avoid type errors due to the timing of an
update~\cite{StoyleHBSN06,neamtiu06dsu,k42usenix}, \DSU{} stops the executing
threads at a \emph{DSU safe point} and then applies the update. DSU
safe points are a subset of VM \emph{safe points}, where it is safe to perform garbage collection (GC) and thread scheduling.  DSU safe points further
restrict the methods that may be on each thread's stack, depending on
the update.  These methods include (1) updated methods,
(2) methods that refer to updated classes (since their machine code 
may contain hard-coded offsets that the update changes), and (3) user-specified methods
as needed for safety~\cite{Gupta94,neamtiu08context}.  \DSU{} installs return
barriers~\cite{return-barrier} on these methods to inform the run-time system when
a running method returns, to speed up reaching a safe point.  \DSU{}
also applies on-stack replacement (OSR) to recompile methods in the
second category even as they run, as long as they do not contain inlined
updated methods.  This approach does not guarantee \DSU{} will reach a DSU safe point, but in our
multithreaded benchmarks it does in nearly all cases. More sophisticated
thread scheduling support could attain greater
flexibility~\cite{neamtiu09stump}. 

\DSU{} makes use of the garbage collector and JIT compiler to
efficiently update the code and data associated with changed classes.
\DSU{} initiates a whole-heap GC to find existing object
instances of updated classes and initialize new versions using the
provided object transformers.  \DSU{} invalidates existing compiled 
code and installs new bytecode for all changed method implementations.
When the application resumes execution these methods
are JIT-compiled when they are next invoked.
The adaptive compilation system naturally optimizes updated methods
further if they execute frequently.

\DSU{} imposes no overhead during steady-state execution.  During an
update, it imposes overheads in 
classloading and garbage collection. After an update, the adaptive
compilation system will incrementally optimize the updated code in its
usual fashion.  Eventually, the code is fully optimized and running
with no additional overhead.  The zero overhead in steady-state
execution for a
VM-based approach is in contrast to DSU techniques for C and C++. These
approaches
must use a compiler or dynamic rewriter to insert levels of
indirection~\cite{neamtiu06dsu, orso:java} or
trampolines~\cite{chen06vee,chen:icse07,altekar05opus,ksplice}, which
add a constant overhead during normal execution.
% Garbage collection also
% provides substantial flexibility.  Rather than padding objects in case they
% become larger due to an update~\cite{neamtiu06dsu}, the collector
% copies changed objects, grows them only if necessary, and thus
% supports a richer set of updates.

We assessed \DSU{} by applying updates corresponding to
one to two years' worth of releases of three open-source
multithreaded applications: Jetty web server,
JavaEmailServer (an SMTP and POP server), and CrossFTP server.  We
found that \DSU{}
can successfully apply 20 of the 22 updates---the two updates it does
not support change a method within an infinite loop that is always on the
stack.
% REFACTOR: add sentence here about the Jetty refactoring, if we can get
% that to work.
Microbenchmark results show that the pause time due to an
update depends on the size of the heap and fraction of transformed
objects. 
% Compared to the highly-optimized copy operation the GC would
% normally use, our compilation of user-provided object transformers is
% significantly slower, which in the limit could incur a high cost.
% However, most updates transform only a small fraction of heap objects
% and this cost is incurred only during the update itself.
Experiments
with Jetty show that applications updated by \DSU{} enjoy the same
steady-state performance as if started from scratch.

In summary, this paper's main contributions are (1) techniques to extend and
integrate standard virtual machine services for managed languages to
support flexible, safe, and efficient dynamic software updating
services, (2) the design, implementation, and evaluation of \DSU, a
Java VM with support for dynamic software updating. \DSU{} is
distinguished from prior work in its realism, flexibility, technical
novelty, and high performance.  We believe our demonstration is a
significant step towards supporting flexible, efficient, and safe
updates in managed code virtual machines.
%% The remainder of the paper first
%% compares our approach carefully to related work, and then describes
%% our model of updates, our implementation, and experimental results.

%%% Local Variables: 
%%% mode: latex
%%% TeX-master: "pldi64"
%%% End: 
