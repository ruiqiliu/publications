\section{Related Work}
\label{sec:related}
We compare our VM-centric approach to DSU with related work on
implementing DSU for managed languages, C, and C++.

\paragraph{Edit and continue.}

Debuggers and IDEs have long provided \emph{edit and continue} (E\&C)
functionality that permits limited modifications to program state to
avoid stopping and restarting during debugging. For example, Sun's
HotSwap VM~\cite{JVMhotswap,Dmit01a}, .NET Visual Studio for C\# and
C++~\cite{VSEnC}, and library-based support~\cite{eaddy05enc} for .NET
applications all provide E\&C.  These systems restrict updates to
code changes within method bodies.  While this restriction reduces safety
concerns and obviates the need for class and object transformers, the resulting
systems are inflexible. They cannot perform more than half of the updates discussed in
Section~\ref{sec:experience}.

\paragraph{DSU for managed languages without VM support. } To avoid
changing the VM to support DSU, researchers have developed special-purpose
classloaders and/or compiler support.  The main drawbacks of these
approaches are inflexibility and high overhead.
For example, Eisenbach and Barr~\cite{BarrE03} and Milazzo et
al.~\cite{Milazzo05updates} use custom classloaders for
binary-compatible and component-level changes respectively, but
cannot perform class field additions.
% The former targets libraries and the latter is part of the design of a
% special-purpose software architecture.

%% Eisenbach and Barr: safe upgrading without restarting.  They support
%% upgrades that satisfy binary compatibility.  Uses a custom classloader and
%% JMX to replace the code of existing objects.  No way to modify the state of
%% the objects.

%% Milazzo use a modified class loader to load individual replacements to
%% classes in a special-purpose architecture.  The class loader may modify the
%% bytecode of the loaded class to deal with type version namespace issues.
%% Basically this is more limited in scope than our approach.

Orso et al.~\cite{orso:java} use source-to-source translation for DSU
by introducing a proxy object that indirectly accesses
an object that may change.  
%% For each class C that might change in the future they produce a proxy for
%% that class.  All calls from clients of C are redirected to call the wrapper
%% instead.  When C is updated by some new class C', a new C' object is created
%% and initialized using the old state of C and the wrapper is redirected to
%% point to C'.
This approach requires updated classes to export the same public
interface, forbidding new public methods and
fields.
Non VM-based approaches are in general limiting because they are
not \emph{transparent}---they make visible changes to the class hierarchy,
and insert or rename classes.  This approach makes it essentially
impossible to be robust in the face of code using reflection or native
methods.  Moreover, the indirection imposes time and space overheads
on steady-state execution.  Our VM approach
naturally supports reflection and native methods (these are updated
as well), and is more expressive, e.g., it supports signature changes.

\paragraph{VM support for DSU in managed languages.} 
The PROSE system performs short-term, run-time patches to code for
logging, introspection, and performance adaptation, rather performing
general updates~\cite{nicoara:eurosys08}.  An Eclipse plug-in performs
run-time bytecode instrumentation and a modified JIT performs method
code replacement, using an API in the style of aspect-oriented
programming.  % in support of run-time software
PROSE has the same update model as the E\&C systems: it supports updates to method bodies but not
class or method signature changes that require changes to object state. 
% This flexibility is
% similar to the EnC implementations discussed above; 
% indeed, PROSE builds on the HotSwap method replacement support in its
% Sun JDK implementation~\cite{JVMhotswap}.

JDrums~\cite{ritzau00dynamic} and Dynamic Virtual Machine
(DVM)~\cite{Mala00a} both implement DSU for Java within the VM,
providing a programming interface similar to \DSU, but are lacking in
two ways.  First, neither JDrums nor DVM have ever been demonstrated
to support updates from real-world applications.  Second,
their implementations impose overheads during steady-state execution.
They both update \emph{lazily} and use
an extra level of indirection (the \emph{handle
  space}).  Indirection conveniently supports
object updates, but adds extra overhead.  For example, JDrums traps
all object pointer dereferences to apply VM object transformer
function(s) when the object's class changes.  Lazy updating has the
advantage that it amortizes  pauses due to an update over
subsequent execution.  The main drawback is that its overhead persists
during normal execution, even though updates are relatively rare.  DVM
works only with the interpreter.  Relative to this interpreter, which
is already slow, the extra traps result in roughly 10\% overhead.

Compared to these two, \DSU{} performs updates eagerly by employing a
full heap collection at update-time.  This stop-the-world approach imposes a longer pause at
update time, but eliminates overhead during steady-state execution.
Likewise, by invalidating updated methods, \DSU's performance is
slowed just after the update as these methods are being recompiled.
However, compared to running with an interpreter, steady-state
execution is much improved, since methods will be much better
optimized.

%% JDRUMS: implements the conversion lazily.  They have a similar interface
%% (object and class transformers). The drawback here is that there is overhead
%% in the general case of execution---we do not know when the update is
%% complete.  Implemented in Sun's JDK 1.2.  Adds a level of indirection to the
%% new object.  Thus overhead builds up over time.  It also appears they have a
%% more limited interface to what can be referenced in a conversion function.
%% For example, there is no way to refer to fields other than those of the
%% object's class (i.e., no super-class fields) and there is no way to call new
%% methods, like constructors.  Not clear if there are restrictions on how
%% methods can be changed.

%% DVM: use an incremental mark-sweep collector, where mark phase marks objects
%% to be updated and the sweep phase incrementally updates them (prior to being
%% accessed by the mutator).  Like JDRUMS, all accesses to marked objects are
%% trapped.  Imposes a stock 10\% overhead, even only using bytecode.

%% Both of these: no significant experience with real applications, according
%% to how they change in practice.  They also can't handle native methods
%% because they can't trap access to modified objects.  Doing the full GC
%% solves this problem.

% More recently, Nicoara et al. developed PROSE, a system for run-time

Boyapati et al.~\cite{boyapati03lazy} support dynamic updates to
classes kept in a \emph{persistent object store} (POS).  While the
setting is different, their basic update model, and in particular
their notion of object transformer function, is similar to ours.  In
their system, programmers manually write an object transformer
that they view as a method on the old version of the updated class, i.e., the transformer method is type-safe with respect to the old class.
In \DSU, object transformers may access the \emph{new} versions of
objects pointed to from the old class.  Instead, Boyapati et al.'s
transformers may access the \emph{old} versions.  To implement this
model, they rely on
\emph{encapsulation} based on \emph{ownership types}: if an object $a$ of class $A$ has an ``owned''
field pointing to an object $b$ of class $B$, then only $a$ can point
to (and access) $b$.  Encapsulation thus ensures the system will always transform $a$ before $b$, which makes the transformation
algorithm more efficient.  They rely on the programmer to 
enforce encapsulation, and describe how the compiler could automate language support for encapsulation in a non-standard type system.  
\DSU{} takes the opposite tack of forcing old object fields to point
to up-to-date objects, and thus requires no special language support.
Moreover, \DSU's model follows that of earlier
work~\cite{k42usenix,neamtiu06dsu,neamtiu09stump,upstare} which has
proven its effectiveness on a half-dozen realistic applications
across several years' worth of releases.  However, further
research to understand the costs and benefits of the two updating
models would be useful.

Boyapati et al. also differs from \DSU{} in that, like JDRUMS and DVM,
updates are applied incrementally as objects are accessed following
an update rather than all at once using a stop-the-world GC\@.  This
incremental cost is more natural in a POS since indirection is
already required to access external objects.  The POS model also
permits programmers to specify ACID transaction boundaries, which can
help ensure that updates are applied consistently and safely.  In contrast, our work focuses on 
supporting dynamic upgrades in a high-performance VM for Java, and thus many of
the issues we consider---reaching a safe point via return barriers and
OSR, and coexisting with the JIT compiler---are the unique
contributions of our work.

% \paragraph*{Approaches using a standard VM.}
% \suriya{another
% awkward title. Should ask Kathryn what it actually is}
% 

\paragraph{Dynamic Software Updating for C/C++}

There are several substantial systems for dynamically updating C and
C++ programs that target server
applications~\cite{HjalmtyssonG98,altekar05opus,neamtiu06dsu,chen:icse07,upstare,neamtiu09stump}
and operating systems
components~\cite{K42reconfig,k42usenix,chen06vee,lee06linuxmod,dynamos_eurosys_07}.
% These systems are more mature than most of the systems described above, in
% some cases with substantial updating experience.  The flexibility
Although some of these systems are mature, the flexibility
afforded by \DSU{} is comparable or superior.  \DSU's timing
restrictions and Java's type safety also provide comparable or
superior safety; the fact that C and C++ programs often circumvent the
languages' weak type systems greatly complicates efforts 
to ensure that updates behave correctly.  Some prior
systems~\cite{neamtiu09stump,upstare,chen:icse07} have focused 
on means to reach DSU safe points quickly, and we plan similar
efforts as future work.  In particular, we plan to extend our support
for OSR to apply to running methods whose bytecode has changed,
allowing the user to map an active method's PC and
stack frame and those of its new version, similar to support provided by
UpStare~\cite{upstare}.

The lack of a VM is a disadvantage for C and C++ DSU.  For
example, because a VM-based JIT can compile and recompile replacement
classes, it imposes no steady-state execution overhead.  By contrast,
C and C++ implementations must use either statically-inserted
indirections~\cite{HjalmtyssonG98,neamtiu06dsu,K42reconfig,k42usenix,upstare}
or dynamically-inserted trampolines to redirect function
calls~\cite{altekar05opus,chen06vee,chen:icse07,ksplice}.  Both cases
impose persistent overhead on normal execution and inhibit
optimization.  Likewise, because these systems lack a garbage
collector, they either do not update object instances at
all~\cite{ksplice}, update them lazily~\cite{neamtiu06dsu,chen:icse07}
or perform extra allocation and bookkeeping to locate the
objects at update-time~\cite{k42usenix}.  Finally, because these
systems lack support for on-stack replacement, they must pre-compile
potentially long-running methods specially, so that they can be
updated while they run.   These techniques impose time and space overheads on
steady-state execution, and in some cases limit update flexibility.

\paragraph*{Other proposals}

Gilmore et al.~\cite{GilmoreKW97} propose DSU support for modules in
ML programs using a similar, but more restrictive programming interface
compared with \DSU{}.  They formalize an abstract machine for
implementing updates using a copying garbage collector.
Duggan~\cite{DBLP:journals/acta/Duggan05} also proposes dynamic
updates to ML programs, focusing on lazy updates to data type
definitions.  Neither approach was ever implemented.

UpgradeJ~\cite{bierman08upgradej} is an extension to
the Java language design supporting class upgrades, in two flavors:
\emph{revision upgrades}, which may modify method bodies, and
\emph{evolution upgrades}, which may add new methods and fields.
Programmers control the effects of upgrades using \emph{version
annotations}, introduced by Bierman et al.~\cite{BiermanHSS03} in
earlier work.  For 
example, the programmer may write \texttt{o = new 
Button[1=]()} to force \texttt{o} to always use version~1 methods,
while writing \texttt{p = new Button[1+]()} or \texttt{p = new
  Button[1++]()} allows \texttt{p} to be revised or evolved,
respectively.  UpgradeJ's update model is easier to implement than
\DSU's because it need not change existing object instances.  Of
course, the downside is a loss of flexibility.  Many of the updates to
our benchmark applications change field contents and layout.  UpgradeJ does not support these updates.  On the other hand,
evolution upgrades add power over simple method
body updates, and consequently enable more real-world updates to be
supported~\cite{tempero08upgradej}.  There is no implementation of
UpgradeJ.






%% basic swapping for single modules.


%%% Local Variables: 
%%% mode: latex
%%% TeX-master: "pldi64"
%%% End: 
