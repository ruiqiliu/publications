\subsection*{Abstract}
Software evolves to fix bugs and add features. Stopping and restarting
programs to apply changes is inconvenient and often costly.  Dynamic
software updating (DSU) addresses this problem by updating programs
while they execute, but existing DSU systems for managed languages do
not support many updates that occur in practice and are inefficient.
This paper presents the design and implementation of \DSU, a
DSU-enhanced Java VM.  Updated programs may add, delete, and replace fields
and methods anywhere within the class hierarchy.  \DSU{} implements
these updates by adding to and coordinating VM classloading,
just-in-time compilation, scheduling, return barriers, on-stack
replacement, and garbage collection.  \DSU{} is \emph{safe}: its use
of bytecode verification and VM thread synchronization ensures that an
% ISSUE: no bytecode verification in JikesRVM
update will always produce type-correct executions. \DSU{} is
\emph{flexible}: it can support 20 of 22 updates to three
open-source programs---Jetty web server, JavaEmailServer, and CrossFTP
server---based on actual releases occurring over 1 to 2 years.
%REFACTOR: though slight refactorings would allow the other updates.
\DSU{} is \emph{efficient}: performance experiments show that
\DSU{} incurs \emph{no overhead} during steady-state execution.  These
results demonstrate that this work is a significant step towards
practical support for 
dynamic updates in virtual machines for managed languages.

%%% Local Variables: 
%%% mode: latex
%%% TeX-master: "pldi64"
%%% End: 
