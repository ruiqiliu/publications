\begin{table}[t]
\begin{footnotesize}
\begin{center}
\begin{tabular}{|l||c|c||c|c|c|r|c|c|} \hline \T
Ver.   & \multicolumn{2}{c||}{\# classes} &    \multicolumn{6}{c|}{\# changed}             \\
       & add & del & classes & \multicolumn{3}{c|}{methods} & \multicolumn{2}{c|}{fields}  \\
       &     &     &         & add & del & chg   & add & del                               \\ \hline \hline \T
1.06   & 4   & 1   & 1       & 0   & 0   & 3/0   & 1   & 0                                 \\
1.07   & 0   & 0   & 3       & 4   & 0   & 14/0  & 5   & 0                                 \\
1.08   & 0   & 1   & 3       & 2   & 0   & 10/0  & 0   & 2                                 \\ \hline
\end{tabular}
\end{center}
\end{footnotesize}
\caption{Summary of updates to CrossFTP server}
\label{tab:crossftp-changes}
\end{table}

\subsection{CrossFTP server}
\label{subsec:crossftp}
CrossFTP server is an easily configurable, security-enabled FTP server.
CrossFTP allows simple configuration through a GUI and more advanced
customization using configuration files. We did not use the GUI interface
and therefore do not consider changes to that part of the program.  We
looked at 4 versions--- 1.05 through 1.08, details shown in Table~\ref{tab:crossftp-changes}---spanning a duration of more
than a year. Version 1.08 contains about 18,000 lines of code spread across
161 classes. \DSU{} successfully applies all three updates to this
application.  Note that since all updates either add or delete fields,
simple method body updating support on its own would be insufficient.

\DSU{} could only apply the update from version 1.07 to 1.08 when the
server was relatively idle.
%MWH: I added relatively here, since the update should work when there
%is one connection, by our explanation: i.e., one thread running
%RequestHandler.run() will hit its return barrier and the update takes
%place.  Maybe two connections too, if you get lucky, etc.
The server runs a new {\tt RequestHandler} thread to process each FTP
session, and the \texttt{RequestHandler.run()} method was changed by
the update.   \DSU{} installs a return barrier on this method,
but with many active sessions, this method is essentially always on stack,
preventing the update.  Future work could address this problem using scheduler support
for coordinating updates among active threads~\cite{neamtiu09stump}.
