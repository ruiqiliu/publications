\subsection{JavaEmailServer}
\label{subsec:jes}

\begin{table}[t]
\begin{footnotesize}
\begin{center}
\begin{tabular}{|l||c|c||c|r|c|r|r|c|} \hline \T
Ver.   & \multicolumn{2}{c||}{\# classes} &    \multicolumn{6}{c|}{\# changed} \\
       & add & del & classes & \multicolumn{3}{c|}{methods} & \multicolumn{2}{c|}{fields} \\
       &     &     &         & add & del & chg   & add & del \\ \hline \hline \T
1.2.2  & 0   & 0   & 3       & 0   & 0   & 3/0   & 0   & 0   \\
1.2.3  & 0   & 0   & 7       & 0   & 0   & 14/2  & 12  & 0   \\
1.2.4  & 0   & 0   & 2       & 0   & 0   & 4/0   & 0   & 0   \\
1.3*   & 4   & 9   & 2       & 11  & 3   & 6/9   & 12  & 5   \\
1.3.1  & 0   & 0   & 2       & 0   & 0   & 4/0   & 0   & 0   \\
1.3.2  & 0   & 0   & 8       & 4   & 2   & 4/2   & 3   & 1   \\
1.3.3  & 0   & 0   & 4       & 0   & 0   & 3/0   & 0   & 0   \\
1.3.4  & 0   & 0   & 6       & 2   & 0   & 6/0   & 2   & 0   \\
1.4    & 0   & 0   & 7       & 6   & 1   & 4/1   & 6   & 0   \\ \hline
\end{tabular}
\end{center}
\end{footnotesize}
\caption{Summary of updates to JavaEmailServer}
\label{tab:jes-changes}
\end{table}

For JavaEmailServer, we considered 10 versions---1.2.1 through
1.4---spanning a duration of about two years.   Version 1.4 
consists of 20 classes and about 5000 lines of code. 
Table~\ref{tab:jes-changes} shows the summary of changes for each new
version. Approaches that only support updates to method bodies will be able
to handle only four of these updates. We
could successfully construct updates for all versions we examined, and
we could successfully apply all of them except the update to version 1.3.
This update reworks the configuration framework of the server, among other
things removing a GUI tool for user administration and adding several
new classes that implement a file-based configuration system.  As a result, many of
the classes are modified to point to a new configuration object.
Among these classes are threads with infinite processing loops (e.g., to accept
POP and SMTP requests). Because these threads are always active, the
safety condition can never be met and thus the update cannot be
applied.

The update from 1.3.1 to 1.3.2 indirectly changes the
\texttt{SMTP\-Sen\-d\-er.run()} and \texttt{Pop3\-Processor.run()} methods.  These
methods contain processing loops run by several threads.  Though these
methods are always running, \DSU{} applies OSR and the update succeeds.
\DSU{} also uses OSR for the update from 1.3.2 to 1.3.3.
%MWH: suriya says this statement just isn't true.  Updates fine.
% A method used to process connections was also changed, and this method
% is frequently active when the server is under load.  (A similar
% method was changed in the update to version 1.3.3.)  A return barrier
% installed on this method permits the update to take effect.  
