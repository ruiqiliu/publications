
\newcommand{\ChangedClassesColumn}{third}
\begin{table}
\begin{footnotesize}
\begin{center}
\begin{tabular}{|l||c||c|c|c|r|c|c|} \hline \T
Ver.    & \#      & \multicolumn{6}{c|}{\# changed} \\
        & classes & \multicolumn{1}{c|}{classes} & \multicolumn{3}{c|}{methods} & \multicolumn{2}{c|}{fields} \\
        & added   &         & add & del & \multicolumn{1}{c|}{chg}             & add & del \\ \hline \hline \T
5.1.1   & 0       & 14      &\ 4  & 1   &  38/0            &  0  & 0   \\
5.1.2   & 1       &\ 5      &\ 0  & 0   &  12/1            &  0  & 0   \\
5.1.3*  & 3       & 15      & 19  & 2   &  59/0            & 10  & 1   \\
5.1.4   & 0       &\ 6      &\ 0  & 4   &   9/6            &  0  & 2   \\
5.1.5   & 0       & 54      & 21  & 4   & 112/8            &  5  & 0   \\
5.1.6   & 0       &\ 4      &\ 0  & 0   &  20/0            &  5  & 6   \\
5.1.7   & 0       &\ 7      &\ 8  & 0   &  11/2            &  9  & 3   \\
5.1.8   & 0       &\ 1      &\ 0  & 0   &   1/0            &  0  & 0   \\
5.1.9   & 0       &\ 1      &\ 0  & 0   &   1/0            &  0  & 0   \\
5.1.10  & 0       &\ 4      &\ 0  & 0   &   4/0            &  0  & 0   \\ \hline
\end{tabular}
\end{center}
\end{footnotesize}
\caption{Summary of updates to Jetty}
\label{tab:jetty-changes}
\end{table}

\subsection{Jetty webserver}
\label{subsec:jetty}

Jetty is a popular webserver written in Java. It supports static
and dynamic content and can be embedded 
within other Java applications. \JikesRVM, and thus \DSU, is not able to run the
most recent versions of Jetty (6.x).  Therefore we considered 11
versions, consisting of 5.1.0 through 5.1.10 (the last version prior to
version 6).  Version 5.1.10 contains 317 classes and about 45,000 lines
of code.  Table~\ref{tab:jetty-changes} shows a summary of the changes
in each update.  Each row tabulates the changes relative
to the prior version. For the column listing changed methods, the
notation $x/y$ indicates that $x+y$ methods were changed, where $x$
changed in body only, and $y$ changed their type signature as well.
For dynamic updating systems that only support changes to method
bodies, only the first and last three of the ten updates could be
supported, since the rest either change method signatures
and/or add or delete fields.

\paragraph{Reaching a safe point in Jetty.}
We  successfully wrote dynamic updates to all
versions of Jetty that we examined. For each version starting at 5.1.0, we ran
Jetty under full load. After 30 seconds we tried to apply the update to the
next version. We did this five times per version.  Other than the
update to 5.1.3, all versions immediately 
reached a safe point every time, with no need of return barriers.

We could not apply the update to version 5.1.3 (denoted with an
asterisk in the table) 
because \DSU{} was never able to reach a safe point. The update modified
{\tt ThreadedServer.accept\-Socket()}, a method that waits for a connection
from the client, and this method is nearly always on stack. 
We installed a return barrier that is triggered when {\tt
  acceptSocket} returns, but this barrier is not sufficient to perform the
update since the main methods of several threads were themselves
modified. For example, we also install a return barrier on {\tt
  Pool\-Thread.run()}, but this barrier is never triggered because
this method never becomes inactive.

% REFACTOR: add para back if we get things to work.
% We refactored the various long-running main methods in versions 5.1.2 and
% 5.1.3 to extract the modified bodies of long running methods into separate
% methods and leave the main method containing the loop unmodified between
% the two versions.  (This sort of transformation is performed by other
% systems automatically by programmer directive in
% Ginseng~\cite{neamtiu06dsu}.) 
% When we attempted to perform a dynamic update, \DSU{}
% installed a return barrier for {\tt acceptSocket()}. This function waits
% for a connection and returns after a timeout. \DSU{} was able to update the
% application when this function returned. 

% To understand why, we instrumented the VM to emit information about
% restricted method set and, if a safe point cannot be reached, which
% restricted method was active. 

% \suriya{The following information about inlining is commented out. What
% should we do with this? See .tex file}
%     For each version, starting at 5.1.0, we
%     ran Jetty under full load.  After 30 seconds we tried to apply the
%     update to the next version; if a safe point could not be immediately
%     reached, we deemed the attempt as failed.
%     The results are presented in
%     Table~\ref{tab:inlining}.  Column 2 shows the number of times out of
%     five such runs where the application reached a safe point.  The methods
%     whose presence on a thread stack precluded the application from
%     reaching a safe point are mentioned below the table.  For the update to
%     5.1.3, the offending method was always active because it contained an
%     infinite loop.  The other updates either always succeeded, or did after a
%     small number of retries.
%     
%     Column 3 contains the total number
%     of methods in the program at runtime, where the number in parentheses is the
%     number of those which the compiler inlined when using aggressive
%     optimization.
%     This provides an upper bound on the effect of inlining in reaching a
%     safe point.  The next group of columns contains the restricted method
%     set. Each column in the group specifies the number of methods loaded at
%     run time by the VM, followed by the total number of methods in that
%     category in the program. The first column in this group is the number of methods in
%     classes involved in a class update. Recall that when a class is updated, say by adding
%     a field, all its methods are considered restricted (see section~\ref{sec:safe}).
%     \suriya{Provide reference to where we explain restricted. and redefine
%     restricted}
%     The second column in this group is the number of methods whose
%     bodies are updated, the third is the number of methods
%     indirectly updated, and the fourth sums these, with the number of methods
%     that were inlined written in parentheses. % The first and second
%     % columns in this group together constitute all the methods of classes that
%     % were changed, as enumerated in the \ChangedClassesColumn{} column in table~\ref{tab:jetty-changes}.
%     The final two columns list
%     the total number of methods in the restricted set; they differ from the
%     first number in the fourth column by the number of (transitively)
%     inlined callers of the restricted methods that were not already
%     restricted.  The final column uses \DSU{}'s OSR analysis to determine the
%     number of restricted methods although OSR itself was not applied.
%     
%     The table shows that both indirect method calls \suriya{or updates?} and inlining significantly
%     add to the size of the restricted set.  Inlining though, is small by
%     comparison, because all callers of an updated class's methods are
%     \emph{already} included in the indirect set. Therefore, inlining these
%     methods adds no further restriction.
%     In most cases OSR support would dramatically reduce the number of
%     restricted methods and increase the likelihood of reaching a DSU safe
%     point.
%     Interestingly, having a greater number of restricted methods overall
%     does not necessarily reduce the likelihood that an update will take
%     effect; rather, it depends on the frequency with which methods in this
%     set are on the stack.

%%% Local Variables: 
%%% mode: latex
%%% TeX-master: "../pldi64"
%%% End: 
