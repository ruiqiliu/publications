\subsection{Class and Object Transformers}
\label{subsec:transformers}

For classes whose signatures have changed, an object
transformer\index{object transformer} method  initializes a new version of
the object based on the old version. For example, consider the class
{\tt Point} in Figure~\ref{fig:point}. The {\tt Point} class in the old
version represents points in a 2-dimensional space with fields {\tt x} and
{\tt y}. After the update, the new version represents points in a
3-dimensional space with the additional field {\tt z}. The object transformer's
job is to modify each object instance of type {\tt Point} in the heap to conform to its
new class definition. Class transformers serve a similar purpose and update
static fields, rather than instance fields.  The \ac{UPT} generates default
class and object transformers automatically, retaining unchanged fields and
initializing new or changed ones.  The default object transformer, show in
Figure~\ref{fig:point}~(c)  for our changed \texttt{Point} class copies
fields {\tt x} and {\tt y} from an old object to a transformed object and
initializes {\tt z} to zero.

\begin{figure}[t]
\lstset{frame=single}
\begin{tabular}{@{}m{0.5\textwidth}@{}m{0.5\textwidth}@{}}
\BC \begin{minipage}{0.25\textwidth}
\begin{lstlisting}
class Point {
  double x, y;
}
\end{lstlisting}
\end{minipage} \EC &
\BC \begin{minipage}{0.29\textwidth}
\begin{lstlisting}
class Point {
  double x, y, z;
}
\end{lstlisting}
\end{minipage} \EC \\[-2ex]
\BC (a) Old version \EC &
\BC (b) New version \EC \\[-2ex]
\end{tabular}
\centering\begin{minipage}{0.55\textwidth}
\begin{lstlisting}
class JvolveTransformers {
  public static void jvolveObject(
    Point to, old_Point from) {
    to.x = from.x;
    to.y = from.y;
    to.z = 0.0;
  }
}
\end{lstlisting}
(c) Default UPT-generated object transformer
\end{minipage}
\caption{Example of a simple class and the default object transformer}
\label{fig:point}
\lstset{frame=none}
\end{figure}


% For example, consider a class \texttt{List} with field
% \texttt{next} of type \texttt{List} and an update that adds a new
% \texttt{int} field {\tt x} to \texttt{List}. The object transformer's
% job is to modify each object instance of type \texttt{List} to conform
% to its new class definition. Class transformers serve a similar
% purpose and update static fields, rather than instance
% fields.  The \ac{UPT} generates default class and object transformers
% automatically, retaining unchanged fields and initializing new or
% changed ones.  The default object transformer for our changed
% \texttt{List} copies the \texttt{next} field from an old object to a
% transformed object and initializes {\tt x} to zero, i.e,
% \texttt{transformed.next = old.next} and \texttt{transformed.x = 0}.

\begin{figure}[t]
\centering
\begin{minipage}{0.9\textwidth}
\begin{lstlisting}[frame=single]
public class v131_User {
  private final String username, domain, password;
  private String[] forwardAddresses;
}
public class JvolveTransformers {
 ...
 public static void jvolveClass(User unused) {}
 public static void jvolveObject(User to, v131_User from) {
    to.username = from.username;
    to.domain = from.domain;
    to.password = from.password;
    // default transformer would have:
    //   to.forwardAddresses = null
    int len = from.forwardAddresses.length;
    to.forwardAddresses = new EmailAddress[len];
    for (int i = 0; i < len; i++) {
      String[] parts =
        from.forwardAddresses[i].split("@", 2);
      to.forwardAddresses[i] =
        new EmailAddress(parts[0], parts[1]);
    }
  }
}
\end{lstlisting}
\end{minipage}
\hangcaption{\User object transformer for update from JavaEmailServer version
1.3.1 to version 1.3.2}
\label{fig:jes-transformer-code}
\VspaceFixForHangcaption
\end{figure}


For our example from JavaEmailServer in
Figure~\ref{fig:jes-string-emailaddress-example}, the \UPT identifies that
the \User and {\tt ConfigurationManager} classes have changed, and produces
default object transformers.  The programmer elects to modify the object
transformer for the class \User, as illustrated in
Figure~\ref{fig:jes-transformer-code}.
  
Object and class transformer methods are simply \static methods in the
class \JT, which is created by \UPT and loaded by \JV at update time. The
class transformer method \JC takes an instance of the new class as a dummy
argument. Standard overloading in Java distinguishes the \JC methods for
different classes. (In our example, \JC does nothing.) The object
transformer method \JO takes two reference arguments: \TO, the
uninitialized new version of the object, and \FROM, the old version of
the object. We prepend a version number to the names of old classes to
distinguish them from the new versions. Based on the \UPT specification,
but before the VM loads the \JT class, the VM renames the old class in all
its internal data structures.  This renaming makes the class name space and
the \JT class type-correct. In our example, the VM renames the old version
of \User to class \oUser, which is the type of the \FROM argument to
the \JO method in the new \User class. The \oUser class contains only field
definitions from the original class; all methods have been removed since
the updated program may not call them, as discussed below.

A typical transformer initializes a new field to its default value (e.g.,
{\tt 0} for integers or \NULL for references) and copies references to
the old values.  In the example, the first three lines simply copy the
previous values of {\tt username}, {\tt domain}, and {\tt password}.  A
more interesting case is the field type change to {\tt forwardedAddresses}.
The default transformer function would initialize the {\tt
forwardedAddresses} field to \NULL because of the type change.
The customized update function in Figure~\ref{fig:jes-transformer-code}
instead allocates a new array
of {\tt EmailAddress}es and initializes them to substrings of the {\tt
String} objects from the old array.  

Because the transformer class is separate from the old and new object
classes, the Java type system would normally forbid the transformer, access
to their \private fields. There is no obvious solution to this problem
that conforms to the Java type system during an update. We could define object transformers
as methods of the old changed classes, which would grant access to the old
fields, but not the new ones.  Defining transformers as methods of the new
changed class has the reverse problem. Also, the Java type system would
disallow writes to \final fields from within the transformer
functions. A \final field is ``write once'' fields and can be written
to only in constructor methods.  To avoid these problems, we compile our
transformation class separately.  We extend the JastAdd Java-to-bytecode
extensible compiler \cite{JastAddJ} to ignore access modifiers (e.g., \private
and \protectd) and allow methods to assign to \final
fields only during an update.  Bytecode that ignores these modifiers would not normally verify.
% so we have to modify the \VM to allow it in this special
% circumstance.
\RVM, on which \JV is built, does not implement a
bytecode verifier.  Aside from this exceptional case, \JV classes are
compiled normally and would pass verification. The \VM executes these
Java functions normally, because they are otherwise standard Java. Since
the transformation class is only active and available during the update,
after the update the system no longer accesses the transformer functions.
% the \VM may delete it after transformation.
Separating transformers from
updated classes avoids cluttered class files at run-time, and makes dynamic
updating more transparent to developers.

Supported in its full generality, a transformer method may reference any
object reachable from the global (\static) namespace of both the old
and new classes, and read or write fields or call methods on the old
version of an updated object and/or any objects reachable from it. \JV
presents a more limited interface (similar to past
work~\cite{ritzau00dynamic,Mala00a}). In particular, the only access to the
new class namespace is via the \TO pointer, whose fields are
uninitialized. The old class namespace is accessible, with two caveats.
First, fields of old objects may be dereferenced, but only if the update
has not changed the object's class, or if it has, after the referenced
objects are transformed to conform to the new class definition. Second, no
methods may be called on any object whose class was updated. In
Figure~\ref{fig:jes-transformer-code} class \oUser is defined in terms of
the fields it contains; no methods are shown. As explained in
Section~\ref{sec:xformers}, these limitations stem from the goal of keeping
our garbage collector-based traversal safe and relatively simple. This
interface is sufficient to handle all of the updates we tested.
Section~\ref{sec:transformer-model} goes into detail on how our
implementation influences our model for object transformers, its
limitations and alternative approaches.

%MWH: moved/modified from 3.5
An alternative programming model would be that transformers could dereference
\FROM object fields and see the \emph{old} objects, rather than the
transformed ones.  Boyapati et al.~\cite{boyapati03lazy} implement this
model, as described in Section~\ref{sec:boyapati}. Our experience and that of
others~\cite{k42usenix,neamtiu06dsu,neamtiu09stump,upstare} indicate
that our model expresses many updates well.  We leave to future work a
detailed investigation of the semantics and expressiveness of both
models.
