\subsection{Preparing the update}
\label{sec:prep}

To determine the changed classes and methods for a given release, we wrote
\acf{UPT}. \UPT is built using jclasslib, a Java bytecode library
\cite{jclasslib}. \UPT examines differences between the old and new classes
provided by the user, and groups them into the following categories
described in Section~\ref{sec:supported-changes}.

\begin{description}
\item[Class updates:] These updates change the class signature by
  adding, removing, or changing the types of fields and methods.
\item[Method body updates:] These updates change only the internal
implementation of a method. 
\item[Indirect method updates:] These are methods whose bytecode is
unchanged, but the VM recompiles them because they refer to fields and
methods of updated classes.   The compiled code uses hard-coded field
offsets, and the update may change these offsets.
\end{description}

\ac{UPT} generates default object and class transformer functions for all
class updates, which the programmer may optionally modify. After compiling
the transformers with our custom JastAdd compiler (described in
Section~\ref{subsec:transformers}), the user initiates the update by
signaling the \JV \VM and providing the new version of the application,
the update specification file, and the transformers class file.
