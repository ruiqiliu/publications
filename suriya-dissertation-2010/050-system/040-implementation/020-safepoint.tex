\subsection{DSU safe points}
\label{sec:safe}

\JV enforces various update safety properties by restricting at what points
in the program an update can happen, called {\em DSU safe points}.  DSU
safe points occur at \emph{VM safe points} but further restrict the methods
on the threads' stacks.  These restrictions provide sensible update
semantics: no code from the new version executes before the update
completes, and no code from the old version executes afterward. These
restrictions also ensure that the update is type-safe, that the compiled
version of methods are consistent with the update, and that the update
respects program semantics.  As mentioned in Section~\ref{sec:system-intro} and
above, we
divide restricted methods into three categories: (1) methods whose bytecode
has changed; (2) methods whose bytecode has not changed but that access an
updated class; and (3) methods the user blacklists.

We next discuss why these restrictions improve the safety and semantics of
updates, and then describe the actions \JV takes to reach a DSU safe
point.

\paragraph{Semantics of DSU safe points.}

To understand why
category~(1) methods are restricted, consider the update from
Figure~\ref{fig:jes-string-emailaddress-example}.  Assume the thread is
stopped at the beginning of the {\tt ConfigurationManager.loadUser} method.
If the update takes effect at this point, the new implementation of {\tt
User.setFor\-wardedAddresses} will take an object of type {\tt
EmailAddress[]} as its argument.  However, if the old version of {\tt
loadUser} were to resume, it would still call {\tt setForwardedAddresses}
with an array of {\tt String}s, resulting in a type error.

Preventing an update until changed methods are no longer on the stack
ensures type safety because when the new version of the program resumes, it
will be self consistent.  If a programmer changes the type signature of a
method {\tt m}, for the program to compile properly, the programmer must
also change any methods that call {\tt m}.  In our example, the fact that
{\tt setForwardedAddresses} changed type necessitated changing the function
{\tt loadUser} to call it with the new type.  With this safety condition,
there is no possibility that the signature of method {\tt m} could change
and some old caller could call it---the update must also include all
updated callers of {\tt m}.
These cases must be considered by dynamic updating systems.
Our choice of restricted methods is similar to other DSU
systems~\cite{ritzau00dynamic, Mala00a, altekar05opus, eaddy05enc,
JVMhotswap, VSEnC, chen:icse07, K42reconfig}.

\paragraph{Ensuring type-safety}

% As mentioned above,
\JV does not allow methods with changed bytecodes to
be active on stack when performing the update. This restriction by itself
ensures type-safety of \JV's update process. Let us consider the set of
all method bodies in the old and new versions of the application. Some
methods exist in the old version and not the new, either because these
these methods are modified, or are completely removed in the new version.
Similarly, some methods exist in the new version but not in the old one.
Several method bodies (bytecodes) are common to both the old and
new versions. From standard Java semantics, the old and the new
versions are individually type-safe. \JV requires that only methods
common to both versions be active on the stack at the time of the update.
This restriction together with the fact that we transform all heap objects
to conform to their new type signature, ensures that the
application is type-safe after the update.

\paragraph{Ensuring correct compiled code}

Category~(2) methods stem from indirect updates as mentioned in
Section~\ref{sec:dsu-view-of-changes}. Suppose some method {\tt getStatus}
calls method {\tt getForwardedAddresses} from our example, but {\tt
getStatus} source code and bytecode has not changed from versions 1.3.1 to
1.3.2.  Nevertheless, {\tt getStatus}'s \emph{machine code}, produced by
the JIT compiler, may need to be recompiled.  For example, if the new
compiled version of \getForwardedAddresses is at a different offset
than before, then the VM must recompile \getStatus to correctly refer
to the new offset.  An update may also change field offsets in modified
classes, which requires recompiling any class that accesses them as well.
Ginseng~\cite{neamtiu06dsu} and POLUS~\cite{chen:icse07}, two DSU systems
for C, likewise consider functions as changed if their source code is the
same but they access data types whose (compiled) representation is
different.  Note that a VM would not need to restrict category~(2)
methods if it used an interpreter that looked up offsets at each access.

Note that when the \VM \JIT compiler uses inlining, we may need to
increase the number of restricted methods to include those into which
the compiler inlined restricted methods.
In particular, if a category~(1),~(2),
or~(3) method {\tt m} is inlined into method {\tt n}, we also
restrict {\tt n} (and recompile it lazily after the update) to prevent the old
{\tt m} from running after the update. \RVM initially compiles a method
with its \emph{base}-compiler, which generates machine code but does not
apply sophisticated optimizations. Based on run-time profiling information,
the VM may recompile the same method later using its optimizing compiler,
which performs standard optimizations, including inlining.  It performs
inlining of small, frequently used methods; cost-based inlining for larger
methods; and may inline multiple levels down a hot call
chain~\cite{jalapeno}.  As a
consequence, \JV restricts inlined callers of restricted methods.

%   The compiler records its inlining
% decisions and \DSU restricts these methods as well.
% , when an update is available,
% computes a transitive closure of all methods that have inlined methods
% appearing in the changed set.
% \DSU adds these additional methods to the set of restricted methods
% used to determine a safe point.
% \suriya{\DSU does not compute this transitive closure. It only checks for
% this information when it encounters an opt compiled method on stack.}
% \ksm{I guess we care--is the above correct enough?}


% enough---we must also restrict those methods updated indirectly, 
% because these methods
% depend on the particular field and method offsets of the classes to
% which they refer, and 
% must be recompiled.  

\paragraph{Ensuring program specific semantics}

Even if a method has not changed, a user may need to manually blacklist it.
For example, suppose a method \texttt{handle} calls methods
\texttt{process} and \texttt{cleanup}, and the method \texttt{cleanup}
initializes a field that it uses.  Now suppose we update this program to
move the initialization statement into \texttt{process}, because
\texttt{process} needs to use the field as well.  In both versions, the
field is properly initialized when the program runs from scratch.  However,
suppose that \JV  applies the update and the thread running
\texttt{handle} yields in between the calls to \texttt{process} and
\texttt{cleanup}.  In this case, \texttt{handle}'s bytecode has not been
changed, so we could go ahead with the update.  But if we did, then the
program would have called the old \texttt{process} method, which did not
perform any initialization, and then would call the new \texttt{cleanup}
method, which performs no initialization either, since the new version
\texttt{process} does it, leading to incorrect semantics.  To avoid such
\emph{version consistency} problems~\cite{neamtiu08context} the programmer
should include \texttt{handle} in the restricted set. DSU testing can also
help produce such a list~\cite{dsu-testing}.
Our benchmarks
discussed in Chapter~\ref{chap:experience} did not require manual
restrictions, but a DSU system must support it to provide correctness in
many cases.

While these restrictions informally assure the safety of updates,
more work is needed to formally define update semantics and guarantee
safety, as mentioned in Section~\ref{sec:safety}.

\paragraph{Reaching a DSU safe point.} To safely perform \VM services such
as thread scheduling, garbage collection, and JIT compilation, \RVM, like
most production VMs, inserts yield points on all method entries, method
exits, and loop back edges.  If the VM wants to perform a garbage
collection or schedule a higher priority thread, it sets a yield flag, and
the threads stop at the next VM safe point. \JV piggybacks on this
functionality.  When \JV is informed that an update is available, it sets
the yield flag.  Once application threads on all processors have reached
\VM safe points, \JV checks the paused threads' stacks.  If no stack
refers to a restricted method, \JV applies the update.  

If any thread is running a restricted method, \JV defers the update and
installs a \emph{return barrier}\index{return
barrier}~\cite{return-barrier} on the topmost restricted method of each
thread.  A generic return barrier replaces the specified method return branch
back to the next instruction in the calling method with a branch instead to
\emph{bridge code}, which performs some special action and then executes
the return branch.  We added this generic return barrier functionality to
\RVM. This technology is standard in other VMs.  Our bridge code
restarts the update process.  When a restricted method returns,
the thread will block and \JV will restart the update process, which will
either reach a DSU safe point, or the VM will insert more return barriers.
If \JV does not reach a safe point within 15 seconds, it aborts the update
(the length of the timeout is arbitrary, and can be configured by the user).
However, with some additional care and stack state transformation, we
proceed 
% it turns out we can sometimes proceed
with some updates despite
category~(2) methods active on stack, as described next.

\begin{figure}[p]
\begin{tabular}{@{}l@{\hspace{10ex}}l@{}}
\begin{minipage}{0.53\textwidth}
\begin{lstlisting}[xleftmargin=3ex]
class C {
  static int sum(int c) {
    int y = 0;
    for (int i = 0; i < c; i++) {
      y += i;
    }
    return y;
  }
}
\end{lstlisting}
(a) A simple example showing method {\tt sum}
\end{minipage} &
\begin{minipage}{0.3\textwidth}
\begin{lstlisting}[xleftmargin=3ex]
0 iconst_0 
1 istore_1 
2 iconst_0 
3 istore_2 
4 goto 14 
7 iload_1 
8 iload_2 
9 iadd 
10 istore_1 
11 iinc 2 1 
14 iload_2 
15 iload_0 
16 if_icmplt 7 
19 iload_1 
20 ireturn 
\end{lstlisting}
(b) bytecode for {\tt sum}
\end{minipage} \\ \\ \\
\begin{minipage}{0.53\textwidth}
Running thread: MainThread  \\
Frame Pointer: 0xSomeAddress  \\
Program Counter: 16  \\
Local variables: \\
\hspace*{5ex} L0 (c) = 100;\\
\hspace*{5ex} L1 (y) = 1225;\\
\hspace*{5ex} L2 (i) = 50;  \\
Stack expressions:\\
\hspace*{5ex} S0 = 50;\\
\hspace*{5ex} S1 = 100;  \\
(c) State extracted from the stack frame
\end{minipage} &
\begin{minipage}{0.3\textwidth}
\begin{lstlisting}[xleftmargin=3ex]
ldc 100 
istore_0 
ldc 1225 
istore_1 
ldc 50 
istore_2 
ldc 50 
ldc 100 
goto 16 
0 iconst_0 
... 
16 if_icmplt 7 
... 
20 ireturn 
\end{lstlisting}
(d) Version of {\tt sum} with specialized prologue
\end{minipage} \\
\end{tabular}
\caption{Example of \RVM's \OSR
mechanism \cite{osr}\label{fig:fink-osr-example}}
\end{figure}


\subsection{On-Stack Replacement to lift category~(2) restrictions}
\label{sec:osr}
\JV reduces the number of
restricted methods in category~(2) by leveraging \VM support for
\acf{OSR}~\cite{osr-self-1,osr-self-2}. State-of-the-art \VM{}s use
\emph{adaptive}
strategies to selectively compile and recompile methods at increasing
levels of optimization as they get invoked more number of times, i.e.
become \emph{hot}. Usually, after recompilation, the next method invocation
runs the optimized version. However, some methods are long-running and
\VM{}s need a mechanism to transition from an actively running compiled
version of a method to a more optimized version.  \RVM normally uses \OSR
to replace a \emph{base}-compiled version of an active method with an
optimized version.  We observe that for category~(2) restricted methods,
the situation is much the same. An unchanged, on-stack method requires
recompilation, in our case to fix any changed offsets. If the stack only
contains unchanged and category~(2) methods, \JV first performs OSR on the
category~(2) methods, and then starts the
update. \JV currently supports \OSR only for \emph{base}-compiled
category~(2) methods. We leave engineering \JV to support \OSR for
optimized methods to future work.

\RVM has an excellent \OSR\index{\acl{OSR}} functionality \cite{osr} that is
simple and mostly compiler independent, expect for extracting stack state. The \OSR mechanism does not depend
on the compilers used to compile the two different versions of methods.
\OSR in \RVM takes effect when the thread running the to-be-recompiled
method reaches a yield point. After reaching the yieldpoint, \RVM
recompiles the topmost method on the thread's stack and modifies the
thread's \PC to switch to the newly recompiled method body. The key
challenge in making \OSR work is to correctly transition the \PC to the new
compiled code. The \VM must construct a new stack frame as expected by the
new code and find the \PC value in the new code that corresponds to the old
one. \RVM first examines the currently active stack frame and extracts
the values of local variable. It then generates a special prologue, in
bytecode, that will set local variables to their correct values. The last
bytecode instruction of this special prologue jumps to the bytecode at which to
resume execution. \RVM makes this transition
compiler-independent by expressing it in bytecode. The optimizing compiler
can compile the method with this special prologue. Jumping to the start
of the method will allow the \VM to set up the stack frame as expected by the
new version and resume execution at the right location.
Figure~\ref{fig:fink-osr-example} shows an example of \RVM's \OSR
transition mechanism. This example is taken from Fink \EA \cite{osr}.
Figure~\ref{fig:fink-osr-example} shows the specialized prologue that sets
up the stack frame.

We  extend \RVM's OSR facilities to support multiple stack activation
records, and multiple stack frames on the same stack. This later addition
makes it more likely to reach a DSU safe point when more than one
category~(2) method precedes a changed method on the stack.  Given this
support, \JV ignores \emph{base}-compiled category~(2) methods when
testing for a safe point.  If any \emph{base}-compiled category~(2) methods
are on stack at an otherwise DSU safe point, \JV uses \OSR to replace
them. Once \JV reaches a DSU safe point, it next installs the modified
classes.
