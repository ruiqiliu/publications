\section{Performance}
\label{sec:performance}

\JV's impact on performance can be divided into the following. 1) the
steady state impact of implementing dynamic updating support, 2) the pause
in application execution to perform the update, and 3) the cost of
recompiling updated and invalidated methods upon resuming execution. The
latter is very hard to measure and can be grouped with the steady state
performance impact. Our experiments show that
the main performance impact of \JV is the pause time while applying an update. Once
updated, the application eventually runs without further overhead.  To
confirm this claim, we measured the throughput and latency of two 
Jetty versions while running on
stock \RVM and on \JV after dynamically updating to the next version.
Section~\ref{subsec:jetty-webserver-performance} shows that the performance of these
configurations is essentially identical.

% We report
% on this experiment in Section~\ref{subsec:jetty-perf}.

The cost of applying an update is the time to load any new classes, invoke
a full heap garbage collection, and apply the transformation methods on
objects belonging to updated classes.  Roughly, the time to suspend threads
and check that the application is at a safe-point is less than a
millisecond, and classloading time is usually less than 20ms.
The update disruption time is primarily due to the GC and
object transformers, and is proportional to the size of the heap and the
fraction of objects being transformed.  We wrote a simple microbenchmark to
measure these overheads.  The results reported in
Section~\ref{subsec:microbench} show that object transformation
is the dominant cost.

We conducted all our experiments on an Intel Core 2 Quad machine running at
2.4 GHz machine with 2 GB of RAM.  The machine ran Ubuntu 7.10 on Linux
kernel version 2.6.22. We implemented \JV on top of \RVMversion and built a
FullAdaptiveSemiSpace configuration of the \VM. The FullAdaptive
configuration means that the \VM's code is compiled at the highest level of
optimization while creating the \VM's boot image (Full) and the application
code is compiled adaptively adaptively (Adaptive)~\cite{AAB+:00}. In
adaptive compilation, the \VM baseline compiles an application method the
first time it is executed, and recompiles the method at increasing levels
of optimization as it gets invoked more often.  The SemiSpace configuration
refers to \RVM's semi-space garbage collector.
