\BCode[p]{0.75}
\begin{lstlisting}[frame=single]
public class FtpServer implements Runnable {
  // main loop of server
  public void run() {
    while (true) {
      Socket cl = serverSocket.accept();
      RequestHandler rh = new RequestHandler(cl);
      (new Thread(rh)).start();
    }
  }
}

// Handle a single client
public class RequestHandler implements Runnable {
  private BufferedReader reader;
  private FtpWriter writer;
  private boolean isClosed;

  // main loop of server-client conversation
  public void run() {
    while (!isClosed) {
      String command = reader.readLine();
      service(command, writer);
    }
  }
}

// Creates a single server thread
public class CommandLine {
  public static void main(String args[]) {
    FtpServer server = new FtpServer();
    (new Thread(server)).start();
  }
}
\end{lstlisting}
\ECode{CrossFTP code: High level organization\label{fig:crossftp-code}}


\subsection{CrossFTP server}
\label{subsec:crossftp}

CrossFTP server is an easily configurable, security-enabled \FTP server.
CrossFTP allows simple configuration through a \GUI and more advanced
customization using configuration files. The \GUI interface also allows
detailed monitoring of server operation and a real-time look at active
clients and the commands they issue.  We did not use the GUI interface and
therefore do not consider changes to that part of the program.

\begin{table}[t] \footnotesize \centering
\begin{tabular}{lcccccccrcc} \toprule
     &  \mc{2}{c}{SLOC} & \mc{2}{c}{\# classes} &    \mc{6}{c}{\# changed}             \\ \cmidrule(l){2-3} \cmidrule(lr){4-5} \cmidrule{6-11}
Ver. & \mr{2}{*}{Total}   & \mr{2}{10ex}{Without JmDNS} & \mr{2}{*}{add} & \mr{2}{*}{del} & \mr{2}{*}{classes} & \mc{3}{c}{methods} & \mc{2}{c}{fields}  \\ \cmidrule{7-9} \cmidrule(l){10-11}
     &                    & &     &     &         & add & del & chg   & add & del             \\ \toprule
1.05 &  13,852            & 13,852   &     &     &         &     &     &       &     &                 \\
1.06 &  13,926            & 13,926   & 4   & 1   & 1       & 0   & 0   & 3/0   & 1   & 0               \\
1.07 &  18,081            & 14,066   & 0   & 0   & 3       & 4   & 0   & 14/0  & 5   & 0               \\
1.08 &  18,108            & 14,093   & 0   & 1   & 3       & 2   & 0   & 10/0  & 0   & 2               \\ \bottomrule
\end{tabular}
\caption{Summary of updates to CrossFTP server\label{tab:crossftp-changes}}
\end{table}

% vim:tw=0


We looked at 4 versions of CrossFTP---1.05 through 1.08, details shown in
Table~\ref{tab:crossftp-changes}---spanning a duration of more than a year.
Version 1.08 contains about 18,000 \SLOC across 161 classes.  Version 1.07
added over four thousand \SLOC by including code from JmDNS, a Java
implementation of multi-cast DNS~\cite{jmdns}. We did not exercise this
functionality in our runs. 

Figure~\ref{fig:crossftp-code} shows the high level organization of the
server's code. The server has one active thread, an instance of {\tt
FtpServer} which listens on a socket for client connections. The main loop
of the server is shown in function {\tt FtpServer.run()}. For each client,
the server spawns a new thread, an instance of the class {\tt
RequestHandler}.  At any point of time, the application is running one
server thread and multiple connection threads, one for each active client
connection. The main loop to handle a client is given in {\tt
RequestHandler.run()}. The server reads a command from the client, services
the command and waits for the next one. In steady state, {\tt FtpServer}'s
{\tt run} method, {\tt RequestHandler}'s {\tt run} method and methods
involved in processing particular \FTP commands from the client are active
on stack.

\subsubsection{Updates to CrossFTP}

\JV successfully applies all three updates to this application. For all
three updates, the default transformers we sufficient. Note that since all
updates either add or delete fields, simple method body updating support on
its own would be insufficient.

\JV could only apply the update from version 1.07 to 1.08 when the server
was relatively idle. The server runs a new {\tt RequestHandler} thread to
process each \FTP session, and the \texttt{RequestHandler.run()} method was
changed by the update. \JV installs a return barrier on this method, but
with many active sessions, this method is essentially always on stack,
preventing the update.
