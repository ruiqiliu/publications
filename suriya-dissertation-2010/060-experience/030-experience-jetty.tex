\subsection{Jetty webserver}
\label{subsec:jetty}

Jetty is a popular webserver written in Java. It supports static and
dynamic content and can be embedded within other Java applications. It is
used by various popular projects such as Eclipse, Google App Engine, and
the Hadoop map-reduce framework \cite{jetty-uses}.  \RVM, and thus \JV, is
not able to run the most recent versions of Jetty (6.x).  Therefore we
considered 11 versions, starting at 5.1.0, released in November, 2004
through 5.1.10, released in January, 2006.
% (the last % version prior to version 6).
Version 5.1.10 contains 317 classes and about 45,000 \SLOC (lines of code
excluding comments and blank lines, as reported by the sloccount
tool~\cite{sloccount, sloccount-wiki}).

\begin{table} \centering \footnotesize
\begin{tabular}{lrrccccrcc} \toprule
% First row
\mr{3}{5ex}{\BC Ver. \EC} &
\mr{3}{9ex}{\BC Date \EC} &
\mr{3}{7ex}{\BC SLOC \EC} &
\mr{3}{10ex}[2ex]{\BC \# classes added \EC} & \mc{6}{c}{\# changed} \\ \cmidrule{5-10}
% Second row 
       &                     &         &         & \mr{2}{*}{classes} & \mc{3}{c}{methods} & \mc{2}{c}{fields} \\ \cmidrule(lr){6-8} \cmidrule(lr){9-10}
       &                     &         &         &                    & add & del & \mc{1}{c}{chg}   & add & del \\ \midrule
5.1.0  & Nov 2004            & 42,981  &         &                    &     &     &                  &     &     \\
5.1.1  & Dec 2004            & 43,073  & 0       & 14                 &\ 4  & 1   &  38/0            &  0  & 0   \\
5.1.2  & Jan 2005            & 43,277  & 1       &\ 5                 &\ 0  & 0   &  12/1            &  0  & 0   \\
5.1.3* & Apr 2005            & 43,612  & 3       & 15                 & 19  & 2   &  59/0            & 10  & 1   \\
5.1.4  & Jun 2005            & 43,578  & 0       &\ 6                 &\ 0  & 4   &   9/6            &  0  & 2   \\
5.1.5  & Nov 2005            & 44,027  & 0       & 54                 & 21  & 4   & 112/8            &  5  & 0   \\
5.1.6  & Nov 2005            & 43,948  & 0       &\ 4                 &\ 0  & 0   &  20/0            &  5  & 6   \\
5.1.7  & Dec 2005            & 44,081  & 0       &\ 7                 &\ 8  & 0   &  11/2            &  9  & 3   \\
5.1.8  & Dec 2005            & 44,082  & 0       &\ 1                 &\ 0  & 0   &   1/0            &  0  & 0   \\
5.1.9  & Dec 2005            & 44,084  & 0       &\ 1                 &\ 0  & 0   &   1/0            &  0  & 0   \\
5.1.10 & Jan 2006            & 44,102  & 0       &\ 4                 &\ 0  & 0   &   4/0            &  0  & 0   \\ \bottomrule
\end{tabular}
\caption{Summary of updates to Jetty\label{tab:jetty-changes}}
\end{table}


Table~\ref{tab:jetty-changes} shows a summary of the changes in each
update. For each version, the first three columns list the version number,
release date and \SLOC. For versions 5.1.1 through 5.1.10,
each row tabulates the changes relative to the prior version. The fourth
column shows the number of classes that the new version added. The fifth
column shows the number of classes that contained changes from the previous
version. Columns 6 though 10 enumerate changes such as the addition,
deletion and modification of methods; and the addition and deletion of
fields. For the eighth column listing changed methods, the notation $x/y$
indicates that $x+y$ methods were changed, where $x$ changed in body only,
and $y$ changed their type signature as well. For dynamic updating systems
that only support changes to method bodies, only the first and last three
of the ten updates could be supported, since the rest either change method
signatures and/or add or delete fields.


\BCode[p]{0.72}
\begin{lstlisting}[frame=single]
// There are different listeners for various
// virtual hosts Each listener has multiple
// threads
public class HttpListener {
  public void run() {
    while (true) {
      if (job == null)
        wait();
      if (job != null)
        handle(job);
    }
  }
}

// Maps a collection of HttpListener objects
// which generates requests and HttpContext
// objects which maintain collection of
// HttpHandlers
public class HttpServer {
  public void start() {
    for (HttpContext c : contexts) {
      c.start();
    }
    for (HttpListener l : listeners) {
      l.start();
    }
  }

  public static void main(String args[]) {
    HttpServer server = new HttpServer();
    HttpContext context = server.getContext();
    context.addHandler(...);
    server.addListener(...);
    server.start();
  }
}
\end{lstlisting}
\ECode{Jetty webserver code: High level organization\label{fig:jetty-code}}


Figure~\ref{fig:jetty-code} shows the high-level organization of the
application. The main class of the application {\tt HttpServer} services
\HTTP requests by maintaining a map between a collection of \HttpListener
objects and {\tt HttpContext} objects. The \HttpListener
objects listen for client requests. The application supports virtual hosts
(multiple websites on the same server listening possibly at different
addresses) by having a \HttpListener object for each virtual host. The
listener also maintains a fixed pool of threads to service client requests,
thereby avoiding thread creation overhead for each request. The application
uses complex producer-consumer synchronization between the client sockets
and thread pools. {\tt HttpContext} objects provide context such as
filesystem path prefix, classpath, and resource location for {\tt
HttpHandler} objects. Each handler supports a different type of
request---regular file request, running servlets and web applications,
returning error messages, and so on. When the server is idle, threads in
the thread pools wait to be woken up by listeners which wait on client
sockets.


\subsubsection{State transformer functions in Jetty}

Between the default class and object transformers the \acf{UPT} generated
and those we wrote by hand, we successfully wrote dynamic updates to all
versions of Jetty that we examined. The update to version~5.1.2 demonstrates
the usefulness of automatically generated default transformers in the
common case. Version~5.1.2 changes the access protection of method {\tt
setHttpContext} in class {\tt HttpResponse} to \public. This method is now
called by methods outside class {\tt HttpResponse}
and \JV correctly loads these method bodies to
update the application. Since the update changes the class signature of
{\tt HttpResponse}, the entire class has to be reloaded again, and \JV must
run class and object transformers for this class.
Figure~\ref{fig:jetty-update-to-5.1.2} shows the default transformer that
\UPT generated.
In this situation, where
there is no semantic change, the default transformers
correctly set the fields in the new version. Moreover, the fact that the
{\tt HttpResponse} class has over fifty fields make the default
transformers extremely useful.

\begin{figure}[t]
\centering
\begin{minipage}{0.98\textwidth}
\begin{lstlisting}[frame=single]
import org.mortbay.http.HttpResponse;
import org.mortbay.http.r_5_1_1_HttpResponse;

public final class JvolveTransformers {
  public static void jvolveObject(HttpResponse to,
                          r_5_1_1_HttpResponse from) {
    to._status = from._status;
    to._reason = from._reason;
    to._httpContext = from._httpContext;
  }
  public static void jvolveClass(HttpResponse unused) {
    HttpResponse.log = r_5_1_1_HttpResponse.log;
    HttpResponse.__100_Continue =
          r_5_1_1_HttpResponse.__100_Continue;
    HttpResponse.__101_Switching_Protocols =
          r_5_1_1_HttpResponse.__101_Switching_Protocols;
    ...
    HttpResponse.__505_HTTP_Version_Not_Supported =
          r_5_1_1_HttpResponse.__505_HTTP_Version_Not_Supported;
    HttpResponse.__507_Insufficient_Storage =
          r_5_1_1_HttpResponse.__507_Insufficient_Storage;
    HttpResponse.__statusMsg =
          r_5_1_1_HttpResponse.__statusMsg;
    HttpResponse.__Continue =
          r_5_1_1_HttpResponse.__Continue;
  }
}
\end{lstlisting}
\end{minipage}
\hangcaption{UPT-generated transformers for the update to Jetty v5.1.2}
% These transformers were correct and used for the update.}
\label{fig:jetty-update-to-5.1.2}
\VspaceFixForHangcaption
\end{figure}


\lstset{frame=single}
\begin{figure}[p]
\begin{tabular}{@{}m{0.5\textwidth}@{}m{0.5\textwidth}@{}}
\BC \begin{minipage}{0.42\textwidth}
\begin{lstlisting}
public class HttpContext {
  private List
    _systemClasses;
}
\end{lstlisting}
\end{minipage} \EC &
\BC \begin{minipage}{0.42\textwidth}
\begin{lstlisting}
public class HttpContext {
  private String[]
    _systemClasses =
      new String [] {...};
}
\end{lstlisting}
\end{minipage} \EC \\[-3.5ex]
\BC (a) Version 5.1.5 \EC &
\BC (b) Version 5.1.6 \EC \\
\end{tabular}

\begin{tabular}{@{}b{\textwidth}@{}}
\centering
\begin{minipage}{0.7\textwidth}
\begin{lstlisting}
public static void jvolveObject(
  HttpContext to, r_5_1_5_HttpContext from) {
    to._systemClasses = null;
}
\end{lstlisting}
\end{minipage} \\[-0.5ex]
(c) Default transformer \\[1.5ex]
\begin{minipage}{0.7\textwidth}
\begin{lstlisting}
public static void jvolveObject(
  HttpContext to, r_5_1_5_HttpContext from) {
  if (from._systemClasses == null) {
    to._systemClasses = null;
  } else {
    to._systemClasses =
      new String[from._systemClasses.size()];
    int i = 0;
    for (Object o : from._systemClasses) {
      to._systemClasses[i] = (String) o;
      i++;
    }
  }
}
\end{lstlisting}
\end{minipage} \\[-0.5ex]
(d) User generated transformer
\end{tabular}
\caption{Object transformer from Jetty version 5.1.5 to
5.1.6\label{fig:sjt}}
\end{figure}
\lstset{frame=none}


We now describe updates where default transformers were not sufficient.  In
Version~5.1.3 class {\tt NCSARequestLog} added two boolean fields {\tt
\_logLatency} and {\tt \_logCookies}. {\tt \_logLatency} specifies whether
the application should include the latency to process HTTP requests when it
logs information about a request, and {\tt \_logCookies} controls whether
or not that application logs
cookies information. The default transformer sets these boolean
fields to {\tt false}. However, the actual values to set these fields to
depend on the configuration parameters when running the new version. In
the same update, Class {\tt Pool} which manages a pool of threads to handle
HTTP requests adjusts the number of threads based on system load, added a
\private field {\tt \_lastShrink} to store the time the number of threads
were last shrunk. Jetty uses this field to prevent shrinking
available threads frequently.  Setting this field to the default value of
zero should not pose any major problems because this field will have the
correct time the next time the pool is shrunk.

The update from version~5.1.5 to 5.1.6 changed the type of two fields {\tt
\_systemClasses} and {\tt \_serverClasses} from {\tt List} in the old
version to \StringArray in the new, in class {\tt HttpContext}.
Figure~\ref{fig:sjt} shows the changes to one of these fields. In
Figure~\ref{fig:sjt}~(c), \UPT's default transformer sets the field {\tt
\_systemClasses} to \NULL (similar to the example with JavaEmailServer,
Figure~\ref{fig:jes-transformer-code}). Since the new version's field
declaration (Figure~\ref{fig:sjt}~(b)) already declares the value of these
fields, the developer writing to object transformer has to decide whether
to use this value or convert the value stored in the old version's list
into a String array. We choose the latter option, shown in
Figure~\ref{fig:sjt}~(d).

\subsubsection{Reaching a safe point in Jetty}

\begin{sidewaystable} \centering \footnotesize
\begin{tabular}{@{}lcrrrrrrr@{}} \toprule
\CC{Upd.}   &                & \CC{Number of}  & \mc{4}{c}{\# methods not allowed on stack, due to}        & \mc{2}{c}{Number of}   \\ \cmidrule{4-7}
\CC{to}     & Reached        & \CC{methods at} & \CC{class}   & \CC{method}  & \CC{indirect} &             & \mc{2}{c}{restricted}  \\
\CC{ver.}   & safe point?    & \CC{runtime}    & \CC{updates} & \CC{body}    & \CC{method}   & \CC{Total}  & \mc{2}{c}{methods}     \\ \cmidrule{8-9}
            &                &                 &              & \CC{updates} & \CC{updates}  &             & \mc{1}{c}{w/o OSR}
                                                                                                                      & \mc{1}{c}{w/ OSR}\\ \toprule
5.1.1   &  always        & 1378 (376) & 26/49          & 7/12           & 20/29           & 53/90  (17)  & 67         & 10         \\
5.1.2   &  4/5$^\dagger$ & 1374 (375) & 25/25          & 3/5            & 35/43           & 63/73  (35)  & 67         & 4          \\
5.1.3   &  0/5$^*$       & 1374 (375) & 326/382        & 4/6            & 42/45           & 370/433 (97) & 373        & 23         \\
5.1.4   &  always        & 1384 (374) & 82/82          & 5/6            & 15/16           & 101/104 (24) & 101        & 10         \\
5.1.5   &  always        & 1380 (372) & 14/80          & 39/60          & 13/15           & 62/155 (17)  & 62         & 60         \\
5.1.6   &  3/5$^\dagger$ & 1394 (378) & 203/219        & 3/3            & 16/19           & 222/241 (40) & 223        & 20         \\
5.1.7   &  always        & 1394 (380) & 186/187        & 1/2            & 53/69           & 239/258 (74) & 243        & 12         \\
5.1.8   &  always        & 1402 (379) & 0/0            & 1/1            & 0/0             & 1/1   (1)    & 1          & 1          \\
5.1.9   &  always        & 1402 (379) & 0/0            & 0/1            & 0/0             & 0/1   (0)    & 0          & 0          \\
5.1.10  &  always        & 1402 (379) & 0/0            & 4/5            & 0/0             & 4/5   (2)    & 6          & 4          \\ \bottomrule
\mc{9}{c}{\begin{minipage}{0.6\textwidth}
\vspace{2ex}
$^\dagger$Restricted method \texttt{HttpConnection.handleNext()} was active
\end{minipage}} \\[1ex]
\mc{9}{c}{\begin{minipage}{0.6\textwidth}
$^*$Restricted method \texttt{ThreadedServer.acceptSocket()} was always active
\end{minipage}} \\
\end{tabular}
\caption{Impact of safe point restrictions on updates to Jetty}
\label{tab:safe-point-jetty}
\end{sidewaystable}

% vim:tw=0


For the updates we considered, we tried to study how our safe point
restrictions (Section~\ref{sec:safe}) inhibited a dynamic update.
Other than the update to 5.1.3, all versions immediately
reached a safe point every time, with no need of return barriers.

% To understand why, we instrumented the VM to emit information about
% restricted method set and, if a safe point cannot be reached, which
% restricted method was active. 

% \suriya{The following information about inlining is commented out. What
% should we do with this? See .tex file}

For each version, starting at 5.1.0, we ran Jetty under 20\% load (160
connections per second from our httperf experiments). After 30 seconds we
tried to apply the update to the next version; if a safe point could not be
immediately reached, we deemed the attempt as failed. We tried five such
attempts for each version, starting up from the server from scratch for
each attempt. The results are presented in
Table~\ref{tab:safe-point-jetty}. Column 2 shows the number of times out of
five such runs where the application reached a safe point. The methods
whose presence on a thread stack precluded the application from reaching a
safe point are mentioned below the table. For the update to 5.1.3, the
offending method was always active because it contained an infinite loop.
The other updates either always succeeded, or did after a small number of
retries.

We could not apply the update to version 5.1.3 (denoted with an asterisk in
the table) because \JV was never able to reach a safe point. The update
modified {\tt ThreadedServer.acceptSocket()}, a method that waits for a
connection from the client, and this method is nearly always on stack.  We
installed a return barrier that is triggered when {\tt acceptSocket}
returns, but this barrier is not sufficient to perform the update since the
main methods of several threads were themselves modified. For example, we
also install a return barrier on {\tt PoolThread.run()}, but this barrier
is never triggered because this method never becomes inactive.

Column 3 contains the total number of methods in the program at runtime,
where the number in parentheses is the number of those which the compiler
inlined when using aggressive optimization.  This provides an upper bound
on the effect of inlining in reaching a safe point.  The next group of
columns contains the restricted method set. Each column in the group
specifies the number of methods loaded at run time by the \VM, followed by
the total number of methods in that category in the program. The first
column in this group is the number of methods in classes involved in a
class update. Recall that when a class is updated, say by adding a field,
all its methods are considered restricted (see section~\ref{sec:safe}).  The
second column in this group is the number of methods whose bodies are
updated, the third is the number of category~(2) or indirectly updated
methods, and the
fourth sums these, with the number of methods that were inlined written in
parentheses. The final two columns list the total number of methods in the
restricted set; they differ from the first number in the fourth column by
the number of inlined callers of the restricted methods that were not
already restricted.  The final column lists the actual number of restricted
methods when \JV's \OSR capability is enabled.

The table shows that both indirect method calls and inlining significantly
add to the size of the restricted set.  Inlining though, is small by
comparison, because all callers of an updated class's methods are
\emph{already} included in the indirect set. Therefore, inlining these
methods adds no further restriction.  In most cases \OSR support
dramatically reduces the number of restricted methods and increases the
likelihood of reaching a \USD safe point.  Interestingly, having a greater
number of restricted methods overall does not necessarily reduce the
likelihood that an update will take effect; rather, it depends on the
frequency with which methods in this set are on the stack.

% REFACTOR: add para back if we get things to work.
% We refactored the various long-running main methods in versions 5.1.2 and
% 5.1.3 to extract the modified bodies of long running methods into separate
% methods and leave the main method containing the loop unmodified between
% the two versions.  (This sort of transformation is performed by other
% systems automatically by programmer directive in
% Ginseng~\cite{neamtiu06dsu}.) 
% When we attempted to perform a dynamic update, \DSU{}
% installed a return barrier for {\tt acceptSocket()}. This function waits
% for a connection and returns after a timeout. \DSU{} was able to update the
% application when this function returned. 
