\chapter{Evaluation}
\label{chap:experience}

To evaluate \JV, we used it to update three open-source servers written in
Java: Jetty webserver~\cite{jetty}, JavaEmailServer~\cite{jes}, % an SMTP
% and POP e-mail server,
and CrossFTP server~\cite{crossftp}. These programs
belong to a class that should benefit from DSU because they typically run
continuously.  DSU would enable deployments to incorporate bug fixes or add
new features without having to halt currently-running instances or create
duplicate instances with built-in special purpose redirection
functionality.

We explored updates corresponding to releases made over roughly one to two
years of each program's lifetime.  None of these programs were maintained
with \USD support in mind. Of the 22 updates we considered, \JV could
support 20 of them---the two updates we could not apply changed classes
with infinitely-running methods, and thus no safe point could be reached.
To our knowledge, no existing DSU system for Java could support all these
updates; indeed, previous systems with simple support for updating method
bodies would be able to handle only 9 of the 22 updates.  Although \JV
cannot support every update, it would substantially reduce required
downtime. It is the first DSU system for Java that has been shown to
support changes to realistic programs as they occur in practice over a long
period of time.

In the rest of this chapter, we first examine the performance impact of
\JV, and then look at updates to each of the three applications in detail.
