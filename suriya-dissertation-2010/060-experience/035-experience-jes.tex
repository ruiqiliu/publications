\begin{table}[t] \footnotesize \centering
\begin{tabular}{lrrcccccrcc} \toprule
% First row
\mr{3}{5ex}{\BC Ver. \EC} &
\mr{3}{9ex}{\BC Date \EC} &
\mr{3}{7ex}{\BC SLOC \EC} & \mc{2}{c}{\# classes} & \mc{6}{c}{\# changed} \\ \cmidrule{4-5} \cmidrule(l){6-11}
% Second row 
       &                     &         & \mr{2}{*}{add} & \mr{2}{*}{del}      & \mr{2}{*}{classes} & \mc{3}{c}{methods} & \mc{2}{c}{fields} \\ \cmidrule(lr){7-9} \cmidrule(lr){10-11}
       &                     &         &       &       &                    & add & del & \mc{1}{c}{chg}   & add & del \\ \midrule
% \begin{tabular}{|l|r|r||c|c||c|r|c|r|r|c|} \hline \T
%       &                      &        & \mc{2}{c||}{\# classes} &    \mc{6}{c|}{\# changed} \\
% Ver.  & \mc{1}{c|}{Date}     & SLOC   & add & del & classes & \mc{3}{c|}{methods} & \mc{2}{c|}{fields} \\
%       &                      &        &     &     &         & add & del & chg   & add & del \\ \hline \hline \T
1.2.1 & Dec 2002             &  2,841 &     &     &         &     &     &       &     &     \\
1.2.2 & Feb 2003             &  2,841 & 0   & 0   & 3       & 0   & 0   & 3/0   & 0   & 0   \\
1.2.3 & Jul 2003             &  2,924 & 0   & 0   & 7       & 0   & 0   & 14/2  & 12  & 0   \\
1.2.4 & Sep 2003             &  2,961 & 0   & 0   & 2       & 0   & 0   & 4/0   & 0   & 0   \\
1.3*  & Oct 2003             &  2,305 & 4   & 9   & 2       & 11  & 3   & 6/9   & 12  & 5   \\
1.3.1 & Oct 2003             &  2,307 & 0   & 0   & 2       & 0   & 0   & 4/0   & 0   & 0   \\
1.3.2 & Nov 2003             &  2,359 & 0   & 0   & 8       & 4   & 2   & 4/2   & 3   & 1   \\
1.3.3 & Nov 2003             &  2,368 & 0   & 0   & 4       & 0   & 0   & 3/0   & 0   & 0   \\
1.3.4 & Feb 2004             &  2,447 & 0   & 0   & 6       & 2   & 0   & 6/0   & 2   & 0   \\
1.4   & Jul 2004             &  2,529 & 0   & 0   & 7       & 6   & 1   & 4/1   & 6   & 0   \\ \bottomrule
\end{tabular}
\caption{Summary of updates to JavaEmailServer\label{tab:jes-changes}}
\end{table}


\subsection{JavaEmailServer}
\label{subsec:jes}

For JavaEmailServer, we considered 10 versions---1.2.1 through
1.4---spanning a duration of about two years.  Version 1.4 consists of 20
classes and about 2500 \SLOC.  Table~\ref{tab:jes-changes} shows the
release date, source lines of code and summary of updates to each new
version.

Figure~\ref{fig:jes-code} shows the
high-level structure of JavaEmailServer's code. The application creates
multiple \POP and \SMTP protocol handler threads. The number of threads can
be configured by the user. Each thread listens on its respective port (110
for \POP and 25 for \SMTP), waiting for new client connections. Upon
receiving a client request, the server thread communicates with the client,
services the client's command, and goes back to listening again. The
main loop of \POP and \SMTP servers are in {\tt Pop3Processor.run()} and
{\tt SMTPProcessor.run()} respectively. The \POP server authenticates
users, reads their e-mail messages from the filesystem and sends them to
the client through the network. The \SMTP server reads in e-mail messages
from the network and writes them to the filesystem. These messages may be meant
to be delivered to local users or relayed to other servers on the internet.
Message delivery is handled by a single {\tt SMTPSender} thread, which
polls the filesystem for new messages and writes them into user mailboxes,
or relays them to other servers. The number of threads and therefore the
maximum number of clients that can simultaneously be serviced is fixed and
known when the application starts. The server will ignore clients that try
to connect when all server threads are responding to requests. At any
moment, the {\tt run()} methods of all threads and based on server load,
additional methods to handle the \POP and \SMTP protocols and methods to
deliver messages are active on stack.

\BCode[p]{0.75}
\begin{lstlisting}[frame=single]
// Listen on socket for POP client connections
// Handle a client
public class Pop3Processor implements Runnable {
  public void run() {
    while (true) {
      Socket client = serverSocket.accept();
      handleCommands();
    }
  }
}

// SMTPProcessor is similar to Pop3Processor

// Poll the filesystem and deliver messages
public class SMTPSender implements Runnable {
  public void run() {
    while (true) {
      if (messageToBeDelivered)
        deliverMessages();
      sleep();
    }
  }
}

public class Mail {
  public static void main(String args[]) {
    // Create 5 POP threads
    for (int i = 0; i < 5; i++)
      new Thread(new Pop3Processor()).start();

    // Create 5 SMTP threads
    for (int i = 0; i < 5; i++)
      new Thread(new SMTPProcessor()).start();

    // Create one thread that delivers mail
    new Thread(new SMTPSender()).start();
  }
}
\end{lstlisting}
\ECode{JavaEmailServer code: High level organization\label{fig:jes-code}}


\subsubsection{Updates to JavaEmailServer}
Approaches that only support updates to method bodies will be able to
handle only four of the updates shown in Table~\ref{tab:jes-changes}.  We
could successfully construct updates for all versions we examined and we
could successfully apply all of them except the update to version 1.3. For
all updates but the one from 1.3.1 to 1.3.2 (shown in
Figure~\ref{fig:jes-string-emailaddress-example}), default transformers were
sufficient.


Version 1.3 has 20\% fewer \SLOC than its prior version because the update
reworked the entire
configuration framework of the server. Among other things, this version
removes a GUI tool for user administration and adds several new classes
that implement a file-based configuration system. As a result, many
classes are modified to point to a new configuration object.  Among these
classes are threads with infinite processing loops that accept POP and SMTP
requests.  Because these threads are always active, the safety condition
can never be met and thus the update cannot be applied.

The update from 1.3.1 to 1.3.2 indirectly changes the
\SMTPSenderrun
and \texttt{Pop3Processor.run()} methods.  These
methods contain processing loops run by several threads.  Though these
methods are always running, \JV applies \OSR and the update succeeds.  \JV
also uses OSR for the update from 1.3.2 to 1.3.3.
%MWH: suriya says this statement just isn't true.  Updates fine.  A method
%used to process connections was also changed, and this method is
%frequently active when the server is under load.  (A similar method was
%changed in the update to version 1.3.3.)  A return barrier installed on
%this method permits the update to take effect.  
