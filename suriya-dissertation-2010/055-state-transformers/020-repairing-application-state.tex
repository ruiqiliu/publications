\section{Repairing Application State}
\label{sec:repair}

The state transformer model presented above makes the
implicit
assumption that a to-be-updated application's execution state is correct.
In particular, dynamic patches that are used to initialize the new
version's state by examining the current state assume that this state is
well-formed.  Most times, this assumption is correct. Perhaps the bug does
not corrupt the heap, or has not yet been exercised, or only
causes incorrect input/output processing. However, in some cases, this
assumption can be faulty.

\begin{figure}[p]
\lstset{frame=single}
\BC
\begin{minipage}{0.78\textwidth}
\begin{lstlisting}[title={(a) Object is reachable, but never accessed by
the application}]
public void foo() {
  List<String> ll = new LinkedList<String>();
  // populate the list
  while (...) {
    ll.add(...);
  }
  if (...) {
    // some long running loop
    while (...) {
      // read elements from ll
    }
  } else {
    // another long running loop
    while (...) {
      // never use ll
    }
  }
}
\end{lstlisting}
\end{minipage} \\[1ex]
\begin{minipage}{0.78\textwidth}
\begin{lstlisting}[title={(b) Failing to remove objects from a collection}]
public void process(HashSet hs, Order order) {
  hs.add(order);
  if (...) {
    if (order.processed) {
      ...
      // forgets to remove order from hs
    }
  } else {
    hs.remove(order);
  }
}
\end{lstlisting}
\end{minipage}
\hangcaption{Examples of ``memory leaks'' in Java
\label{fig:leaks-definition}}
\EC
\lstset{frame=none}
\VspaceFixForHangcaption
\end{figure}


Consider a memory leak. After a while, there are
many reachable, but dead objects in memory that need to be freed. \JV and other
existing dynamic updating systems can apply a provided fix to the code to prevent
further leaks, but DSU researchers have paid little attention to the problem of finding
and freeing existing objects once the code is fixed. In this section, we
explore source code updates that fix memory leaks in real programs, and our
extending state transformers to repair application state at update time by
cleaning up existing leaked objects.
% 
While this discussion is specific to one bug type, we use it to gain
insight to more general mechanisms for fixing heap corruption.

\subsection{Memory leaks in Java}
In the context of a garbage collected language like Java, where unreachable
objects are automatically reclaimed, researchers define a memory leak as
follows. Any object that is still reachable from the roots of the heap
(globals and locals) but will not be accessed by the application in the
future is considered a memory leak. Consider the
simple program in Figure~\ref{fig:leaks-definition}~(a). The function defines a
local variable that points to a linked list. After populating the list,
execution reaches an if-then-else conditional. The linked list is accessed
only in the true branch of the if code block. If the conditional is
false, the linked list will not be accessed again in the future and
can be freed. This example leak is not too
problematic because it does not grow over time. More problematic are leaks that
continue to grow and eventually exhaust memory.
% We present this contrived example to
% present one instance of a memory leak in a garbage collected language. We
% do not expect programmers to issue patches to fix such leaks.

\begin{table}[t]
\begin{center}
\begin{tabular}{llp{0.57\textwidth}} \toprule
\CC{Application} & \CC{Patch}        & \CC{Description of leak} \\ \midrule
jEdit            & SVN r8329         & Application stores unnecessary information about closed
                                       buffers that do not correspond to any file in
                                       the file system. \\ \cmidrule{2-3}
                 & SVN r5178         & Code that splits a line into tokens continues to
                                       maintain references to that line. \\ \cmidrule{2-3}
                 & SVN r14027        & Similar to r5178, application but maintains a reference to
                                       an object of class TokenHandler. A TokenHandler
                                       knows how to split a string of a particular
                                       syntax (C, Java, Verilog, etc.) into
                                       tokens. \\ \midrule
Eclipse IDE      & Bug \#115789      & Application maintains reference counts to
                                       objects in a collection. In some
                                       cases, it neglects to correctly
                                       decrement this reference count such
                                       that application logic never
                                       reclaims the object. \\ \bottomrule
\end{tabular}
\caption{Memory leak fixes to real applications\label{tab:bugfixes}}
\end{center}
\end{table}


Another commonly used definition of a memory leak is semantic and typically
involves collections of objects such as an array, linked list, or hashmap.
Leaky objects are those that are reachable, but is never used by the
application for any real purpose.
% They can still be accessed by library
% code, for example, when a hashmap rehashes its buckets.
Consider
Figure~\ref{fig:leaks-definition}~(b).
The programmer fails to remove from a
hashset an object that is not required by application in the future. Note
that if the set is reachable, the object is not only reachable, but also
likely to be read by the program, when the hashset library re-hashes
buckets in the set.  We define objects that will not used by application
logic in the future as leaks. Developers do inadvertently introduce such
leaks in applications and fix them in future
versions~\cite{leak-pruning,cork}. In this work, when
we dynamically apply the source patch that fixes the leak, we also remove
leaky objects as part of state transformation.

\subsection{Fixing corrupt heap state for leaks}
We considered a total of four leaks in two Java applications jEdit and
Eclipse IDE.  jEdit is a popular text editor used by programmers. jEdit
provides common features such as syntax highlighting, folding, and
automatic indentation, and advanced features such as plugin support, macro
recording, and a built-in macro language. Eclipse IDE is a widely-used
multi-language Integrated Development Environment (IDE). Eclipse is well
known for its modularity and plugin architecture. While these applications
are not prime candidates for dynamic updating, they are complex, easily
available, and can run for a long time. They exhibit the same long-running
loop structure of mission-critical always-on programs and are good
candidates to demonstrate our work.

\begin{figure}[p]
\lstset{frame=single}
\begin{center}
\begin{tabular}{@{}c@{}}
\begin{minipage}{0.95\textwidth}
\begin{lstlisting}
  public class EditPane {
    private void handleBufferUpdate(BufferUpdate msg) {
      if(msg.getWhat() == BufferUpdate.CREATED) { ...
      } else if(msg.getWhat() == BufferUpdate.CLOSED) { ...
+       Buffer b = msg.getBuffer();
+       if (b.isUntitled()) {
+         // the buffer was a new file so I do
+         // not need to keep its info
+         Map carets = (Map) getClientProperty(CARETS);
+         if (carets != null)
+           carets.remove(b.getPath());
        }
      } else if(msg.getWhat() == BufferUpdate.SAVED) { ...
      } else ...
      }
    }
  }
\end{lstlisting}
\end{minipage} \\
(a) Patch applied by SVN revision 8329. \\
Lines 5--11 are added in the new version \\[1ex]
\begin{minipage}{0.95\textwidth}
\begin{lstlisting}
public static void jvolveObject(EditPane ep) {
  Map<String, CaretInfo> carets =
                   ep.getClientProperty("Buffer->Caret");
  if (carets != null) {
    for (String path : carets.getKeys()) {
      Buffer b = jEdit.getBuffer(path);
      if (b.isClosed() && b.isUnitled()) {
        carets.remove(path);
      }
    }
  }
}
\end{lstlisting}
\end{minipage} \\
(b) State transformer
\end{tabular}
\caption{jEdit leak and fix: SVN revision 8329\label{fig:r8329}}
\end{center}
\lstset{frame=none}
\end{figure}


Table~\ref{tab:bugfixes} shows descriptions of four leaks, three in jEdit
and one in Eclipse IDE. For all leaks, we wrote object transformers that
fixed the existing leak in the application during update time. \JV is not
yet robust enough to update the patch for Eclipse IDE. Therefore we
simulated the update and the object transformer from within the application,
by doing the following. We let the application initially run as normal.
After some number of iterations of the outer loop, we invoked the object
transformer code that would fix the leak, and in future iterations we
called the updated code without the leak. \JV correctly updates all three
jEdit patches and runs object transformers that fix the leaks at update
time.  We now discuss the individual leaks and their object transformers.

\begin{figure}[t]
\lstset{frame=single}
\begin{center}
\begin{tabular}{@{}c@{}}
\begin{minipage}{0.78\textwidth}
\begin{lstlisting}
  public class TokenMarker {
    private TokenHandler tokenHandler;
    public void markTokens(TokenHandler th)  {
      this.tokenHandler = th;
      // lots of processing
      ...
      ...
+     this.tokenHandler = null;
    }
  }
\end{lstlisting}
\end{minipage} \\
(a) Patch applied by SVN revision 14027. \\[1ex]
\begin{minipage}{0.78\textwidth}
\begin{lstlisting}
public static void jvolveObject(TokenMarker tm) {
  tm.tokenMarker = null;
}
\end{lstlisting}
\end{minipage} \\
(b) State transformer
\end{tabular}
\caption{jEdit update: SVN revision 14027\label{fig:r14027}}
\end{center}
\lstset{frame=none}
\end{figure}


\paragraph{jEdit SVN r8329} Figure~\ref{fig:r8329} shows source code patch
for revision 8329 that fixes the leak and the object transformer that
repairs the leak. jEdit maintains the last cursor position of each file in
a hashmap. This information is meaningless for an ``untitled buffer,'' i.e.,
an editor buffer that does not correspond to any file in the file system.
jEdit creates an untitled buffer when a user opens a new empty buffer.
After such a buffer is closed, jEdit has no use for its cursor position,
and this information should be removed from the hashtable. Versions prior
to r8329 failed to do so, and as a result leaked memory.
Figure~\ref{fig:r8329}~(b) shows the object transformer function that
repairs the leak. The function walks though all keys in the {\tt CARETS}
hashtable, and removes those entries that correspond to untitled buffers.

\begin{figure}[p]
\begin{center}
\begin{minipage}{0.6\textwidth}
\begin{lstlisting}[frame=single]
   class Editor { int refCount; }

   class History {
     Editor editor;
     void dispose() {
       this.editor = null;
     }
   }

   public class NavigationHistory {
     ArrayList<History> history;
     ArrayList<Editor> editors;

     void disposeEntry(History h) {
       if (h.editor == null) return;
       h.editor.refCount--;
       if (h.editor.refCount == 0)
         editors.remove(h.editor);
       h.dispose();
     }

     void updateNavigationHistory() {
       for (History h : history)
         if (...) {
           history.remove(h);
-          h.dispose();
+          disposeEntry(h);
         }
     }
   }
\end{lstlisting}
\end{minipage} \\
\caption{Eclipse IDE memory leak patch: Bug \#115789\label{fig:115789-patch}}
\end{center}
\end{figure}


\paragraph{jEdit SVN r5178 and r14027} The leaks fixed by versions r5178
and r14027 and similar and occur in the same function, but to two
different fields. We only show the code for version r14027 here.
Figure~\ref{fig:r14027} shows the source patch and object transformer for
this leak. jEdit calls function {\tt markTokens} when it needs to split a
string into tokens based on the type of the file being edited. Each file
type (C, Java, Verilog, etc.) has special logic to split text in a line,
embedded in an object of type {\tt TokenHandler}. The leaky jEdit version
fails to set the field {\tt TokenMarker.tokenHandler} to {\tt null}. The
object transformer is simple, it sets the leaky field to {\tt null}. It is
safe to execute the transformer only when the {\tt markTokens} function is
not active on stack. Since the update changes {\tt markTokens}, it will not
be active at a DSU safe point.

\begin{figure}[t]
\begin{center}
\begin{minipage}{0.83\textwidth}
\begin{lstlisting}[frame=single]
public static void jvolveObject(NavigationHistory nh) {
  for (Editor e : nh.editors)
    e.refcount = 0;
  for (History h : nh.history)
    if (h.editor != null)
      h.editor.refcount++;
  for (Editor e : nh.editors)
    if (e.refcount == 0)
      nh.editors.remove(e);
}
\end{lstlisting}
\end{minipage} \\
\hangcaption{State transformer that fixes Eclipse IDE memory leak bug
\#11\-5789\label{fig:115789-transformer}}
\end{center}
\VspaceFixForHangcaption
\end{figure}


\paragraph{Eclipse IDE \#115789} Figure~\ref{fig:115789-patch} shows the
source for a memory leak in Eclipse IDE. In the class {\tt
NavigationHistory}, the IDE maintains two lists: one of {\tt History}
objects called history, and another of {\tt Editor} objects called editors.
Each {\tt History} object has an {\tt editor} field. Thereby, each object in the
history list point to an object in editors list. An editor object may remain
in the editors list only as long as it is pointed to by some history
object. To enforce this property, the program maintains a reference count
field with each editor object. This field represents the number of history
objects pointing to that particular editor object. Whenever the
application removes a history object, it should decrement the reference
count of the editor object that the history object points to, and remove
the editor object from the editors list if the count is zero. Function {\tt
disposeEntry} implements this functionality. In one instance, the
programmer failed to follow this procedure, causing a memory
leak. Figure~\ref{fig:115789-transformer} shows the object transformer that frees
the leak at update time. The transformer walks though all history objects,
recomputes reference counts for editor objects, and frees editor objects
with a reference count of zero.
