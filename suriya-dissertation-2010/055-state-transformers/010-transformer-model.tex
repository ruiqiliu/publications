
\section{Object Transformation Model}
\label{sec:transformer-model}

% \todo{Fix this}
% This section explains \JV's object transformation model and discusses its
% limitations and alternatives. We first go over an example of a simple
% update, how intuitive transformers express this update, and how they lead to
% our chosen model. We then revisit our singly-linked to doubly-linked
% example introduced in Section~\ref{sec:xformers}.

This section explains \JV's object transformation model and compares it to
other alternatives by revisiting the singly-linked to doubly-linked example
introduced in Section~\ref{sec:xformers}.

% \JV provides to the
% user and our rationale. We go over the simple change in
% Figure~\ref{fig:two-classes-change} and argue that developers will find
% this model intuitive.

Any dynamic updating system must convert old process state at
the time of the update to the one expected by the new version at the point it
resumes execution. The system must convert state maintained in global
static data, method stack variables, and heap data as required for the new
version. Even ignoring semantics of the application, at the very least, the
system needs to represent data correctly in the form expected by the new
version.

In an object-oriented programming language like Java, data is primarily
stored as objects in the heap. An object is statically-typed\footnote{In
contrast, dynamically-typed languages such as Python may change fields
and methods of an object over time, and objects of the same class
need not be consistent.}, and is an instance of a particular {\tt class}.
A dynamic updating system for Java should convert heap objects to the new version in a
class-specific manner, leading to class or type-specific
object transformation functions \cite{neamtiu06dsu, K42reconfig}.
% In \JV's case, are \JO functions.
Such a type-specific transformation function specifies how to obtain an
object of the new version's type given an object of the old one. As
explained in Section~\ref{sec:updating-data}, there exist two main design
choices on when to execute these transformation functions. One design is to
lazily transform objects. In a lazy approach, objects are transformed when
they are first accessed by the application after the update. A lazy
approach requires that application code be instrumented to check the state
of an object and transform it if necessary. The other design alternative is
to eagerly identify and transform all objects at once. For \JV, we followed
the eager approach because we did not want to incur the steady-state
performance overhead that comes with instrumenting code and because we
could efficiently transform the entire heap by piggybacking on garbage collection. We
first discuss design choices with the eager transformation model and then
present the lazy transformation model and how one might implement it in
\JV.

\subsection{Eager transformation models}

For the rest of this discussion, we assume that the interface to specify
object transformers is the \JO function. This function accepts two
parameters: an object corresponding to the old version called \FROM and an
object for the new version called \TO. The programmer specifies how to
initialize the \TO object with data from the \FROM object.  The
interesting question in this and other models is,
what the types of the \FROM and \TO parameters
should be. We require access to the both the old and new version's
definition, yet we must not break Java's type system in order to utilize
the strong type-safety guarantees provided by the language. The definitions
of the \FROM and \TO objects depend on what view of the heap
the developer uses during transformation time and vice versa. In one, the
programmer has access to the old version's code and any pointer dereference
will return an object that is consistent with the old version.  We call
this model the ``old world'' model. In the ``new world'' model, the programmer has access to
the new version's code and any pointer dereference will return an object of
the new version. The old world model
has some natural advantages over the new world model. Since the old world
model presents the well-formed old heap from before the update, any code or
data access on this heap will be safe. The new world model exposes the
transformation function to a heap that is yet to be populated.  Dereferencing
pointers to objects whose contents are yet filled can result in
incorrect data values or null pointer exceptions. Before transformer
code can access an object, the programmer or the compiler has to ensure
that its contents are correctly filled in. In \JV we implemented the new
world model because it integrated naturally with our VM implementation.

\begin{figure}[t]
\lstset{frame=single}
\begin{tabular}{@{}m{0.5\textwidth}@{}m{0.5\textwidth}@{}}
\BC \begin{minipage}{0.3\textwidth}
\begin{lstlisting}
public class X {
  private Y y;
}
public class Y {
  private int i;
}
\end{lstlisting}
\end{minipage} \EC &
\BC \begin{minipage}{0.32\textwidth}
\begin{lstlisting}
public class X {
  private String s;
  private Y y;
}
public class Y {
  private int i, j;
}
\end{lstlisting}
\end{minipage} \EC \\[-3ex]
\BC (a) Old version \EC &
\BC (b) New version \EC \\[-2ex]
\end{tabular}
\hangcaption{Example of a simple update where the field of an updated class
refers to another updated class\label{fig:two-classes-change-only-diff}}
\lstset{frame=none}
\VspaceFixForHangcaption
\end{figure}

% \begin{figure}[t]
\lstset{frame=single}
\begin{tabular}{@{}m{0.5\textwidth}@{}m{0.5\textwidth}@{}}
\BC \begin{minipage}{0.3\textwidth}
\begin{lstlisting}
public class X {
  private Y y;
}
public class Y {
  private int i;
}
\end{lstlisting}
\end{minipage} \EC &
\BC \begin{minipage}{0.32\textwidth}
\begin{lstlisting}
public class X {
  String s;
  private Y y;
}
public class Y {
  private int i, j;
}
\end{lstlisting}
\end{minipage} \EC \\[-3ex]
\BC (a) Old version \EC &
\BC (b) New version \EC \\[-2ex]
\end{tabular}
\BC \begin{minipage}{0.75\textwidth}
\begin{lstlisting}
class old_X { public Y y; }
class old_Y { public int i; }
class JvolveTransformers {
  public static void
      jvolveObject(X to, old_X from) {
    to.s = null; to.y = from.y;
  }
  public static void
      jvolveObject(Y to, old_Y from) {
    to.i = from.i; to.j = 0;
  }
}
\end{lstlisting}
\vspace{-1ex} \BC (c) Stub classes used in transformers \EC
\end{minipage} \EC
\hangcaption{Example of a simple update where the field of an updated class
refers to another updated class\label{fig:two-classes-change}}
\VspaceFixForHangcaption
\lstset{frame=none}
\end{figure}


Figure~\ref{fig:two-classes-change-only-diff} shows an example of an update
with two classes {\tt X} and {\tt Y}.  Class {\tt X} has a field {\tt y}
pointing to an object of type {\tt Y}.  The update adds an additional field
each to classes {\tt X} and {\tt Y}.
Figure~\ref{fig:two-classes-change-transformers} shows the object
transformers for classes {\tt X} and {\tt Y} in the old and new world
models. In the old world model, the \FROM parameters are of the old types
{\tt X} and {\tt Y}. The types of the \TO parameters are artificially
constructed stub classes that correspond to the definition of the new
versions of {\tt X} and {\tt Y}. Note that fields of the \nX and \nY
classes, specifically, {\tt \nX.y} refers to an old version object, thus
maintaining the old world invariant. Conversely, in the new world model the \TO
parameters are of the new version's types while the \FROM parameters are
stub classes that correspond to the definition of the old version. The
field {\tt \oX.y} refers to a new version object.

\begin{figure}[p]
\BC
\begin{minipage}{0.62\textwidth}
\begin{lstlisting}[frame=single]
// new_X.y refers to the old version
// The new X and Y do not exist, yet
class new_X { String s; Y y; }
class new_Y { int i, j; }

class JvolveTransformers {
  public static void
      jvolveObject(new_X to, X from) {
    to.s = null; to.y = from.y;
  }
  public static void
      jvolveObject(new_Y to, Y from) {
    to.i = from.i; to.j = 0;
  }
}
\end{lstlisting}
\end{minipage} \\
(a) Old World Model: Stub classes and transformers \\[2ex]
\begin{minipage}{0.62\textwidth}
\begin{lstlisting}[frame=single]
// old_X.Y refers to the new version
// The old X and Y do not exist now
class old_X { public Y y; }
class old_Y { public int i; }

class JvolveTransformers {
  public static void
      jvolveObject(X to, old_X from) {
    to.s = null; to.y = from.y;
  }
  public static void
      jvolveObject(Y to, old_Y from) {
    to.i = from.i; to.j = 0;
  }
}
\end{lstlisting}
\end{minipage} \\
(b) New World Model: Stub classes and transformers \\[2ex]
\hangcaption{Stub classes and transformers for the update in
Figure~\ref{fig:two-classes-change-only-diff}\label{fig:two-classes-change-transformers}}
\EC
\VspaceFixForHangcaption
\end{figure}


In the new world model, the stub classes
\oX and \oY have the same fields (both
types and names) as defined in the old version and provide access to the
fields of old-version objects in a type-safe manner. These stub classes are
used by the Java-to-bytecode compiler to compile object transformer
functions. However, as explained in Section~\ref{sec:loading}, \JV
does not load these classes. It merely renames the old version of the class
that is already loaded by the application to the names of these stubs. This
renaming eliminates the naming conflict between old and new class names. In
the old world model, the runtime replaces stub classes \nX and \nY used while transforming
objects with the new versions of {\tt X} and {\tt Y} before the
application resumes execution.

% Now that we have established sensible type declarations for the \FROM and
% \TO parameters, let us delve a little bit deeper into what the definition
% of \oX and \oY should actually be. The \oY's definition is straightforward.
% It has a single integer field as shown in
% Figure~\ref{fig:two-classes-change}~(c). \oX's definition is a bit more
% involved. The question is whether field {\tt y} should be of type {\tt Y},
% or should it be of type \oY? Do we want it to refer to refer to an object
% of the old version or the object of the new version? Both solutions seem
% reasonable.



% As described in Chapter~\ref{chap:related}, Boyapati et
% al.~\cite{boyapati03lazy} have references to old-version objects. In such a
% system, the programmer has access to the old version's state
% 
% while and
% others~\cite{k42usenix,neamtiu06dsu,neamtiu09stump,upstare} refer to new
% objects. In \JV, we chose the latter approach for the following reason.

\begin{figure}[t]
\begin{center}
\includegraphics[scale=0.48]{100-images/singly-doubly/old-world-singly-doubly-before-transform}
\\
(a) View of the heap before running object transformers \\[1ex]
\includegraphics[scale=0.48]{100-images/singly-doubly/old-world-singly-doubly-transformed}
\\
(b) View of the heap after running object transformers \\
\hangcaption{Old World Model: A view of the linked lists, before and
after running transformation functions\label{fig:old-world-heap}}
\VspaceFixForHangcaption
\end{center}
\end{figure}


The simplest meaningful transformer functions possible are those that copy
fields from the old-version object to the new one. In both models, the
programmer can copy both primitive and reference fields. In the old world view,
reference fields point to old version objects, while in the new world view
they point to new version objects. In the old world view, the dynamic
updating system changes every reference to an old version object to its
corresponding new one, before the application resumes execution.

% With \JV we wanted
% to allow the programmer to effortlessly copy fields with a simple
% assignment such as {\tt to.y = from.y}. This choice forces {\tt \oX.y} to be of
% the same type as {\tt X.y} which in this case happens to be the new
% version's {\tt Y}. If {\tt \oX.y} were pointing to an old-version object,
% the programmer would have to explicitly call a special \VM function that would
% pretend that the types were compatible and force the assignment. The
% system would also have to point to new-version objects as required by the
% new version, before resuming execution.  This conflict motivates our choice
% that the types of these
% fields should be of the new-version, we now consider the
% nature of objects referred to by these fields.  Since these are new-version
% objects and types, the question arises as to whether these objects have
% their contents filled in or not at the time a transformer runs.

% \JV's implementation does not
% % currently
% guarantee at the time {\tt X}'s transformer is invoked whether the
% field {\tt y} points to an object waiting to be transformed (and is therefore
% empty), or whether it points to an object that is transformed and has its
% contents filled in. Enforcing an ordering can get tricky especially in the
% presence of circular references. Moreover, {\tt y}'s state is important
% only if the programmer were to dereference the field. It does not matter if
% the transformer merely copies the field {\tt y}. While not the most elegant
% solution, \JV forces the programmer to specify that {\tt y} be transformed
% on-demand, if the transformer needs to dereference the field.
% 
% In summary, \JV updates program state by allowing \UPT or the programmer
% to specify per type transformation functions. These functions receive as
% parameters an object of the old version's type and an object of the new
% version's type. The fields of the old-version object are of the
% new-version type and point to new-version objects. The objects pointed to
% by these fields may or may not be have been transformed when the
% transformer function for the object containing these fields is invoked. The
% programmer can on-demand force these fields to be transformed.

% \todo{Talk about semantics?} With this model, what are our pitfalls now?
% 1) Transformer code may depends on application semantics.  2) Transformer
% code may depends on where we are, in the application.

% \todo{This might look out of place}

\begin{figure}[t]
\BC
\begin{minipage}{0.8\textwidth}
\begin{lstlisting}[frame=single]
public static void jvolveObject(
            r1_LinkedList to, LinkedList from) {
  to.head = from.head;
  Node prev = null;
  Node current = from.head;
  while (current != null) {
    prev = current;
    current = current.next;
  }
  to.tail = prev;
}
\end{lstlisting}
\end{minipage}
\hangcaption{Old World Model: Object Transformers to convert a
singly-linked list into a doubly-linked list\label{fig:old-world-transformer}}
\EC
\VspaceFixForHangcaption
\end{figure}


\paragraph{Updating from a singly-linked list to a doubly-linked list}

We revisit the singly-linked to doubly-linked example from
Figure~\ref{fig:singly-doubly} to explain transformers in the old world and
new world models.  As mentioned in Section~\ref{sec:xformers}, it is
straightforward to set the {\tt head} pointer of the doubly-linked list.
Setting the {\tt tail} pointer to the last node involves traversing the
linked list. In the old world model, the list is well-formed and the
transformer code obtains the last node with a simple traversal of the list.
The new world model requires special handling since it involves traversing
a list whose nodes are yet to be populated.
% Whereas, in the new world model, special care is required

We first present the old world model. Figure~\ref{fig:old-world-heap} shows
the heap before and after running object transformers. The heap
is logically divided into objects of the old version and those of the new
version. All pointers refer to objects of the old version. The object
transformer functions given in Figure~\ref{fig:old-world-transformer}
populate the empty new version objects. The code to set the {\tt tail}
field of the list involves a simple list traversal. After transforming all
objects the dynamic updating system scans the heap and converts all
pointers to refer to the new version objects and the application can resume
execution. After this phase, the heap is identical to the new world heap
after transformation shown in Figure~\ref{fig:new-world-heap}~(b).

Converting the singly-linked list to a doubly-linked list in the new world
model is a bit more involved. Figure~\ref{fig:new-world-heap} shows a view
of the heap before and after running object transformers in the new world
model. All pointers refer to objects of the new version. The heap is ready
to be used by the application immediately after running the transformers.

\begin{figure}[t]
\begin{center}
\includegraphics[scale=0.48]{100-images/singly-doubly/new-world-singly-doubly-before-transform}
\\
(a) View of the heap before running object transformers \\[1ex]
\includegraphics[scale=0.48]{100-images/singly-doubly/new-world-singly-doubly-transformed}
\\
(b) View of the heap after running object transformers \\
\hangcaption{New World Model: A view of the linked lists, before and
after running transformation functions\label{fig:new-world-heap}}
\VspaceFixForHangcaption
\end{center}
\end{figure}

\begin{figure}[p]
\lstset{frame=single}
\BC \begin{tabular}{@{}c@{}}
% \begin{minipage}{0.85\textwidth}
% \begin{lstlisting}[title={(a) Default \UPT-generated transformer}]
% public static void jvolveObject(
%             LinkedList to, r0_LinkedList from) {
%   to.head = from.head;
%   to.tail = null; // no such field in from
% }
% \end{lstlisting}
% \end{minipage} \\
\begin{minipage}{0.9\textwidth}
\begin{lstlisting}[title={(a) Explicitly traversing old and new version objects}]
public static void jvolveObject(
            LinkedList to, r0_LinkedList from) {
  to.head = from.head;
  Node prev = null;
  Node current = from.head;
  while (current != null) {
    prev = current;
    if (! VM.is_transformed(current)) {
      r0_Node current_old = VM.old_version_object(current);
      current = current_old.next;
    } else {
      current = current.next;
    }
  }
  to.tail = prev;
}
\end{lstlisting}
\end{minipage} \\
\begin{minipage}{0.9\textwidth}
\begin{lstlisting}[title={(b) Explicitly transforming new version objects}]
public static void ensure_transformed(Object o) {
  if (! is_transformed(o)) {
    // The jvolveObject method corresponding to the
    // object's type is invoked using reflection.
    jvolveObject(o, old_version_object(o));
  }
}
public static void jvolveObject(
            LinkedList to, r0_LinkedList from) {
  to.head = from.head;
  Node prev = null;
  Node current = from.head;
  while (current != null) {
    prev = current;
    VM.ensure_transformed(current);
    current = current.next;
  }
  to.tail = prev;
}
\end{lstlisting}
\end{minipage}
\end{tabular}
\vspace*{-2.6ex}
\hangcaption{New World Model: Object Transformers to convert a
singly-linked list into a doubly-linked list\label{fig:setting-tail-field}}
\EC
\lstset{frame=none}
\VspaceFixForHangcaption
\end{figure}


In order to set the {\tt tail} of the list to point to the last node, the
transformer code needs to traverse the list to get to the end. At the time
we traverse the list, not all nodes might have been transformed, because we
cannot always control the order in which objects are transformed. There are two
ways to support this update. One is to allow the programmer to access old
version objects from the new world view by making a request to the runtime.
The other is to provide a way for the programmer to request that an object
be transformed on-demand as the programmer traverses the list to retrieve
the {\tt tail} element of the list.
% Section~\ref{sec:discussion} explores this example in more detail and
% presents alternative models to transform objects.


\paragraph{1. Explicitly traversing old and new-version objects.}
We could traverse the list using both objects of the old version and
objects of the new version. It is important to note that, the system we
currently have ensures type-safety. All reference fields, in the old and
new objects, point to objects of the right type. The issue though is that
fields in new objects might be \NULL because these objects have not
been transformed yet.  The \VM could use meta-data in object headers to provide an \API that allows a
programmer to get the old-version object corresponding to a new-version
object and vice versa. % The \VM could use meta-data in object headers to
% provide this facility.
The programmer should carefully traverse the list,
getting the old-version object of a new-version object, and following the
{\tt next} field of the old-version object.
Figure~\ref{fig:setting-tail-field}~(a) shows the object transformer in
this model.

\paragraph{2. Explicitly transforming new-version objects.}
Another approach would be to traverse the linked list, while ensuring that
the nodes are transformed before we dereference them.
Figure~\ref{fig:setting-tail-field}~(b) shows the object transformer in
this model.  The inherent issue we face is that we want to
enforce an order in which objects are transformed. \JV provides an \API
that allows the programmer to transform a particular object on demand. The
programmer could invoke this function {\tt ensure\_transformed} for each
node as they traverse the list. Note that, this VM function calls {\tt
jvolveObject}, the object transformer if an object is not yet transformed.
This mechanism could potentially lead to a cyclic dependency, i.e., transforming an
object might require that the same object already be transformed.
Currently, we do not
address this problem and leave it to the programmer to guarantee
acyclicity.
We could imagine a more sophisticated approach that analyzes
transformation functions and automatically chooses the order in which they
are invoked. Alternatively, we could automatically instrument the
transformer function to make the required \API calls.
% Like the previous solution, this is inelegant and relies on the
% programmer to get the ordering correct.

% \todo{This was there from the paper. How do we integrate this paragraph
% cleanly?} In our example, the \JO function only copies the contents of the
% old \User object's fields.  More generally, our update model allows old
% object fields to be dereferenced in transformer functions so long as the
% fields point to transformed objects.  If some object $o$ is dereferenced
% while running $p$'s transformer method, but $o$ has not yet been
% transformed, we must find $o$ and pass it and its uninitialized new version
% to the \JO method to initialize it.  Since $p$ points to the new version of
% $o$, we could scan the remainder of the update log to find the old version.
% To avoid this cost, we instead cache a pointer to the old version in the
% new version during the collection.  We take care that \JO functions invoked
% recursively in this manner do not loop infinitely, which would constitute
% one or more ill-defined transformer functions. We detect cycles with a
% simple check, and abort the update.  In our current implementation, the
% programmer uses a special VM function to force a field's referenced object
% to be transformed.  We should be able to handle this case automatically,
% through a read barrier or a simple analysis of the \JO bytecode. 

\subsection{Discussion}
%MWH: moved this. Makes more sense in 2.3
% %MWH: leaving this here, but probably it belongs in 2.3 instead...
% In our update model, if a transformer dereferences an object reference
% which also refers to a transformed object, it refers to the new
% version.  Our experience and that of others%
% ~\cite{k42usenix,neamtiu06dsu,neamtiu09stump,upstare}
% indicate that this model expresses many updates well.  Boyapati et
% al.~\cite{boyapati03lazy} also forbid cycles, but in their model,
% transformers only reference old objects. We explain their model in
% more detail in Section~\ref{sec:related}. We leave to future work a
% detailed investigation of the semantics and expressiveness of both models.

While the object transformer functions written for the old world and new
world models indicate that the old world model provides a simpler logical
model for object transformation, we implemented
the new world model for simplicity of implementation and efficiency. The old world model requires
two traversals of the heap --- one pass to identify objects that need to be
transformed, and another pass to scan and convert all pointers that refer
to old version objects to point to new version ones. The old version model
also requires two steps to load classes of the new version. A first step to
load stub classes to run transformers and another step to load the real
classes of the new version after the update. The new world model on the
other head integrates cleanly with our VM implementation. \JV renames old
classes to stub classes and loads new version classes in a single step.
\JV's modified garbage collector walks through the heap, identifies objects
that need to be updated, creates additional copies for such objects while
ensuring that all references point to these new copies. After invoking
object transformers on all updated objects, the application can resume
execution in the new version.

Our implementation of object transformers uses an extra copy of all updated
objects and adds temporary memory pressure.  We  could copy the old
versions to a special block of memory and reclaim it when the collection
completes. We could avoid extra copies altogether by invoking
object transformer functions during collection.  This approach is more
complicated because our transformer model requires recursively invoking the
collector from the transformer if a dereferenced field has not yet been
processed.  We also would need to use a  GC-time read barrier to follow
forwarding pointers before dereferencing an object in order to determine
whether an object has been transformed.

% In addition, to provide access to the fields 
% of changed objects in transformer functions, we can customize the
% collector to completely process the children first using a postorder
% traversal.  This traversal order would guarantee that any object
% reachable from the object transformer function is sure already initialized.
% Other programming 
% interfaces are possible, too.  For example, Boyapati et al.~\cite{boyapati03lazy}
% take advantage of encapsulation information to allow transformer code
% to access the old versions of pointed-to objects.  Our approach allows
% programs to only see the new versions of objects within transformers,
% similar to the \emph{representation consistency} invariant proposed by
% Stoyle et al.~\cite{StoyleHBSN06}.  We will let future experience
% guide our understanding of which programming interfaces are most
% useful.

% These disadvantages have not proved problematic in practice.  Very often the
% default transformer is sufficient, in which case there is no extra copying.
% The custom object transformer functions we have written are similar to the
% one in Figure~\ref{fig:example-xform}.  By and large, they copy fields
% (either base types or pointers) from the old object to the new one.
% Transformers also tend to allocate new objects for changed or added fields,
% and these new objects may, once again, refer to objects directly pointed to
% by the old object.  In the figure, the new {\tt EmailAddress} objects
% point to the {\tt String}s that used to serve as e-mail addresses in the
% old object.

\subsection{Lazy transformation model}
We use a stop-the-world garbage collection-based approach that requires the
application to pause for the duration of a full heap GC in order to perform
the update. An alternative to this model is applying object transformers
lazily~\cite{ritzau00dynamic,Mala00a,boyapati03lazy,neamtiu06dsu,chen:icse07}.
In a lazy model, the system instruments the application to check, at each
dereference, whether an accessed object is up-to-date and invoke its object
transformer if necessary. The main drawback of this approach is that it
adds overhead during steady state execution. However, the lazy approach is
useful in application contexts that cannot tolerate the pause caused by a
full heap GC.

\BCode[t]{0.67}
\begin{lstlisting}[frame=single]
function getfield(object, field):
    value = object.field
    new_value = ensure_transformed(value)
    if (value != new_value):
        setfield(object, field, new_value)
    return new_value

function getstatic(class, field):
    // similar to getfield
    // uses setstatic

function aaload(array, index):
    // similar to getfield
    // uses aastore

function ensure_transformed(o):
    if o.isForwardingPointer():
        return o.ForwardedValue
    else if o.isUpToDate():
        return o
    else:
        new_o = new Object()
        jvolveObject(new_o, o)
        o = (forwarding pointer to new_o)
        return new_o
\end{lstlisting}
\ECode{Lazy object transformation model
implementation\label{fig:read-barrier}}


The lazy approach can be implemented in \JV by performing the following
steps. 1) At update time, invoke object transformers on all stack local
variables. 2) Maintain the invariant that any object accessed by the
application is up-to-date. We can maintain this invariant by adding
read-barrier code to all object dereferences. An object can be dereferenced
in Java only using one of the following three opcodes: {\tt getfield} that
returns a field from an object; {\tt getstatic} that returns the value of a
global variable; and {\tt aaload} that return an array element. The
read-barrier code to check whether or not an object is transformed can use
the same forwarding pointer mechanism used by the garbage collector while
scanning the heap.  Figure~\ref{fig:read-barrier} shows the code that
implements this functionality.  The first time an object is accessed after
the update, it will be transformed leaving a forwarding pointer in its
place. Future accesses to the object that go through the forwarding pointer
return the transformed object and update the corresponding reference. The
forwarding pointer will be garbage collected away when no field refers to
it.

While the lazy transformation model reduces pause time, it is not as
flexible as the stop-the-world eager transformation model. When using the
lazy transformation model, the programmer must take special care to ensure
that application correctness is not affected by when objects are
transformed.  For example, in the update that goes from a singly to
doubly-linked list, an object's {\tt prev} field is set by its previous
object's transformer and not by itself. No general mechanism for obtaining
the previous object exists without eagerly transforming the list.
When the application accesses an
object's {\tt prev} field and the field is {\tt null}, there is no way to
know if it is actually {\tt null} or whether it is yet to have its value set.
Thus, lazy transformation limits the types of updates a dynamic updating
system can support.

% This pause time could be mitigated by
% piggybacking on top of a concurrent
% collector.  We could also consider applying object and 
% class transformers lazily, as they are
% needed~\cite{ritzau00dynamic,Mala00a,boyapati03lazy,neamtiu06dsu,chen:icse07}.  The
% main drawback here is efficiency. The VM would need to insert code to check, at each
% dereference, whether the object is up-to-date, imposing extra overhead
% on steady-state execution.  Moreover, stateful actions by the program after
% an update may invalidate assumptions made by object transformer
% functions. It is possible that a hybrid solution could be adopted,
% similar to Chen et al.~\cite{chen:icse07}, which removes checking
% code once the system updates all objects.  We leave
% exploration of these ideas to future work.

% vim:spell:ft=tex:
