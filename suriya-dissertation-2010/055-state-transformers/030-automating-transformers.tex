\section{Automating State Transformer Generation}
\label{sec:automate}

Thus far, we have relied on the programmer to provide heap transformer
functions. In this section, we explore techniques to automate generation of
the transformer based on the patch to ease programmer burden.

This section presents a technique that starts from a patch and
automatically produces state transformers that repair application state.
Our methodology takes into account the code fix for a particular
heap-corrupting bug, and uses a combination of dynamic and static analysis
to discover predicates on heap objects.
% object transformers can use to repair the heap.
We employ the garbage collector to invoke object transformers on relevant
heap objects satisfying the discovered heap predicates. While our focus is
on memory leaks, our technique is general and we have applied it to other
sorts of conditional heap updates as well.

Our technique works as follows. Suppose we have a Java program in which
field \texttt{f} of class \texttt{C} is the source of a memory leak, and
the fix is to insert a line \texttt{x.f = null;} in some method
\texttt{foo}. Figure~\ref{fig:simple-automatic-transformers} shows the
patch for this method. The new version sets field {\tt f} to {\tt null}
causing the object pointed to by {\tt f} to be garbage collected and freed.
% (assuming \texttt{x} has type \texttt{C})
% so that the object
% pointed to by \texttt{f} is properly garbage collected (GCed).
Given this fix, we would like to dynamically update the program to correct
the code of \texttt{foo}. The patch prevents further leaks, and our
object transformer needs to modify the existing leak to free already-leaked
objects. We use a combination of dynamic and static analysis to
automatically generate the object transformer function.

\BCode[t]{0.38}
\begin{lstlisting}[frame=single]
  public void foo() {
    C x;
    ...
    if (...) {
      ...
    } else {
      ...
+     x.f = null;
    }
  }
\end{lstlisting}
\ECode{Example of a simple patch that fixes a memory leak\label{fig:simple-automatic-transformers}}



The object transformer function must find existing objects \texttt{x} of
class \texttt{C} (or a subclass of \texttt{C}) and assign \texttt{null} to
\texttt{x.f}.  But we must be careful to only null out \texttt{x.f} if that
object is actually leaked; if we null out \texttt{x.f} while the program is
still using it, then the update will introduce incorrect behavior or
a crash.  We thus need to distinguish leaked objects from in-use ones. In
Figure~\ref{fig:simple-automatic-transformers}, the fixed version of
\texttt{foo} introduces the statement \texttt{x.f = null;} on line 8.  The
\emph{leaky objects} consist of all objects \emph{o} that have been bound
to \texttt{x} during execution (of the old program) that have reached this
line.  If the fix had been in place, \texttt{x.f = null} would have been
executed, and thus the \texttt{f} field of these objects would have been
set to null.  All other objects of class \texttt{C} (or subtypes of
\texttt{C}) that have \emph{not} been bound to \texttt{x} at line 8 are not
leaking, and so we should leave their \texttt{f} field alone.  We
call these the \emph{non-leaky objects}.

The problem with literally implementing this idea is that the bug is
discovered, and the fix discerned, only after the program is deployed.  At
this point, the patch has no information about which objects reached line 8
and which did not. To remedy this problem, as part of update preparation, we attempt to generate a
\emph{heap predicate} offline, that unambiguously distinguishes the leaky and
non-leaky objects. Then we
generate a dynamic patch to correct the heap:

\begin{center}
\begin{minipage}{0.53\textwidth}
\begin{lstlisting}[frame=single]
for all objects x of class D < C:
   if heap_pred(x):
       x.f = null
\end{lstlisting}
\end{minipage}
\end{center}

The code to fix a particular object's state (line 3) comes from the code
patch and is applied to all objects of type D satisfying the heap
predicate.  We execute this code at update time, together with other object
and class transformers. We use a modified version
of the \JV garbage collector to traverse the heap to find the matching
objects and null their fields.

We generate the heap predicate by performing a dynamic analysis. We run the
old program and \emph{mark} all objects of class C that reach the said
line. 
In our current implementation, we modify the concerned class and explicitly add a boolean field named {\tt \_\_marked}.
To mark an object, we set this field to {\tt true}.
In addition, each time program execution reaches a
legal update point, we dump the heap. We use the Sun JVM's heap dumping
support. Our goal is to now discover a heap predicate that distinguishes
marked objects from the unmarked ones, i.e., {\tt (heap-predicate(x) ==
true)} if
and only if {\tt (marked(x) == true)}. We do this by feeding information about
marked objects to an invariant detection tool. For our work, we use Daikon,
a tool that discovers dynamic invariants in a running program and across
multiple runs of the program~\cite{daikon}. We construct a special Java
program and run it through Daikon. The program contains a function whose
parameters match the signature of the concerned class. The main method
reads each marked object from the snapshot and calls this function with the
fields of the object as parameters. We control how deep to look for
invariants by passing the contents of referred objects, objects pointed by
them, and so on. We also pass the type and count of references pointing to
marked objects.
We run this Java program though Daikon which infers invariants among the
parameters of our specially constructed function. The invariants that hold
across all runs can be used as predicates to identify marked objects.
% These snapshots come from several
% legal update points in the source program, and from numerous runs of the
% source program.
However, we can use only those predicates that definitely \emph{do not}
hold for any unmarked object. To do so, we negate each predicate identified
by Daikon and check whether the negated predicate holds for every unmarked
object in all heap snapshots. A predicate that is satisfied by every
marked object and is not satisfied by even a single unmarked object can be
used at runtime to distinguish marked and unmarked objects. We use the
conjunction of all predicates identified by this process as our heap
predicate to identify leaky objects.

% We then only keep the those invariants inferred by
% Daikon that definitely \emph{do not} hold for the unmarked objects. The
% final heap predicate is the conjunction of the remaining invariants, which
% we can be sure uniquely identify the marked from unmarked objects.

There are two caveats that we need to keep in mind. First, this is a
dynamic analysis, so we need adequate test coverage to be convinced that
the predicates we have inferred are general. Second, we need to consider
that some objects should be unmarked
if the program subsequently writes to the field that the patch modifies.
In the example in
Figure~\ref{fig:simple-automatic-transformers}, the absence of line~8 in
the old version is a leak. However, it might be the case that there is a
future legitimate write to field {\tt f} that makes the object non-leaky.
% However, we cannot assume that the field {\tt f} not being {\tt null} constitutes a leak.
Therefore, whenever there is a write to field {\tt f}, we unmark the
concerned object, i.e., change its identification to non-leaky.
%
In general, whenever a leak occurs in the old version we mark the concerned
object as leaky, and after any access to that object that invalidate its
leaky state we mark it as non-leaky.
% For example, if failing to remove an
% object from a collection constitutes a leak,
% not removing an object
% from

This approach can be generalized for other simple patches.
For example, a patch that either adds a boolean field in the new version,
or fixes incorrect setting of the field in the old version. In both cases,
we run the new version and discover dynamic invariant relationships between
the value of the concerned boolean field and other objects in the heap. We
use this discovered predicate to set the new field's value with the
following state transformer:

\begin{center}
\begin{minipage}{0.53\textwidth}
\begin{lstlisting}[frame=single]
for all objects x of class D < C:
    if heap-pred(x):
        x.boolean_field = true
\end{lstlisting}
\end{minipage}
\end{center}

\subsection{Invariants discovered from real fixes}
We applied this methodology to three real fixes and present the
automatic state transformers we obtained. Two of them are similar, and are for
the memory leaks fixed in jEdit by SVN revisions 5178 and 14027, mentioned
in Table~\ref{tab:bugfixes}. The other patch is from the Bittorent client
Azureus. The new version adds a boolean field {\tt isExpanded} for a GUI
element.

\paragraph{jEdit fixes in SVN revision 5178 and 14027}
Figure~\ref{fig:r14027} shows the patch and transformer for the leaky field
{\tt tokenHandler}. For this version, we discover that the following heap
predicate distinguishes leaky objects from the non-leaky ones:
{\tt (o.tokenHandler != null)}.
% \begin{center}
% \begin{minipage}{0.58\textwidth}
% \begin{lstlisting}[frame=single]
% (o.tokenHandler != null)
% \end{lstlisting}
% \end{minipage}
% \end{center}
This predicate validates our transformer in Figure~\ref{fig:r14027}~(b),
which sets all leaky fields to {\tt null}.

\paragraph{Adding a new boolean field in Azureus}
The version number of the fix is {\tt a04dc20b6b} in our Git repository of
Azureus' CVS repository. This bugfix corresponds to commits made around
2010/02/14. The new version adds a field {\tt isExpanded} to class {\tt
BaseMdiEntry}. The GUI element corresponds to a tree that displays data,
and the {\tt isExpanded} field specifies the state of a particular node in
the tree. For this update, the default transformer generated by our
\acf{UPT} would set the boolean field to {\tt false}. However,
a dynamic analysis on the new version of the application produces
the following invariants among objects of type {\tt BaseMdiEntry}:

\begin{center}
\begin{minipage}{0.8\textwidth}
\begin{lstlisting}[frame=single]
this.iconBarEnabler.visible == this.sidebar.visible
this.isExpanded == this.soParent.paintListenerHooked
\end{lstlisting}
\end{minipage}
\end{center}

The second invariant set the value of the {\tt isExpanded} field
in the object transformer. Stephen Magill collaborated on studying this
fix, ran it through Daikon, and discovered the
above heap predicate.

% Here we summarize our results.  Talk about this working for leaks that
% requiring nulling fields, leaks that are fixed by function calls, and
% other bugs (initialization errors).
%
% We also need to talk about our other technique, if that's going into
% the paper.  What is the story there?
