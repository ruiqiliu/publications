\section{Dynamic Software Updating for C/C++}

There are several substantial systems for dynamically updating C and C++
programs that target server applications~\cite{HjalmtyssonG98,
altekar05opus, neamtiu06dsu, chen:icse07, upstare, neamtiu09stump} and
operating systems components~\cite{K42reconfig, k42usenix, chen06vee,
lee06linuxmod, dynamos_eurosys_07}.  While these systems are mature
and offer substantial updating experience, the flexibility afforded by \JV
is comparable or superior. 

\JV's timing restrictions and Java's type
safety also provide comparable or superior safety. Because C is a type-unsafe
language, DSU systems for C have to restrict certain unsafe C idioms and
perform conservative analysis to enforce type-safety of updates.
% ; the fact that C and C++
% programs often circumvent the languages' weak type systems greatly
% complicates efforts to ensure that updates behave correctly. 
The lack of a VM is a disadvantage for
C and C++ DSU.  For example, because a VM-based JIT can compile and
recompile replacement classes, it imposes no steady-state execution
overhead.  By contrast, C and C++ implementations must use either
statically-inserted indirections~\cite{HjalmtyssonG98, neamtiu06dsu,
K42reconfig, k42usenix, upstare} or dynamically-inserted trampolines to
redirect function calls~\cite{altekar05opus, chen06vee, chen:icse07,
ksplice}.  Both cases impose persistent space and time overhead on normal
execution and inhibit optimization.  Likewise, because these systems lack a
garbage collector, they either do not update object instances at
all~\cite{ksplice}, update them lazily~\cite{neamtiu06dsu, chen:icse07} or
perform extra allocation and bookkeeping to locate the objects at
update-time~\cite{k42usenix}.

Because these systems lack support
for on-stack replacement, they must pre-compile potentially long-running
methods specially, so that they can be updated while they run. These
techniques impose time and space overheads on steady-state execution, and
in some cases limit update flexibility.
Some prior
systems~\cite{neamtiu09stump, upstare, chen:icse07} have focused on means
to reach DSU safe points quickly, and \JV is comparable in support when it
comes to single-threaded applications. For multithreaded applications, {\sc
Stump}'s synchronization of multiple threads to reach a safe
point~\cite{neamtiu09stump} and Upstare's support to update active methods
on stack~\cite{upstare} are superior to what \JV provides. While we did not
implement such features in \JV, its model does not preclude such support.

\subsection{K42 Operating System}\index{K42}

K42 is an object-oriented research operating system with dynamic updating
support out of IBM Research. Updates are performed at the granularity of an
{\em object instance}. K42 is structured as a set of objects, with objects
exporting an interface by declaring a virtual base class. All objects that
are updated must provide state transfer functions to export the old version
to a common format and import from the common format to the new version.
K42 reaches a safe point by reaching a quiescent state. The OS services
requests by creating a new kernel thread for each request. To perform an
update, K42 restricts new accesses to updated objects, waits until all prior
requests/threads have completed, and then performs the update. Finally, to
support access to the new version of an object, K42 uses an {\em object
translation table}. Each object in the system has an entry in the table,
and all object invocations go through the table. At update time, K42
modifies the table to invoke the new versions of an object.

\subsection{Ksplice}\index{Ksplice}

Ksplice~\cite{ksplice} is a dynamic updating system for the Linux kernel.
Ksplice is very easy to use and takes as input a source patch to the
currently running kernel, and generates a binary patch that can be applied
to the running kernel. Ksplice only supports changes to method bodies, and
does not support function or type signature changes. However, this
flexibility has been sufficient to support most patches that fix security
vulnerabilities. Like all C systems, Ksplice needs some form of function
indirection to jump to the new version of a method. Ksplice does so by
installing {\em trampolines} at update time. Ksplice overwrites the first
few instructions of the old version's method body with a jump instruction
to the new method. Future calls to an updated method jump through this
trampoline to the latest version's body.  Ksplice uses activeness check and
prevent updates when changed methods are active on some thread's stack.

\subsection{Ginseng}\index{Ginseng}

Ginseng~\cite{neamtiu06dsu, neamtiu-thesis} is a \USD system for
C server applications, and is very flexible in supporting changes to method
bodies, method signatures, structure definitions, and global data.
Ginseng uses standard techniques needed of \USD systems without a managed
runtime to support updates to code and data. Ginseng uses function
indirection to make calls to the right version of a method and ``type
wrapping'' to check that accessed data has the right version of a type.
Ginseng's offline patch generator generates state transformers for updated
types, and the runtime uses padding to accommodate additional space required
by new versions of objects. The application invokes state transformers
lazily when it first accesses an object after the update.

While Ginseng does not support changes to active methods on stack,
programmers may annotate long running loops, extract loop bodies
into their own methods, and then update between loop iterations. To
guarantee type-safety, Ginseng must first deal with the type-unsafeness of
C\@. Ginseng prohibits the use of certain C idioms that are {\em unsafe}.
% and performs an {\em abstraction violation alias analysis} for types, and
% {\em updatability analysis} that guarantees type-safety of programmer
% specified update points.
Ginseng also introduced a notion of safety called
{\em transaction safety} in which programmers mark transaction regions. The
system's analysis restricts certain update points within a region to ensure
that a transaction runs entirely as the old version or as the new version.

\subsection{Upstare}\index{Upstare}

Upstare~\cite{upstare, upstare-thesis} supports dynamic updates to
multi-threaded C programs. Upstare performs a whole-program compilation and
extensively instruments an application to make it update-compatible.
Upstare uses function indirection for function calls,
instruments function entry and exit points, and loop back edges to
guarantee ``immediate updates'' when the user signals the application to
update. These instrumentation points are precisely the same \VM safe points
that \JV uses. Like \JV, Upstare allows the programmer to specify update
points by making \API calls from within the application, or initiate an
immediate update upon receiving a signal from a user. Upstare suspends all
application threads and performs atomic updates, just like \JV does. It
relies on complex synchronization to suspend all but one application
thread, the {\em update co-ordinator} thread which performs the update.
Upstare cannot perform an update when threads are waiting on
blocking system calls. This restriction is especially problematic for
multithreaded server application threads which are often blocked waiting
for requests. \JV's thread synchronization mechanism is much simpler and
supports blocking system calls in a straightforward manner.

Upstare's support for {\em stack reconstruction} is unique among dynamic
updating systems.
% is an improvement over \JV.
Upstare extracts the state of a function's stack, and
reconstructs it as expected by the new version of a function. This
reconstruction support enables even modified functions to be active on
stack. Stack reconstruction requires the programmer to specify state
transformers for the local variables of functions, and correspondence
between execution points in the old and new function bodies. \JV restricts
its \acf{OSR} support to methods with identical bytecodes in the old and
new versions. As a result, \JV currently does not require a stack state
transformer or a mapping or execution. There is no conceptual reason why \JV
wouldn't be able to support an extended \OSR, but for the additional burden
on the programmer.
