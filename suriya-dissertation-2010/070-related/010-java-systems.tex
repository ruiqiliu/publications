\section{Dynamic Software Updating for managed languages}

Researchers have adopted several approaches to bring \USD support to managed
languages. These include special-purpose compilation, class loaders,
or \VM-support. The main drawback of approaches that do not change the \VM
are inflexibility and high overhead.

\subsection{Edit and continue development}

Debuggers and IDEs have long provided \emph{edit and continue} (E\&C)
functionality that permits limited modifications to program state to avoid
stopping and restarting during debugging. For example, Sun's HotSwap
VM~\cite{JVMhotswap, Dmit01a}, .NET Visual Studio for C\# and
C++~\cite{VSEnC}, and library-based support~\cite{eaddy05enc} for .NET
applications all provide E\&C.  These systems restrict updates to code
changes within method bodies.  While this restriction reduces safety
concerns and obviates the need for class and object transformers, the
resulting systems are inflexible. A request for enhanced dynamic updating
has the fourth highest number number of votes among enhancement requests
recorded in the Hotspot VM bug database~\cite{voted-bug,bug-rank-list}.
These systems cannot perform more than half of the
updates discussed in Chapter~\ref{chap:experience}.

% To avoid changing the \VM to support \USD, researchers have developed
% special-purpose classloaders and/or compiler support~\cite{javarebel,
% BarrE03, Milazzo05updates}. The main drawbacks of these approaches are
% inflexibility and high overhead.

\subsection{Solutions without VM-support}

JRebel~\cite{javarebel} is a productivity tool targeting Java EE
(Enterprise Edition) developers. JRebel watches changes to the source tree
of a web application under development and applies these changes to a
running VM. Developers do not have to restart their application after every
change, thereby speeding up development. JRebel is implemented on top of
the Sun Hotspot VM's instrumentation API. When loading a class, JRebel
rewrites the bytecodes of all methods, intercepting all method calls and
field accesses. JRebel implements method and field accesses by performing a
dictionary lookup, and incurring a high performance overhead, like in an
interpreted language such as Python or Ruby.  When a new version of a class
is available, JRebel updates the dictionary to point to the new method
versions. While JRebel supports addition and deletion of fields, it does
not update the state of existing objects, rendering updates type-unsafe. It
also has no notion of update timing for safety. Because of the performance
overhead and the lack of type safety, JRebel is suited mainly to ease
debugging during development.

Barr and Eisenbach~\cite{BarrE03} support updates to Java libraries in a
client server model. The system supports a subset of updates to Java
programs that do not cause linker errors.  Milazzo
\EA~\cite{Milazzo05updates} support dynamic updating in a distributed
computing environment. Their work proposes a specialized software
architecture to monitor updates, and distribute them across the
application.  Both systems use custom classloaders for binary-compatible
and component-level changes respectively, but cannot support signature
changes such as class field additions.

%% Eisenbach and Barr: safe upgrading without restarting.  They support
%% upgrades that satisfy binary compatibility.  Uses a custom classloader and
%% JMX to replace the code of existing objects.  No way to modify the state of
%% the objects.

%% Milazzo use a modified class loader to load individual replacements to
%% classes in a special-purpose architecture.  The class loader may modify the
%% bytecode of the loaded class to deal with type version namespace issues.
%% Basically this is more limited in scope than our approach.

Orso et al.~\cite{orso:java} use source-to-source translation for DSU by
introducing a proxy object that indirectly accesses an object that may
change. For each class C that might change in the future they produce a
proxy for that class.  All calls from clients of C are redirected to call
the wrapper instead.  When C is updated by some new class C', a new C'
object is created and initialized using the old state of C and the wrapper
is redirected to point to C'. This approach requires updated classes to
export the same public interface, forbidding new public methods and fields.

Non VM-based approaches are in general limiting because they are not
\emph{transparent}---they make visible changes to the class hierarchy, and
insert or rename classes. This approach makes it essentially impossible to
be robust in the face of code using reflection or native methods.
Moreover, the indirection imposes time and space overheads on steady-state
execution.  Our VM approach naturally supports reflection and native
methods (these are updated as well), is more expressive, e.g., it
supports signature changes, and imposes no overhead on steady-state
execution.

\subsection{VM support for DSU in managed languages} 

The PROSE system performs short-term, run-time patches to code for logging,
introspection, and performance adaptation, rather performing general
updates~\cite{nicoara:eurosys08}.  An Eclipse plug-in performs run-time
bytecode instrumentation and a modified JIT performs method code
replacement, using an API in the style of aspect-oriented programming.
PROSE has the same update model as the E\&C systems: it supports updates to
method bodies but not class or method signature changes that require
changes to object state.  This flexibility is similar to the E\&C
implementations discussed above; indeed, PROSE builds on the HotSwap method
replacement support in its Sun JDK implementation~\cite{JVMhotswap}.

JDrums~\cite{ritzau00dynamic} and the Dynamic Virtual Machine
(DVM)~\cite{Mala00a} both implement DSU for Java within the VM, providing a
programming interface similar to \JV, but are lacking in two ways.  First,
neither JDrums nor DVM have ever been demonstrated to support updates from
real-world applications.  Second, their implementations impose overheads
during steady-state execution.  They both update \emph{lazily} and use an
extra level of indirection (the \emph{handle space}).  Indirection
conveniently supports object updates, but adds extra overhead.  For
example, JDrums traps all object pointer dereferences to apply VM object
transformer function(s) when the object's class changes.  Lazy updating has
the advantage that it amortizes  pauses due to an update over subsequent
execution.  The main drawback is that its overhead persists during normal
execution, even though updates are relatively rare.  DVM works only with
the interpreter.  Relative to this interpreter, which is already slow, the
extra traps result in roughly 10\% overhead.

Compared to these two, \JV performs updates eagerly by employing a full
heap collection at update-time.  This stop-the-world approach imposes a
longer pause at update time, but eliminates overhead during steady-state
execution.  Likewise, by invalidating updated methods, \JV's performance is
slowed just after the update as these methods are being recompiled.
However, compared to running with an interpreter, steady-state execution is
much improved, since methods will be much better optimized.

\subsection{Dynamic ML}

Gilmore et al.~\cite{GilmoreKW97} propose DSU support for modules in ML
programs using a similar, but more restrictive programming interface
compared with \JV.  They formalize an abstract machine for implementing
updates using a copying garbage collector.
Duggan~\cite{DBLP:journals/acta/Duggan05} also proposes dynamic updates to
ML programs, focusing on lazy updates to data type definitions.  Neither
approach was ever implemented.

\subsection{Language support for \USD}

UpgradeJ~\cite{bierman08upgradej} is an extension to the Java language
design supporting class upgrades, in two flavors: \emph{revision upgrades},
which may modify method bodies, and \emph{evolution upgrades}, which may
add new methods and fields.  Programmers control the effects of upgrades
using \emph{version annotations}, introduced by Bierman et
al.~\cite{BiermanHSS03}.  For example, the programmer may
write \texttt{o = new Button[1=]()} to force \texttt{o} to always use
version~1 methods, while writing \texttt{p = new Button[1+]()} or \texttt{p
= new Button[1++]()} allows \texttt{p} to be revised or evolved,
respectively.  UpgradeJ's update model is easier to implement than \JV's
because it need not change existing object instances.  Of course, the
downside is a loss of flexibility.  Many of the updates to our benchmark
applications change field contents and layout.  UpgradeJ does not support
these updates.  On the other hand, evolution upgrades add power over simple
method body updates, and consequently enable more real-world updates to be
supported~\cite{tempero08upgradej}.  There currently is no implementation of
UpgradeJ.

%% JDRUMS: implements the conversion lazily.  They have a similar interface
%% (object and class transformers). The drawback here is that there is overhead
%% in the general case of execution---we do not know when the update is
%% complete.  Implemented in Sun's JDK 1.2.  Adds a level of indirection to the
%% new object.  Thus overhead builds up over time.  It also appears they have a
%% more limited interface to what can be referenced in a conversion function.
%% For example, there is no way to refer to fields other than those of the
%% object's class (i.e., no super-class fields) and there is no way to call new
%% methods, like constructors.  Not clear if there are restrictions on how
%% methods can be changed.

%% DVM: use an incremental mark-sweep collector, where mark phase marks objects
%% to be updated and the sweep phase incrementally updates them (prior to being
%% accessed by the mutator).  Like JDRUMS, all accesses to marked objects are
%% trapped.  Imposes a stock 10\% overhead, even only using bytecode.

%% Both of these: no significant experience with real applications, according
%% to how they change in practice.  They also can't handle native methods
%% because they can't trap access to modified objects.  Doing the full GC
%% solves this problem.

% More recently, Nicoara et al. developed PROSE, a system for run-time
