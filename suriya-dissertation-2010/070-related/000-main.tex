\chapter{Related Work}
\label{chap:related}

Researchers have designed and implemented dynamic updating solutions in a variety of contexts, ranging from
theoretical models of updates and safety, to language design for dynamic
updating, to practical \USD systems for C and Java, and to operating systems
with \USD support. In this chapter, we compare our VM-centric approach to
DSU with related work on implementing \USD for managed languages, C, and
C++.

\section{Dynamic Software Updating for C/C++}

There are several substantial systems for dynamically updating C and C++
programs that target server applications~\cite{HjalmtyssonG98,
altekar05opus, neamtiu06dsu, chen:icse07, upstare, neamtiu09stump} and
operating systems components~\cite{K42reconfig, k42usenix, chen06vee,
lee06linuxmod, dynamos_eurosys_07}.  While these systems are mature
and offer substantial updating experience, the flexibility afforded by \JV
is comparable or superior. 

\JV's timing restrictions and Java's type
safety also provide comparable or superior safety. Because C is a type-unsafe
language, DSU systems for C have to restrict certain unsafe C idioms and
perform conservative analysis to enforce type-safety of updates.
% ; the fact that C and C++
% programs often circumvent the languages' weak type systems greatly
% complicates efforts to ensure that updates behave correctly. 
The lack of a VM is a disadvantage for
C and C++ DSU.  For example, because a VM-based JIT can compile and
recompile replacement classes, it imposes no steady-state execution
overhead.  By contrast, C and C++ implementations must use either
statically-inserted indirections~\cite{HjalmtyssonG98, neamtiu06dsu,
K42reconfig, k42usenix, upstare} or dynamically-inserted trampolines to
redirect function calls~\cite{altekar05opus, chen06vee, chen:icse07,
ksplice}.  Both cases impose persistent space and time overhead on normal
execution and inhibit optimization.  Likewise, because these systems lack a
garbage collector, they either do not update object instances at
all~\cite{ksplice}, update them lazily~\cite{neamtiu06dsu, chen:icse07} or
perform extra allocation and bookkeeping to locate the objects at
update-time~\cite{k42usenix}.

Because these systems lack support
for on-stack replacement, they must pre-compile potentially long-running
methods specially, so that they can be updated while they run. These
techniques impose time and space overheads on steady-state execution, and
in some cases limit update flexibility.
Some prior
systems~\cite{neamtiu09stump, upstare, chen:icse07} have focused on means
to reach DSU safe points quickly, and \JV is comparable in support when it
comes to single-threaded applications. For multithreaded applications, {\sc
Stump}'s synchronization of multiple threads to reach a safe
point~\cite{neamtiu09stump} and Upstare's support to update active methods
on stack~\cite{upstare} are superior to what \JV provides. While we did not
implement such features in \JV, its model does not preclude such support.

\subsection{K42 Operating System}\index{K42}

K42 is an object-oriented research operating system with dynamic updating
support out of IBM Research. Updates are performed at the granularity of an
{\em object instance}. K42 is structured as a set of objects, with objects
exporting an interface by declaring a virtual base class. All objects that
are updated must provide state transfer functions to export the old version
to a common format and import from the common format to the new version.
K42 reaches a safe point by reaching a quiescent state. The OS services
requests by creating a new kernel thread for each request. To perform an
update, K42 restricts new accesses to updated objects, waits until all prior
requests/threads have completed, and then performs the update. Finally, to
support access to the new version of an object, K42 uses an {\em object
translation table}. Each object in the system has an entry in the table,
and all object invocations go through the table. At update time, K42
modifies the table to invoke the new versions of an object.

\subsection{Ksplice}\index{Ksplice}

Ksplice~\cite{ksplice} is a dynamic updating system for the Linux kernel.
Ksplice is very easy to use and takes as input a source patch to the
currently running kernel, and generates a binary patch that can be applied
to the running kernel. Ksplice only supports changes to method bodies, and
does not support function or type signature changes. However, this
flexibility has been sufficient to support most patches that fix security
vulnerabilities. Like all C systems, Ksplice needs some form of function
indirection to jump to the new version of a method. Ksplice does so by
installing {\em trampolines} at update time. Ksplice overwrites the first
few instructions of the old version's method body with a jump instruction
to the new method. Future calls to an updated method jump through this
trampoline to the latest version's body.  Ksplice uses activeness check and
prevent updates when changed methods are active on some thread's stack.

\subsection{Ginseng}\index{Ginseng}

Ginseng~\cite{neamtiu06dsu, neamtiu-thesis} is a \USD system for
C server applications, and is very flexible in supporting changes to method
bodies, method signatures, structure definitions, and global data.
Ginseng uses standard techniques needed of \USD systems without a managed
runtime to support updates to code and data. Ginseng uses function
indirection to make calls to the right version of a method and ``type
wrapping'' to check that accessed data has the right version of a type.
Ginseng's offline patch generator generates state transformers for updated
types, and the runtime uses padding to accommodate additional space required
by new versions of objects. The application invokes state transformers
lazily when it first accesses an object after the update.

While Ginseng does not support changes to active methods on stack,
programmers may annotate long running loops, extract loop bodies
into their own methods, and then update between loop iterations. To
guarantee type-safety, Ginseng must first deal with the type-unsafeness of
C\@. Ginseng prohibits the use of certain C idioms that are {\em unsafe}.
% and performs an {\em abstraction violation alias analysis} for types, and
% {\em updatability analysis} that guarantees type-safety of programmer
% specified update points.
Ginseng also introduced a notion of safety called
{\em transaction safety} in which programmers mark transaction regions. The
system's analysis restricts certain update points within a region to ensure
that a transaction runs entirely as the old version or as the new version.

\subsection{Upstare}\index{Upstare}

Upstare~\cite{upstare, upstare-thesis} supports dynamic updates to
multi-threaded C programs. Upstare performs a whole-program compilation and
extensively instruments an application to make it update-compatible.
Upstare uses function indirection for function calls,
instruments function entry and exit points, and loop back edges to
guarantee ``immediate updates'' when the user signals the application to
update. These instrumentation points are precisely the same \VM safe points
that \JV uses. Like \JV, Upstare allows the programmer to specify update
points by making \API calls from within the application, or initiate an
immediate update upon receiving a signal from a user. Upstare suspends all
application threads and performs atomic updates, just like \JV does. It
relies on complex synchronization to suspend all but one application
thread, the {\em update co-ordinator} thread which performs the update.
Upstare cannot perform an update when threads are waiting on
blocking system calls. This restriction is especially problematic for
multithreaded server application threads which are often blocked waiting
for requests. \JV's thread synchronization mechanism is much simpler and
supports blocking system calls in a straightforward manner.

Upstare's support for {\em stack reconstruction} is unique among dynamic
updating systems.
% is an improvement over \JV.
Upstare extracts the state of a function's stack, and
reconstructs it as expected by the new version of a function. This
reconstruction support enables even modified functions to be active on
stack. Stack reconstruction requires the programmer to specify state
transformers for the local variables of functions, and correspondence
between execution points in the old and new function bodies. \JV restricts
its \acf{OSR} support to methods with identical bytecodes in the old and
new versions. As a result, \JV currently does not require a stack state
transformer or a mapping or execution. There is no conceptual reason why \JV
wouldn't be able to support an extended \OSR, but for the additional burden
on the programmer.

\section{Dynamic Software Updating for managed languages}

Researchers have adopted several approaches to bring \USD support to managed
languages. These include special-purpose compilation, class loaders,
or \VM-support. The main drawback of approaches that do not change the \VM
are inflexibility and high overhead.

\subsection{Edit and continue development}

Debuggers and IDEs have long provided \emph{edit and continue} (E\&C)
functionality that permits limited modifications to program state to avoid
stopping and restarting during debugging. For example, Sun's HotSwap
VM~\cite{JVMhotswap, Dmit01a}, .NET Visual Studio for C\# and
C++~\cite{VSEnC}, and library-based support~\cite{eaddy05enc} for .NET
applications all provide E\&C.  These systems restrict updates to code
changes within method bodies.  While this restriction reduces safety
concerns and obviates the need for class and object transformers, the
resulting systems are inflexible. A request for enhanced dynamic updating
has the fourth highest number number of votes among enhancement requests
recorded in the Hotspot VM bug database~\cite{voted-bug,bug-rank-list}.
These systems cannot perform more than half of the
updates discussed in Chapter~\ref{chap:experience}.

% To avoid changing the \VM to support \USD, researchers have developed
% special-purpose classloaders and/or compiler support~\cite{javarebel,
% BarrE03, Milazzo05updates}. The main drawbacks of these approaches are
% inflexibility and high overhead.

\subsection{Solutions without VM-support}

JRebel~\cite{javarebel} is a productivity tool targeting Java EE
(Enterprise Edition) developers. JRebel watches changes to the source tree
of a web application under development and applies these changes to a
running VM. Developers do not have to restart their application after every
change, thereby speeding up development. JRebel is implemented on top of
the Sun Hotspot VM's instrumentation API. When loading a class, JRebel
rewrites the bytecodes of all methods, intercepting all method calls and
field accesses. JRebel implements method and field accesses by performing a
dictionary lookup, and incurring a high performance overhead, like in an
interpreted language such as Python or Ruby.  When a new version of a class
is available, JRebel updates the dictionary to point to the new method
versions. While JRebel supports addition and deletion of fields, it does
not update the state of existing objects, rendering updates type-unsafe. It
also has no notion of update timing for safety. Because of the performance
overhead and the lack of type safety, JRebel is suited mainly to ease
debugging during development.

Barr and Eisenbach~\cite{BarrE03} support updates to Java libraries in a
client server model. The system supports a subset of updates to Java
programs that do not cause linker errors.  Milazzo
\EA~\cite{Milazzo05updates} support dynamic updating in a distributed
computing environment. Their work proposes a specialized software
architecture to monitor updates, and distribute them across the
application.  Both systems use custom classloaders for binary-compatible
and component-level changes respectively, but cannot support signature
changes such as class field additions.

%% Eisenbach and Barr: safe upgrading without restarting.  They support
%% upgrades that satisfy binary compatibility.  Uses a custom classloader and
%% JMX to replace the code of existing objects.  No way to modify the state of
%% the objects.

%% Milazzo use a modified class loader to load individual replacements to
%% classes in a special-purpose architecture.  The class loader may modify the
%% bytecode of the loaded class to deal with type version namespace issues.
%% Basically this is more limited in scope than our approach.

Orso et al.~\cite{orso:java} use source-to-source translation for DSU by
introducing a proxy object that indirectly accesses an object that may
change. For each class C that might change in the future they produce a
proxy for that class.  All calls from clients of C are redirected to call
the wrapper instead.  When C is updated by some new class C', a new C'
object is created and initialized using the old state of C and the wrapper
is redirected to point to C'. This approach requires updated classes to
export the same public interface, forbidding new public methods and fields.

Non VM-based approaches are in general limiting because they are not
\emph{transparent}---they make visible changes to the class hierarchy, and
insert or rename classes. This approach makes it essentially impossible to
be robust in the face of code using reflection or native methods.
Moreover, the indirection imposes time and space overheads on steady-state
execution.  Our VM approach naturally supports reflection and native
methods (these are updated as well), is more expressive, e.g., it
supports signature changes, and imposes no overhead on steady-state
execution.

\subsection{VM support for DSU in managed languages} 

The PROSE system performs short-term, run-time patches to code for logging,
introspection, and performance adaptation, rather performing general
updates~\cite{nicoara:eurosys08}.  An Eclipse plug-in performs run-time
bytecode instrumentation and a modified JIT performs method code
replacement, using an API in the style of aspect-oriented programming.
PROSE has the same update model as the E\&C systems: it supports updates to
method bodies but not class or method signature changes that require
changes to object state.  This flexibility is similar to the E\&C
implementations discussed above; indeed, PROSE builds on the HotSwap method
replacement support in its Sun JDK implementation~\cite{JVMhotswap}.

JDrums~\cite{ritzau00dynamic} and the Dynamic Virtual Machine
(DVM)~\cite{Mala00a} both implement DSU for Java within the VM, providing a
programming interface similar to \JV, but are lacking in two ways.  First,
neither JDrums nor DVM have ever been demonstrated to support updates from
real-world applications.  Second, their implementations impose overheads
during steady-state execution.  They both update \emph{lazily} and use an
extra level of indirection (the \emph{handle space}).  Indirection
conveniently supports object updates, but adds extra overhead.  For
example, JDrums traps all object pointer dereferences to apply VM object
transformer function(s) when the object's class changes.  Lazy updating has
the advantage that it amortizes  pauses due to an update over subsequent
execution.  The main drawback is that its overhead persists during normal
execution, even though updates are relatively rare.  DVM works only with
the interpreter.  Relative to this interpreter, which is already slow, the
extra traps result in roughly 10\% overhead.

Compared to these two, \JV performs updates eagerly by employing a full
heap collection at update-time.  This stop-the-world approach imposes a
longer pause at update time, but eliminates overhead during steady-state
execution.  Likewise, by invalidating updated methods, \JV's performance is
slowed just after the update as these methods are being recompiled.
However, compared to running with an interpreter, steady-state execution is
much improved, since methods will be much better optimized.

\subsection{Dynamic ML}

Gilmore et al.~\cite{GilmoreKW97} propose DSU support for modules in ML
programs using a similar, but more restrictive programming interface
compared with \JV.  They formalize an abstract machine for implementing
updates using a copying garbage collector.
Duggan~\cite{DBLP:journals/acta/Duggan05} also proposes dynamic updates to
ML programs, focusing on lazy updates to data type definitions.  Neither
approach was ever implemented.

\subsection{Language support for \USD}

UpgradeJ~\cite{bierman08upgradej} is an extension to the Java language
design supporting class upgrades, in two flavors: \emph{revision upgrades},
which may modify method bodies, and \emph{evolution upgrades}, which may
add new methods and fields.  Programmers control the effects of upgrades
using \emph{version annotations}, introduced by Bierman et
al.~\cite{BiermanHSS03}.  For example, the programmer may
write \texttt{o = new Button[1=]()} to force \texttt{o} to always use
version~1 methods, while writing \texttt{p = new Button[1+]()} or \texttt{p
= new Button[1++]()} allows \texttt{p} to be revised or evolved,
respectively.  UpgradeJ's update model is easier to implement than \JV's
because it need not change existing object instances.  Of course, the
downside is a loss of flexibility.  Many of the updates to our benchmark
applications change field contents and layout.  UpgradeJ does not support
these updates.  On the other hand, evolution upgrades add power over simple
method body updates, and consequently enable more real-world updates to be
supported~\cite{tempero08upgradej}.  There currently is no implementation of
UpgradeJ.

%% JDRUMS: implements the conversion lazily.  They have a similar interface
%% (object and class transformers). The drawback here is that there is overhead
%% in the general case of execution---we do not know when the update is
%% complete.  Implemented in Sun's JDK 1.2.  Adds a level of indirection to the
%% new object.  Thus overhead builds up over time.  It also appears they have a
%% more limited interface to what can be referenced in a conversion function.
%% For example, there is no way to refer to fields other than those of the
%% object's class (i.e., no super-class fields) and there is no way to call new
%% methods, like constructors.  Not clear if there are restrictions on how
%% methods can be changed.

%% DVM: use an incremental mark-sweep collector, where mark phase marks objects
%% to be updated and the sweep phase incrementally updates them (prior to being
%% accessed by the mutator).  Like JDRUMS, all accesses to marked objects are
%% trapped.  Imposes a stock 10\% overhead, even only using bytecode.

%% Both of these: no significant experience with real applications, according
%% to how they change in practice.  They also can't handle native methods
%% because they can't trap access to modified objects.  Doing the full GC
%% solves this problem.

% More recently, Nicoara et al. developed PROSE, a system for run-time

% \section{Other proposals}
% 
% This part is not well written. Appreciate any suggestions.

\section{Updates in a persistent object store}
\label{sec:boyapati}

Boyapati et al.~\cite{boyapati03lazy} support dynamic updates to classes
kept in a \emph{persistent object store} (POS).  While the setting is
different, their basic update model, and in particular their notion of
object transformer function, is similar to ours.  In their system,
programmers manually write an object transformer that they view as a method
on the old version of the updated class, i.e., the transformer method is
type-safe with respect to the old class.  In \JV, object transformers may
access the \emph{new} versions of objects pointed to from the old class.
Instead, Boyapati et al.'s transformers may access the \emph{old} versions.
To implement this model, they rely on \emph{encapsulation} based on
\emph{ownership types}: if an object $a$ of class $A$ has an ``owned''
field pointing to an object $b$ of class $B$, then only $a$ can point to
(and access) $b$.  Encapsulation thus ensures the system will always
transform $a$ before $b$, which makes the transformation algorithm more
efficient.  They rely on the programmer to enforce encapsulation, and
describe how the compiler could automate language support for encapsulation
in a non-standard type system.  \JV takes the opposite tack of forcing old
object fields to point to up-to-date objects, and thus requires no special
language support.  Moreover, \JV's model follows that of earlier
work~\cite{k42usenix, neamtiu06dsu, neamtiu09stump, upstare} which has
proven its effectiveness on a half-dozen realistic applications across
several years' worth of releases.
% However, further research to understand
% the costs and benefits of the two updating models would be useful.

Boyapati et al. also differs from \JV in that, like JDRUMS and DVM, updates
are applied incrementally as objects are accessed following an update
rather than all at once using a stop-the-world GC\@. This incremental cost
is more natural in a POS since indirection is already required to access
external objects. The POS model also permits programmers to specify ACID
transaction boundaries, which can help ensure that updates are applied
consistently and safely.  In contrast, our work focuses on supporting
dynamic upgrades in a high-performance VM for Java, and thus many of the
issues we consider---reaching a safe point via return barriers and OSR, and
coexisting with the JIT compiler---are the unique contributions of our
work.


%         & \mc{6}{c}{Java} && \mc{7}{c}{C and C++} \\ \cmidrule(r){2-7} \cmidrule(l){8-10}
% Feature &
% \rotatebox{90}{Jvolve \cite{jvolve}} &
% \rotatebox{90}{Foo} &
% \BS IDEs \cite{JVMhotswap,Dmit01a, VSEnC, eaddy05enc}\ES &
% % Eisenbach \ET~\cite{BarrE03} and Milazzo et al.~\cite{Milazzo05updates}
% \BS DUSC \cite{orso:java} \ES &
% \BS PROSE \cite{nicoara:eurosys08} \ES &
% \BS JDrums~\cite{ritzau00dynamic} \ES &
% \BS DVM~\cite{Mala00a} \ES &
% \BS Boyapati ~\cite{boyapati03lazy} \ES &
% \BS Ksplice ~\cite{k42usenix} \ES &
% \BS Ginseng \cite{neamtiu06dsu} \ES &
% \BS Upstare \cite{upstare} \ES &
% \BS DynamicML ~\cite{GilmoreKW97} \ES &
% \BS Duggan~\cite{DBLP:journals/acta/Duggan05} \ES &
% \BS UpgradeJ~\cite{bierman08upgradej} \ES \\ \midrule

% Ksplice:
%   only changes to functions
%   no changes to types, or function signatures
%   insert trampolines at update time
%   activeness
%   Operating systems
% K42
%   State transformers
%   lazy or update time transformers
%   Updates at class granularity
%   Operating systems
%   Understand K42's quiesence mechanism
%   Activeness safety
% Upstare:
%   multi-threaded
%   C programs
%   stack reconstruction
% Ginseng:
%   state transformers
% 
% Jvolve:
% 
% DVM:
  
% vim:tw=0
% What are the various measures that distinguish one DSU system from
% another.
% 

\newcommand{\Y}{\ding{51}}
\newcommand{\N}{\ding{53}}
% \newcommand{\Y}{\rightthumbsup}
% \newcommand{\N}{\rightthumbsdown}
\newcommand{\Z}{\textbullet}

\begin{table}[p]
\centering \footnotesize
\newcommand{\BM}{\begin{minipage}{0.40\textwidth}}
\newcommand{\EM}{\end{minipage}}
\centering \footnotesize
\begin{tabular}{@{}b{0.40\textwidth}*{8}{c@{\hspace{2ex}}}@{}} \toprule

& \mc{4}{c}{C/C++} & & \mc{3}{c}{Java} \\ \cmidrule(lr){2-5} \cmidrule(lr){7-9}

&
\BS Ginseng \ES &                      % 1
\BS Upstare \ES &                      % 2
\BS Ksplice \ES &                      % 3
\BS K42 \ES &                          % 4
\BS Boyapati \EA \ES &                 % 5
\BS JDrums \ES &                       % 6
\BS DVM \ES &                          % 7
\BS Jvolve \ES \\ \midrule             % 8

%%%%%%%%%%%%%%%%%%%%%%%%%%%%%%%%%%%%%%%%%%%%%%%%%%%%%%%%%%%%%%%%%%%%%%%%%%%%%%%%%%%%%%%%%%%%%%%%%%%%%%%%%%%%%%%%%%%%
% Dsu features %                           1     2     3     4     5     6     7    8
{\bf Supported changes}                                                                                              \\
Method body                            &  \Y  & \Y  & \Y  & \Y  & \Y & \Y  & \Y & \Y                                 \\
Method types                           &  \Y  & \Y  & \N  & \Y  & \Y & \Y  & \Y & \Y                                 \\
Data/type signatures                   &  \Y  & \Y  & \N  & \Y  & \Y & \Y  & \Y & \Y                                 \\
Across class hierarchy                 &   -  &  -  &  -  & \N  &  - & \N  & \N & \Y                                 \\
Updates to active methods              &  \N  & \Y  & \N  & \N  & \N & \N  & \N & \N                                 \\ \midrule
{\bf Update timing }                                                                                                 \\
Old changed code runs after update     &  \Z  &     &     &     &    &     &    &                                    \\ 
Atomic update implementation           &      &     & \Z  &     &    &     &    & \Z                                 \\
Activeness safety                      &      &     & \Z  & \Z  & \Z & \Z  & \Z & \Z                                 \\ \midrule
% Needs quiescence                       &  \Z  &     &     & \Z  &    &     &    &                                    \\
% Supports multiple update points        &  \N  & \Y  & \Y  & \Y  &  ? &  ?  &  ? & \Y                                 \\ \midrule
{\bf Data changes}                                                                                                   \\
No access indirection                  &  \N  & \N  &  -  & \N  & \N &  -  &  - & \Y                                 \\
Does not use padding                   &  \N  & \N  &  -  & \Y  & \Y &  -  &  - & \Y                                 \\ \midrule 
% {\bf Code changes }                                                                                                  \\
% No function indirection                &  \N  & \N  & \Y  & \N  & \Y &  ?  &  ? & \Y                                 \\
% On-stack replacement                   &  \N  & \Y  & \N  & \N  &  ? & \N  & \N & \Y                                 \\
% Stack reconstructions                  &  \N  & \Y  & \N  & \N  & \N & \N  & \N & \N                                 \\ \midrule
{\bf State transformers }                                                                                            \\
Automatically generated                &  \Y  & \Y  &  -  & \Y  & \Y &     &    & \Y                                 \\ 
Refer to old/new state (O/N)           &   O  &  O  &  -  &  O  &  O &  -  &  - &  N                                 \\
Run Lazily/Eagerly (L/E)               &   L  &  L  &  -  &  L  &  L &  -  &  - &  E                                 \\ \midrule
{\bf Flexibility }                                                                                                    \\
Multi-threaded                         &  \Y  & \Y  & \Y  & \Y  & \Y & \N  &  \N& \Y                                 \\ \midrule
{\bf Performance }                                                                                                   \\
No steady-state time overhead          &  \N  & \N  & \Y  & \N  & \Y & \N  &  \N& \Y                                 \\
No steady-state space overhead         &  \N  & \N  & \Y  & \Y  & \N & \N  &  \N& \Y                                 \\ \bottomrule 
\end{tabular} \\[2ex]
\begin{tabular}{c@{\hspace{2ex}}lc@{\hspace{2ex}}l}
\mc{4}{c}{{\bf Legend}} \\
\Y & Advantage & \N & Disadvantage \\
\Z & Neutral   & -  & Not applicable
\end{tabular}
\caption{Comparison of \USD systems\label{tab:related}}
\end{table}

% vim:tw=0


\section{Summary}
Table~\ref{tab:related} summarises features, strengths, and limitations of four
DSU systems for C/C++ --- Ginseng, Upstare, Ksplice, and K42; updates in a
persistent object store; and three DSU systems for Java --- JDrums, DVM,
and Jvolve. The table denotes advantages of a system with a `\Y',
disadvantages with `\N', and neutral features with a `\Z'.
But for Upstare's generic stack reconstruction support, Jvolve
is superior to every other DSU system in one or more ways.

This recent surge in the design and development of dynamic updating system
shows a rising demand for highly available software and the limitations in
other systems reveal the difficulties inherent in combining performance,
efficiency, and flexibility as we did in \JV.
