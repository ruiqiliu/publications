% Updating data
%   state transformers
%   when to apply state transformers
%   how to apply transformers
%   supporting data structure changes
%   type wrapping
%   type representation/padding
%   indirection

\section{Updating data}
\label{sec:updating-data}

The most essential and challenging feature of any dynamic updating system is to change
application state --- stored in local and global variables, and in heap
allocated objects --- to conform to the semantics, type specification, and
concrete representation of the new version. Dynamic updating systems by
Hj\'{a}lmt\'{y}sson and Gray~\cite{HjalmtyssonG98} and
Duggan~\cite{DBLP:journals/acta/Duggan05} allow multiple versions of a type
to coexist, where code and data objects from the old and new program
versions interact freely with each other. In this section, we restrict our
discussion to a model where all data values in the application are
logically of the latest version of their corresponding type, a property
called {\em representation consistency}~\cite{mutatis}. A system that
maintains representation consistency transforms all objects to correspond to their new
type at update time, or transforms each object when the
application next accesses it.

In order to satisfy the semantics of the update, updating systems use
automatically-generated or programmer-written \emph{state transformers}
that return new program state from old state.  Depending on the dynamic
updating context, state transformers operate on stack
state, global variables, heap objects, or database tables. This section
discusses mechanisms that systems employ to support updates to application
data and the semantics of updating data using state transformers.

\subsection{Implementation mechanisms}
To update data, a system must address two questions. First, how does the
concrete representation of objects facilitate updating? Second, when are
object transformers invoked?

\paragraph{Concrete representation}
Systems such as Ginseng create a wrapper type for each updateable type in
the application. The system instruments the program so that all object
accesses go through the wrapper types. The wrapper type uses padding to
allow a new version to add fields to a type. The advantage of padding is
that it is straightforward to implement and integrates seamlessly with the
rest of the application. Objects declared as local variables and those
allocated dynamically on the heap are all update ready and are treated the
same. The disadvantages are that it wastes space and that a type cannot
grow larger than the initially allocated space.

An alternative approach is to use indirection where a field of the object
points to the additional fields in the updated types. Such an approach is
employed in the K42 operating system~\cite{K42reconfig}. Indirection allows
types to grow arbitrarily large in size, but adds a memory access to
dereference the indirection pointer for each access to fields of the new
version.

Another approach is to retain the same representation of objects as in a
system without dynamic updating. With this approach, the system allocates
new objects during update time and appropriately copies over contents from
old objects. However, such a system has to ensure that pointers to objects
are changed during the update to point to the newly allocated objects.  \JV
is the first to implement this functionality. \JV does so by extending the Virtual Machine's
garbage collector, as explained in detail in Section~\ref{sec:xformers}.

\paragraph{Transforming objects}
A design decision that dynamic updating systems make is when and how to
invoke state transformers on objects in the application. One approach is to
lazily transform objects after the update~\cite{neamtiu06dsu,
boyapati03lazy, ritzau00dynamic}. The system instruments every data access
to check whether the concerned object is of the latest type and invoke its
state transformer if the object is not up to date. The disadvantage of this
approach is that the system must always incur the
overhead of instrumentation and the addition of a field in every object to
keep track of the version number.
The advantage is that
lazily transforming objects
amortizes the cost of invoking state transformers by spreading it across
application execution.

The other alternative is to eagerly transform all objects during update
time which requires a way to access all objects in the application.
With this model,
the programmer must specify how to explicitly trace and
transform objects starting from global variables~\cite{hicks-thesis}, or
the system must maintain
a registry of all live objects in the
application~\cite{K42reconfig}. \JV which implements eager transformation
piggybacks on the garbage collector to trace and identify live objects that
need updating, and updates each such object based on programmer
specification.
