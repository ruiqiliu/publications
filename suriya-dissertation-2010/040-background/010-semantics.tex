\subsection{Semantics of state transformers}
% \USD transforms a running program from one version to a newer version on
% the fly.
% A key component of the update procedure is a {\em state transformer
% function}.
The state of a running process consists of values of local and
global variables, heap data, and one or more \PCs indicating the current
execution point of all running application threads. A state transformer
maps state from the old version of a program to state as expected
by the new version. The semantics of the update is as intimately tied to
the definition of the state transformer function, as it is to the
definition of the old and new program versions themselves.

For instance, a state transformer that initializes all variables to the
value ``unknown'' and the \PC to the start address of the new version, is
equivalent to stopping the old version of the program and restarting the
new one. Such a transformer would not be very useful and would defeat
the purpose of \USD. A useful and meaningful transformer has to come with
an understanding of semantics, both of the application and the update.
There might be updates where it is impossible to have a meaningful state
transformer.  Currently, \USD systems rely on the programmer to provide
such a state transformer function.

As an example,
consider a bugfix where a programmer used a 32-bit counter in
an older version and changed it to a 64-bit counter in a newer version,
presumably because the 32-bit counter was insufficient to represent
real-world values of the counter. The best any state transformer can hope
to do in this scenario is to copy the old counter value into the new
version and pad the higher order bits with zeros. If the counter had indeed
wrapped around in the old version, there would be no way for the
transformer to be aware of this fact and know what the higher order bits
should be. Leaving the higher order bits as zeros might or might not affect useful
execution of the updated version. What is acceptable
totally depends on the application and update semantics and the
expectations of the developers and users.

As another simple example, consider an application that stores points with
x and y co-ordinates in a 2-dimensional space. A feature of a newer version
might be that the application now supports a 3-dimensional space with
points having a z co-ordinate as well. In this case as well, a state
transformer function cannot hope to obtain an accurate representation of
the new version's state. Setting the z co-ordinates to zero of existing
points in the application might in fact work. It also seems intuitive that
setting z co-ordinates to values other than zero might cause the
application to work improperly. Such an inference has to come from the
developer with an understanding of the real-world semantics of the
application.

In this work, we assume that a correct and safe state transformer does
indeed exist, and ask what safety guarantees such as update correctness,
type safety, transaction safety, and representation consistency that a \USD
system can provide.
