\chapter{Background\label{chap:dsu}}

% This chapter will cover the following
% \begin{lstlisting}[numbers=none]
% * DSU Semantics [Written]
% * Safety [Written]
% * Update timing [Not yet written]
%    programmer specified update points
%    immediate updates
%    synchronization
%    queiscence points
%    multithreaded updates
%    loop extraction
% * Updating code [Written]
%    Function indirection
%    trampolines
%    on-stack replacement
%    stack transformers
% * Updating data [Half Written, to be polished]
%    state transformers
%    when to apply state transformers
%    how to apply transformers
%    supporting data structure changes
%    type wrapping
%    type representation/padding
%    indirection
% \end{lstlisting}

This chapter presents an overview of the dynamic updating problem.
It discusses the semantics of
updates, explains safety guarantees that \USD systems provide, and presents
mechanisms that systems employ to support \USD. This chapter focuses on DSU
system requirements and mechanisms, whereas Chapter~\ref{chap:related}
covers how prior systems chose among them.

The goal of \USD is to avoid application downtime in the face of software
updates. Researchers have addressed the problem of dynamic updating in
various contexts such as standalone and server applications, distributed
computing, distributed object stores, and databases for various types of
code updates, and types of code and data updates. This dissertation
considers updating a single application process, changing code and data to
be as expected by the new version.
% \USD systems achieve this goal by dynamically updating a running
% application, changing code and data to be as expected by the new version. 

% Updating code
%   Function indirection
%   trampolines
%   on-stack replacement
%   stack transformers

\section{Updating Code}

The most primitive functionality any dynamic updating system must support is
the ability to call new versions of updated methods.  Dynamic updating
systems in all types of contexts, be it in a compiled language like C, or
in a managed language runtime, or in a distributed computation framework,
resort to some form of indirection to call the new version of a
function.

% \paragraph{Function Indirection}
Systems for C/C++ such as Ginseng \cite{neamtiu06dsu}
and K42 \cite{K42reconfig} use indirection for each function call. Each
function call goes through a table that points to the latest version of the
function. At update time, the DSU system updates table entries to point
to new method versions. As a result, all future calls to a method invoke
its latest version. All systems using this approach pay an additional
overhead for all method calls during normal execution.

KSplice \cite{ksplice}, a dynamic updating system for the Linux kernel uses
trampolines to achieve indirection. In the absence of an update, kernel
execution happens normally. At update time, Ksplice overwrites the first few
instructions of an updated method with a call instruction to the new
version of a method. Future function calls to an updated method call the
old body, which transfers execution to the newest version. With this
approach, Ksplice pays almost zero execution time overhead.

Function indirection in managed languages usually comes for free. To make a
function call in an interpreted language, the interpreter gets the method
body by looking it up by name in a dictionary. Dynamic updating systems
that work in the context of an interpreter only have to update this
dictionary to point to the new method versions.

In a managed language Virtual Machine that compiles code down to machine
code, all non-inlined calls typically go through either a \VMT for virtual
calls, or a global table for calls to static methods. These tables point to
the latest compiled version of each method. When the VM compiles a method
at a higher level of optimization because the method is executed
frequently, it updates the table to point to the new version.
\JV extends this functionality for dynamic updating, by rewriting table
entries to point to the new method version. If the compiler has previously
inlined a changed method into an unchanged calling method, \JV also
rewrites the table entries of these calling methods containing an inlined
changed method. We are aware of no other system that handles inlining.

% \subsection{Updates to active methods}
% Supporting updates to active methods in their full generality makes it impossible
% to guarantee even non-semantic safety properties such as type-safety or
% correct execution that respects the language specification. However, actual
% changes to real applications drive DSU systems to support some changes to
% active methods. Systems lie on various points in the spectrum from no
% support at all to more general support for arbitrary code changes.
% 
% Systems that enforce activeness safety, do not support updates to active
% methods. As mentioned in Section~\ref{sec:safety}, these systems are
% type-safe but very restrictive. They do support a simple
% update that changes a version number string printed by the main method of
% the application, since the main method is always active. Systems that
% enforce con-freeness safety, for instance Ginseng~\cite{neamtiu06dsu}, support
% such updates by allowing existing methods to run the old version, while
% executing newer versions for future invocations.
% 
% When talking about changes to methods, it is important to mention the level
% of abstraction at which we compare method versions. In systems that compile
% source code to machine code, either statically in the case of C or
% dynamically in the case of Java, the source code may be the same across
% versions while the compiled machine codes is different, as explained in
% Section~\ref{sec:dsu-view-of-changes}. When DSU systems refer to changes,
% they usually do so at the compiled code level. \JV enforces activeness
% safety at the bytecode level, i.e., only methods with unchanged bytecodes
% can be active at update time. At the machine code level, \JV performs \OSR.
% \JV recompiles the method to generate machine code that conforms to the new
% version of the application and switches execution of an active method to
% this new code.
% 
% \paragraph{Stack extraction} In order to support arbitrary changes to
% active methods, a DSU system must extract an active method's stack state,
% transform it to satisfy the new version of the method, and transfer
% execution to the right instruction in the new method's body.
% Upstare~\cite{upstare}, a DSU system for C server applications, supports
% updates to active methods. Programmers write a stack transformer function
% that takes takes the old version's stack and returns a new one. Upstare's
% \emph{stack extraction} support takes the old version's stack, applies the
% programmer specified transformer and resumes execution after the update.
% This model relies heavily on developer expertise and testing.

% Updating data
%   state transformers
%   when to apply state transformers
%   how to apply transformers
%   supporting data structure changes
%   type wrapping
%   type representation/padding
%   indirection

\section{Updating data}
\label{sec:updating-data}

The most essential and challenging feature of any dynamic updating system is to change
application state --- stored in local and global variables, and in heap
allocated objects --- to conform to the semantics, type specification, and
concrete representation of the new version. Dynamic updating systems by
Hj\'{a}lmt\'{y}sson and Gray~\cite{HjalmtyssonG98} and
Duggan~\cite{DBLP:journals/acta/Duggan05} allow multiple versions of a type
to coexist, where code and data objects from the old and new program
versions interact freely with each other. In this section, we restrict our
discussion to a model where all data values in the application are
logically of the latest version of their corresponding type, a property
called {\em representation consistency}~\cite{mutatis}. A system that
maintains representation consistency transforms all objects to correspond to their new
type at update time, or transforms each object when the
application next accesses it.

In order to satisfy the semantics of the update, updating systems use
automatically-generated or programmer-written \emph{state transformers}
that return new program state from old state.  Depending on the dynamic
updating context, state transformers operate on stack
state, global variables, heap objects, or database tables. This section
discusses mechanisms that systems employ to support updates to application
data and the semantics of updating data using state transformers.

\subsection{Implementation mechanisms}
To update data, a system must address two questions. First, how does the
concrete representation of objects facilitate updating? Second, when are
object transformers invoked?

\paragraph{Concrete representation}
Systems such as Ginseng create a wrapper type for each updateable type in
the application. The system instruments the program so that all object
accesses go through the wrapper types. The wrapper type uses padding to
allow a new version to add fields to a type. The advantage of padding is
that it is straightforward to implement and integrates seamlessly with the
rest of the application. Objects declared as local variables and those
allocated dynamically on the heap are all update ready and are treated the
same. The disadvantages are that it wastes space and that a type cannot
grow larger than the initially allocated space.

An alternative approach is to use indirection where a field of the object
points to the additional fields in the updated types. Such an approach is
employed in the K42 operating system~\cite{K42reconfig}. Indirection allows
types to grow arbitrarily large in size, but adds a memory access to
dereference the indirection pointer for each access to fields of the new
version.

Another approach is to retain the same representation of objects as in a
system without dynamic updating. With this approach, the system allocates
new objects during update time and appropriately copies over contents from
old objects. However, such a system has to ensure that pointers to objects
are changed during the update to point to the newly allocated objects.  \JV
is the first to implement this functionality. \JV does so by extending the Virtual Machine's
garbage collector, as explained in detail in Section~\ref{sec:xformers}.

\paragraph{Transforming objects}
A design decision that dynamic updating systems make is when and how to
invoke state transformers on objects in the application. One approach is to
lazily transform objects after the update~\cite{neamtiu06dsu,
boyapati03lazy, ritzau00dynamic}. The system instruments every data access
to check whether the concerned object is of the latest type and invoke its
state transformer if the object is not up to date. The disadvantage of this
approach is that the system must always incur the
overhead of instrumentation and the addition of a field in every object to
keep track of the version number.
The advantage is that
lazily transforming objects
amortizes the cost of invoking state transformers by spreading it across
application execution.

The other alternative is to eagerly transform all objects during update
time which requires a way to access all objects in the application.
With this model,
the programmer must specify how to explicitly trace and
transform objects starting from global variables~\cite{hicks-thesis}, or
the system must maintain
a registry of all live objects in the
application~\cite{K42reconfig}. \JV which implements eager transformation
piggybacks on the garbage collector to trace and identify live objects that
need updating, and updates each such object based on programmer
specification.

  \subsection{Semantics of state transformers}
% \USD transforms a running program from one version to a newer version on
% the fly.
% A key component of the update procedure is a {\em state transformer
% function}.
The state of a running process consists of values of local and
global variables, heap data, and one or more \PCs indicating the current
execution point of all running application threads. A state transformer
maps state from the old version of a program to state as expected
by the new version. The semantics of the update is as intimately tied to
the definition of the state transformer function, as it is to the
definition of the old and new program versions themselves.

For instance, a state transformer that initializes all variables to the
value ``unknown'' and the \PC to the start address of the new version, is
equivalent to stopping the old version of the program and restarting the
new one. Such a transformer would not be very useful and would defeat
the purpose of \USD. A useful and meaningful transformer has to come with
an understanding of semantics, both of the application and the update.
There might be updates where it is impossible to have a meaningful state
transformer.  Currently, \USD systems rely on the programmer to provide
such a state transformer function.

As an example,
consider a bugfix where a programmer used a 32-bit counter in
an older version and changed it to a 64-bit counter in a newer version,
presumably because the 32-bit counter was insufficient to represent
real-world values of the counter. The best any state transformer can hope
to do in this scenario is to copy the old counter value into the new
version and pad the higher order bits with zeros. If the counter had indeed
wrapped around in the old version, there would be no way for the
transformer to be aware of this fact and know what the higher order bits
should be. Leaving the higher order bits as zeros might or might not affect useful
execution of the updated version. What is acceptable
totally depends on the application and update semantics and the
expectations of the developers and users.

As another simple example, consider an application that stores points with
x and y co-ordinates in a 2-dimensional space. A feature of a newer version
might be that the application now supports a 3-dimensional space with
points having a z co-ordinate as well. In this case as well, a state
transformer function cannot hope to obtain an accurate representation of
the new version's state. Setting the z co-ordinates to zero of existing
points in the application might in fact work. It also seems intuitive that
setting z co-ordinates to values other than zero might cause the
application to work improperly. Such an inference has to come from the
developer with an understanding of the real-world semantics of the
application.

In this work, we assume that a correct and safe state transformer does
indeed exist, and ask what safety guarantees such as update correctness,
type safety, transaction safety, and representation consistency that a \USD
system can provide.
 % This is a sub-section
\section{Safety of updates}
\label{sec:safety}

Supporting updates to code and data in a system should not compromise its
safety. By safety, we mean that we want to make guarantees that a \USD
system and the update are \emph{valid} and that the update and the
application would not perform \emph{illegal} operations that are usually
disallowed by normal execution semantics. For example, the update should
not lead the new version to crash because it accessed an invalid memory
location, or dereferenced a null pointer, or accessed an object of a
different type than it expected.

\paragraph{Update validity}

One guarantee a dynamic updating system may want to make is that the update is
\emph{valid}. Gupta \EA offer the following
definition of update validity~\cite{Gupta96,gupta-thesis}. A process or a running program $P$ is a pair
$(\Pi, s)$, where $\Pi$ refers to the program's code and $s$ to its state.
The state as mentioned above comprises locals, globals, heap data, and the
current \PC. An update to $P$ is a pair $(\Pi', S)$ where $\Pi'$ is the new
version's code and $S$ is the state transformer function. Applying the
update involves applying the state transformer function on the old state.
The \PC value of the old state is called the {\em update point} and the
resulting new state's \PC specifies the instruction at which to resume execution.
The updated process is $(\Pi', s')$ where $s' = S(s)$. An update is valid
if and only if the new program resuming execution at state $s'$ eventually
goes to a {\em reachable} state $s''$. We call $s''$ to be reachable if the
new program starting from its initial state on the same set of inputs at
some point reaches state $s''$. Gupta \EA showed that the problem of
deciding whether or not an update is valid for a state transformer in the
general case is undecidable.

Undecidability means that we cannot come up with a general purpose
algorithm that, given $\Pi$, $\Pi'$, $S$ and $s$, can say whether or not an
update is valid. However, they show that update validity can be verified
formally by restricting at which points an update takes place and what code
the state transformer can contain.  These restrictions are too
conservative. They only admit simple changes to applications.  While
it might be impossible to guarantee update validity, in general, we
consider the following safety properties.

\paragraph{Type safety}
% \label{sec:type-safety}

Type safety is a well-understood and highly desired property of real
programming languages. A type-safe system guarantees that any data element
accessed by code is of the right type expected by the code. A type-safe
\USD system guarantees type-safety of transformer code and new program
code that runs after the update.

\USD systems that support realistic changes provide a way to update user
defined types that change from the old to the new version. Each type $t$
that has changed representation from type $\tau$ in the old to type $\tau'$ in the
new version requires a {\em type transformer} function of type $\tau
\rightarrow \tau'$. To keep update semantics intuitive, \USD systems
enforce that at any given point of time, there is exactly one
representation of a type $t$ and that is the newest representation, a
property called {\em representation consistency}\index{representation
consistency}. To provide type safety, a \USD system guarantees that
no code is run, during or after the update, that expects a representation
of an earlier version.

\paragraph{Activeness safety}
A simple way to support representation consistency
and type safety is by not allowing any changed or deleted methods to be
active on stack at an update point.  This restriction is called {\em
activeness safety}~\cite{walton-thesis, ksplice, altekar05opus,
baumann07reboots}.
Activeness\index{activeness safety} can be
checked with a simple and accurate dynamic test that walks all application
stacks and looks for changed or deleted methods that are active at a
potential update point. It can also be enforced with a conservative static
analysis that examines the call graph of the old version. With either
approach, activeness safety guarantees type safety as follows. Consider the
set of all methods in the old and new version of the application. Some
methods exist in the old version but not in the new, either because these
methods are changed, or removed in the new version.
Conversely, some methods exist in the new but not in the old version.
Presumably, there are unchanged method bodies that are common to both the
old and the new version. In a type-safe language, the old and the new
program versions are independently type-safe. Activeness safety restricts
active methods at an update point to the intersection between the old and new
versions. At update time, type transformer functions convert all object
representations to conform to the new version. After the update, the
application can only execute the new version methods, which are type-safe by
definition.
\JV uses activeness safety because it is simple, guarantees type-safety,
requires only a list of changed methods, and is very efficient to check at
update time.

Restricting modified methods to be not active at update time can be too
constraining for multithreaded programs and for changes that affect methods
high in the call chain. These limitations
stand in the way of correctly dynamically updating more programs
\cite{hicks-thesis, armstrong}. A
system with activeness safety would never be able to update, for instance,
an application that prints its version number at the start of its main
method, because the main method would always be active. An
alternative is to allow old methods to run to completion after the update,
but invoke new version bodies for future method calls.

\begin{figure}[t]
\centering
\begin{minipage}{0.3\textwidth}%
\begin{lstlisting}[frame=single]
function foo() {
  ...
  access type t1;
  ...
  access type t2;
  ...
}
\end{lstlisting}
\end{minipage}%
\\
t1 is changed in the new version, t2 is not
\caption{Simple function illustrating con-freeness safety}
\label{fig:con-freeness}
\end{figure}


\paragraph{Con-freeness safety}

Stoyle \EA have defined a property called
con-free\-ness\index{con-freeness} of an update that ensures type safety and
have developed a static updatability analysis that answers whether or not
an update point in the program would violate con-freeness~\cite{mutatis}. An update point
$p$ is said to be con-free if code that comes after $p$ (which would run
after the update) does not concretely access any updated type. Consider the
simple function {\tt foo} shown in Figure~\ref{fig:con-freeness}. {\tt foo}
concretely accesses objects of two different types {\tt t1} and {\tt t2}
respectively. {\tt t1} is a type whose representation is changed in the new
version, while {\tt t2}'s representation remains the same. The update
process runs type transformers for all objects of type {\tt t1}. Let us
look at con-freeness at update points corresponding to line numbers 2, 4,
and 6. Line 2 is {\em not} con-free for the update because the function
will expect a type {\tt t1} object of the old representation, but encounter
a new version one. Line 4 is con-free for the update because {\tt t2}'s
representation is unchanged between the old and new versions. Line 6, of
course, is con-free as it is the end of the method with no unsafe access
possible. Stoyle \EA test for con-freeness with a flow-sensitive backwards
dataflow analysis. Con-freeness safety is less restrictive than activeness,
but recent work shows that exploiting some of these additional update
points can lead to incorrect updates in real applications~\cite{dsu-testing}.

\paragraph{Transaction safety}

\begin{figure}[t]
\centering
\begin{minipage}{0.3\textwidth}%
\begin{lstlisting}[frame=single]
transaction {
  ...
  foo();
  ...
  bar();
  ...
  baz();
  ...
}
\end{lstlisting}
\end{minipage}
\caption{Simple transaction region marked by the programmer}
\label{fig:transaction-safety}
\end{figure}


Transaction safety is a guarantee that a marked transaction fully obeys
either the semantics of the old version or that of the new version.
Consider the simple example in Figure~\ref{fig:transaction-safety}, where
the programmer has marked a region of code as a transaction. Let as assume
that code in the transaction is itself unchanged but methods {\tt foo},
{\tt bar} and {\tt baz} might have changed. Line~2 is always a safe update
point since the entire transaction will run the new code. Line~8 is also a
safe update point since the entire transaction would have run the old code.
If only one of the three called methods is updated, all program points in
the transaction are update safe, i.e., if the program point occurs before
the call to the changed method, the transaction will have the semantics of
the new version, whereas if the program point occurs after the call to the
changed method, the transaction will have the semantics of the old version.
Now, consider that both {\tt foo} and {\tt baz} are changed in the new
version. Line~4 and 6 are unsafe points because the transaction would run
the old version of {\tt foo}, but run the new version of {\tt bar},
violating transaction safety.

\subsection{Assuring safety by testing}
When program analyses fail to provide formal guarantees of correctness and
safety, software developers use testing to develop confidence that their
programs execute correctly.  The safety of Dynamic Software Updating should
be informally assured the same way. Hayden \EA~\cite{dsu-testing} have
devised a framework that exhaustively tests updating an application.  Their
starting point is a set of regression tests that is already used on a daily
basis to assure developers that the application runs correctly.  Testing
that the program can be safely updated at all possible program points for
each regression test is prohibitively expensive. However, they employ a novel dynamic
analysis that minimizes the space of update points by grouping them into
equivalence classes. Update points in an equivalence class all produce the
same execution behavior for a particular program trace.  Update points
that do not pass all regression tests should be marked as unsafe when
updating a production system. Results from their work show that activeness
safety and con-freeness safety very closely approximate update correctness,
but are not sufficient to guarantee correct program execution.

% * Update timing [Not yet written]
%    programmer specified update points
%    immediate updates
%    synchronization
%    quiescence points
%    multithreaded updates
%    loop extraction

\section{Update Timing}
The previous section dealt with safety properties at each program point for
a specific update.  In this section, we discuss update timing, or how a
dynamic updating system ensures that an application reaches a program point
that is safe for the update.

\paragraph{Update points in a program}
\USD systems provide API calls that check for and perform an update, and
typically allow either the programmer or the compiler to instrument program
update points. Note that these points are potential update points, and it is
not necessarily safe to perform an update at these points.  \JV uses method
entry points, method exit points, and loop backedges as potential update points.
These points are the same as {\em safe points} in a Virtual Machine. At
these safe points, the VM can safely switch threads and perform garbage
collection. Upstare, a \USD system
for C, has the same update points as \JV. In \JV, the compiler already
instruments these safe points for normal program execution, whereas in
Upstare, the \USD system has to instrument the program to make it
updateable.

The developers of Ginseng observed that programs that benefit most from
dynamic updating are typically structured as long-running event processing
loops. Each loop iteration is usually independent of each other and
processes a particular transaction. The start of each loop iteration serves
as a {\em quiescent point} where there are no partially-completed
transactions, and all global state is consistent. The common use case
scenario in Ginseng is for the developer to mark update points at the start
of outer loop iterations. Ginseng performs a static analysis to ensure that
it is safe to update at a marked program point, and updates the application
during runtime.

\paragraph{Return barriers}
In \JV, a user can trigger an update at any time during program execution.
While \JV will suspend the application at the earliest VM safe point it
encounters, this point is not necessarily safe for the update. For instance, a
modified method might be active on stack, violating activeness safety
discussed in Section~\ref{sec:safety}. In such situations, \JV
installs return barriers that trigger an update after all unsafe methods
have returned. Return barriers are most useful for long-running or
frequently-invoked unsafe methods, which are likely to be on stack almost
all the time. Return barriers are not sufficient if an unsafe method
contains an infinite loop, and would never return. There are at least two
good solutions which deal with infinite loops ---
stack reconstruction, used first in Upstare, discussed in
Section~\ref{subsec:updates-active-methods}, and
loop extraction, used
first in Ginseng.

\paragraph{Loop extraction}
In loop extraction, the programmer can mark potentially unsafe long running
loops, and the compiler will extract the loop out into its own function,
that is called on each loop iteration. If an update changes the loop body,
the extracted function will be unsafe for the update, but the update can
happen after it returns and later loop iterations call the new version's
function. Because of the code change, state used by the loop across
iterations might have to be updated as well. Ginseng automatically
generates state transformers for this loop state.

\subsection{Updates to active methods}
\label{subsec:updates-active-methods}
A dynamic updating system can improve flexibility by performing updates to
methods that are active on stack. Supporting updates to active methods in
their full generality makes it impossible to guarantee even non-semantic
safety properties such as type-safety or correct execution that respects
the language specification. However, actual changes to real applications
drive DSU systems to support some changes to active methods. Systems lie on
various points in the spectrum from no support at all to more general
support for arbitrary code changes.

Systems that enforce activeness safety, do not support updates to active
methods. As mentioned in Section~\ref{sec:safety}, these systems are
type-safe but very restrictive. They do support a simple
update that changes a version number string printed by the main method of
the application, since the main method is always active. Systems that
enforce con-freeness safety, for instance Ginseng~\cite{neamtiu06dsu}, support
such updates by allowing existing methods to run the old version, while
executing newer versions for future invocations.

When considering changes to methods, it is important to mention the level
of abstraction at which we compare method versions. In systems that compile
source code to machine code, either statically in the case of C or
dynamically in the case of Java, the source code may be the same across
versions, while the compiled machine codes is different, as explained in
more detail in
Section~\ref{sec:dsu-view-of-changes}. When DSU systems refer to changes,
they usually do so at the compiled code level. \JV enforces activeness
safety at the bytecode level, i.e., only methods with unchanged bytecodes
can be active at update time. At the machine code level, \JV performs
\acl{OSR}.
\JV recompiles the method to generate machine code that conforms to the new
version of the application and switches execution of an active method to
this new code.

\paragraph{Stack extraction} In order to support arbitrary changes to
active methods, a DSU system must extract an active method's stack state,
transform it to satisfy the new version of the method, and transfer
execution to the right instruction in the new method's body.
Upstare is the only dynamic updating system we are aware of, that supports
updates to active methods~\cite{upstare}. Programmers write a stack transformer function that takes takes
the old version's stack and returns a new one. Upstare's \emph{stack
extraction} support takes the old version's stack, applies the programmer
specified transformer and resumes execution after the update.  This model
relies heavily on developer expertise and testing.

% \paragraph{Stack reconstruction}
% Upstare is the only \USD system we are aware of, that supports changes to
% active methods on stack. When a long running method is changed between
% versions, Upstare maps program points between the old and the new method
% bodies, and requires the programmer to specify a stack transformer
% function. Such a function that takes the old function's stack state and
% returns the new functions' stack. As part of the update, Upstare extracts
% stack state, invokes the stack transformer to reconstruct the new version's
% stack, and resumes execution at the right point in the new method body.

\subsection{Multithreaded applications}
The above discussion on update points focussed mainly on single-threa\-d\-ed
applications. Other than stack reconstruction, which can update unsafe
methods in more than one thread, all other mechanisms fail for
multithreaded programs. Unsurprisingly, the challenge of synchronizing
multiple threads to all simultaneously be at a safe update point is hard. A
\USD system has to suspend a thread at a safe point, while waiting for
other threads to also reach respective safe points. This waiting can
adversely affects application throughput, and in the worst case may
deadlock the application. Neamtiu \EA address this problem in their work on
Stump~\cite{neamtiu09stump}. Stump allows the developer to specify a few
program points (in each thread) to be update safe. The system then uses
static analysis and runtime support to expand this list to other program
points with safe behavior. The runtime system synchronizes across threads,
resuming them if they have waited for too long.


\section{Conclusion}
In this chapter, we presented an overview of the dynamic software updating
problem.  We discussed mechanisms for updating code and data that meet
desirable safety guarantees while processing the update in a timely
fashion.
