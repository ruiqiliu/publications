\section{Safety of updates}
\label{sec:safety}

Supporting updates to code and data in a system should not compromise its
safety. By safety, we mean that we want to make guarantees that a \USD
system and the update are \emph{valid} and that the update and the
application would not perform \emph{illegal} operations that are usually
disallowed by normal execution semantics. For example, the update should
not lead the new version to crash because it accessed an invalid memory
location, or dereferenced a null pointer, or accessed an object of a
different type than it expected.

\paragraph{Update validity}

One guarantee a dynamic updating system may want to make is that the update is
\emph{valid}. Gupta \EA offer the following
definition of update validity~\cite{Gupta96,gupta-thesis}. A process or a running program $P$ is a pair
$(\Pi, s)$, where $\Pi$ refers to the program's code and $s$ to its state.
The state as mentioned above comprises locals, globals, heap data, and the
current \PC. An update to $P$ is a pair $(\Pi', S)$ where $\Pi'$ is the new
version's code and $S$ is the state transformer function. Applying the
update involves applying the state transformer function on the old state.
The \PC value of the old state is called the {\em update point} and the
resulting new state's \PC specifies the instruction at which to resume execution.
The updated process is $(\Pi', s')$ where $s' = S(s)$. An update is valid
if and only if the new program resuming execution at state $s'$ eventually
goes to a {\em reachable} state $s''$. We call $s''$ to be reachable if the
new program starting from its initial state on the same set of inputs at
some point reaches state $s''$. Gupta \EA showed that the problem of
deciding whether or not an update is valid for a state transformer in the
general case is undecidable.

Undecidability means that we cannot come up with a general purpose
algorithm that, given $\Pi$, $\Pi'$, $S$ and $s$, can say whether or not an
update is valid. However, they show that update validity can be verified
formally by restricting at which points an update takes place and what code
the state transformer can contain.  These restrictions are too
conservative. They only admit simple changes to applications.  While
it might be impossible to guarantee update validity, in general, we
consider the following safety properties.

\paragraph{Type safety}
% \label{sec:type-safety}

Type safety is a well-understood and highly desired property of real
programming languages. A type-safe system guarantees that any data element
accessed by code is of the right type expected by the code. A type-safe
\USD system guarantees type-safety of transformer code and new program
code that runs after the update.

\USD systems that support realistic changes provide a way to update user
defined types that change from the old to the new version. Each type $t$
that has changed representation from type $\tau$ in the old to type $\tau'$ in the
new version requires a {\em type transformer} function of type $\tau
\rightarrow \tau'$. To keep update semantics intuitive, \USD systems
enforce that at any given point of time, there is exactly one
representation of a type $t$ and that is the newest representation, a
property called {\em representation consistency}\index{representation
consistency}. To provide type safety, a \USD system guarantees that
no code is run, during or after the update, that expects a representation
of an earlier version.

\paragraph{Activeness safety}
A simple way to support representation consistency
and type safety is by not allowing any changed or deleted methods to be
active on stack at an update point.  This restriction is called {\em
activeness safety}~\cite{walton-thesis, ksplice, altekar05opus,
baumann07reboots}.
Activeness\index{activeness safety} can be
checked with a simple and accurate dynamic test that walks all application
stacks and looks for changed or deleted methods that are active at a
potential update point. It can also be enforced with a conservative static
analysis that examines the call graph of the old version. With either
approach, activeness safety guarantees type safety as follows. Consider the
set of all methods in the old and new version of the application. Some
methods exist in the old version but not in the new, either because these
methods are changed, or removed in the new version.
Conversely, some methods exist in the new but not in the old version.
Presumably, there are unchanged method bodies that are common to both the
old and the new version. In a type-safe language, the old and the new
program versions are independently type-safe. Activeness safety restricts
active methods at an update point to the intersection between the old and new
versions. At update time, type transformer functions convert all object
representations to conform to the new version. After the update, the
application can only execute the new version methods, which are type-safe by
definition.
\JV uses activeness safety because it is simple, guarantees type-safety,
requires only a list of changed methods, and is very efficient to check at
update time.

Restricting modified methods to be not active at update time can be too
constraining for multithreaded programs and for changes that affect methods
high in the call chain. These limitations
stand in the way of correctly dynamically updating more programs
\cite{hicks-thesis, armstrong}. A
system with activeness safety would never be able to update, for instance,
an application that prints its version number at the start of its main
method, because the main method would always be active. An
alternative is to allow old methods to run to completion after the update,
but invoke new version bodies for future method calls.

\begin{figure}[t]
\centering
\begin{minipage}{0.3\textwidth}%
\begin{lstlisting}[frame=single]
function foo() {
  ...
  access type t1;
  ...
  access type t2;
  ...
}
\end{lstlisting}
\end{minipage}%
\\
t1 is changed in the new version, t2 is not
\caption{Simple function illustrating con-freeness safety}
\label{fig:con-freeness}
\end{figure}


\paragraph{Con-freeness safety}

Stoyle \EA have defined a property called
con-free\-ness\index{con-freeness} of an update that ensures type safety and
have developed a static updatability analysis that answers whether or not
an update point in the program would violate con-freeness~\cite{mutatis}. An update point
$p$ is said to be con-free if code that comes after $p$ (which would run
after the update) does not concretely access any updated type. Consider the
simple function {\tt foo} shown in Figure~\ref{fig:con-freeness}. {\tt foo}
concretely accesses objects of two different types {\tt t1} and {\tt t2}
respectively. {\tt t1} is a type whose representation is changed in the new
version, while {\tt t2}'s representation remains the same. The update
process runs type transformers for all objects of type {\tt t1}. Let us
look at con-freeness at update points corresponding to line numbers 2, 4,
and 6. Line 2 is {\em not} con-free for the update because the function
will expect a type {\tt t1} object of the old representation, but encounter
a new version one. Line 4 is con-free for the update because {\tt t2}'s
representation is unchanged between the old and new versions. Line 6, of
course, is con-free as it is the end of the method with no unsafe access
possible. Stoyle \EA test for con-freeness with a flow-sensitive backwards
dataflow analysis. Con-freeness safety is less restrictive than activeness,
but recent work shows that exploiting some of these additional update
points can lead to incorrect updates in real applications~\cite{dsu-testing}.

\paragraph{Transaction safety}

\begin{figure}[t]
\centering
\begin{minipage}{0.3\textwidth}%
\begin{lstlisting}[frame=single]
transaction {
  ...
  foo();
  ...
  bar();
  ...
  baz();
  ...
}
\end{lstlisting}
\end{minipage}
\caption{Simple transaction region marked by the programmer}
\label{fig:transaction-safety}
\end{figure}


Transaction safety is a guarantee that a marked transaction fully obeys
either the semantics of the old version or that of the new version.
Consider the simple example in Figure~\ref{fig:transaction-safety}, where
the programmer has marked a region of code as a transaction. Let as assume
that code in the transaction is itself unchanged but methods {\tt foo},
{\tt bar} and {\tt baz} might have changed. Line~2 is always a safe update
point since the entire transaction will run the new code. Line~8 is also a
safe update point since the entire transaction would have run the old code.
If only one of the three called methods is updated, all program points in
the transaction are update safe, i.e., if the program point occurs before
the call to the changed method, the transaction will have the semantics of
the new version, whereas if the program point occurs after the call to the
changed method, the transaction will have the semantics of the old version.
Now, consider that both {\tt foo} and {\tt baz} are changed in the new
version. Line~4 and 6 are unsafe points because the transaction would run
the old version of {\tt foo}, but run the new version of {\tt bar},
violating transaction safety.

\subsection{Assuring safety by testing}
When program analyses fail to provide formal guarantees of correctness and
safety, software developers use testing to develop confidence that their
programs execute correctly.  The safety of Dynamic Software Updating should
be informally assured the same way. Hayden \EA~\cite{dsu-testing} have
devised a framework that exhaustively tests updating an application.  Their
starting point is a set of regression tests that is already used on a daily
basis to assure developers that the application runs correctly.  Testing
that the program can be safely updated at all possible program points for
each regression test is prohibitively expensive. However, they employ a novel dynamic
analysis that minimizes the space of update points by grouping them into
equivalence classes. Update points in an equivalence class all produce the
same execution behavior for a particular program trace.  Update points
that do not pass all regression tests should be marked as unsafe when
updating a production system. Results from their work show that activeness
safety and con-freeness safety very closely approximate update correctness,
but are not sufficient to guarantee correct program execution.
