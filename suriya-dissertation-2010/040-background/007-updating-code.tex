% Updating code
%   Function indirection
%   trampolines
%   on-stack replacement
%   stack transformers

\section{Updating Code}

The most primitive functionality any dynamic updating system must support is
the ability to call new versions of updated methods.  Dynamic updating
systems in all types of contexts, be it in a compiled language like C, or
in a managed language runtime, or in a distributed computation framework,
resort to some form of indirection to call the new version of a
function.

% \paragraph{Function Indirection}
Systems for C/C++ such as Ginseng \cite{neamtiu06dsu}
and K42 \cite{K42reconfig} use indirection for each function call. Each
function call goes through a table that points to the latest version of the
function. At update time, the DSU system updates table entries to point
to new method versions. As a result, all future calls to a method invoke
its latest version. All systems using this approach pay an additional
overhead for all method calls during normal execution.

KSplice \cite{ksplice}, a dynamic updating system for the Linux kernel uses
trampolines to achieve indirection. In the absence of an update, kernel
execution happens normally. At update time, Ksplice overwrites the first few
instructions of an updated method with a call instruction to the new
version of a method. Future function calls to an updated method call the
old body, which transfers execution to the newest version. With this
approach, Ksplice pays almost zero execution time overhead.

Function indirection in managed languages usually comes for free. To make a
function call in an interpreted language, the interpreter gets the method
body by looking it up by name in a dictionary. Dynamic updating systems
that work in the context of an interpreter only have to update this
dictionary to point to the new method versions.

In a managed language Virtual Machine that compiles code down to machine
code, all non-inlined calls typically go through either a \VMT for virtual
calls, or a global table for calls to static methods. These tables point to
the latest compiled version of each method. When the VM compiles a method
at a higher level of optimization because the method is executed
frequently, it updates the table to point to the new version.
\JV extends this functionality for dynamic updating, by rewriting table
entries to point to the new method version. If the compiler has previously
inlined a changed method into an unchanged calling method, \JV also
rewrites the table entries of these calling methods containing an inlined
changed method. We are aware of no other system that handles inlining.

% \subsection{Updates to active methods}
% Supporting updates to active methods in their full generality makes it impossible
% to guarantee even non-semantic safety properties such as type-safety or
% correct execution that respects the language specification. However, actual
% changes to real applications drive DSU systems to support some changes to
% active methods. Systems lie on various points in the spectrum from no
% support at all to more general support for arbitrary code changes.
% 
% Systems that enforce activeness safety, do not support updates to active
% methods. As mentioned in Section~\ref{sec:safety}, these systems are
% type-safe but very restrictive. They do support a simple
% update that changes a version number string printed by the main method of
% the application, since the main method is always active. Systems that
% enforce con-freeness safety, for instance Ginseng~\cite{neamtiu06dsu}, support
% such updates by allowing existing methods to run the old version, while
% executing newer versions for future invocations.
% 
% When talking about changes to methods, it is important to mention the level
% of abstraction at which we compare method versions. In systems that compile
% source code to machine code, either statically in the case of C or
% dynamically in the case of Java, the source code may be the same across
% versions while the compiled machine codes is different, as explained in
% Section~\ref{sec:dsu-view-of-changes}. When DSU systems refer to changes,
% they usually do so at the compiled code level. \JV enforces activeness
% safety at the bytecode level, i.e., only methods with unchanged bytecodes
% can be active at update time. At the machine code level, \JV performs \OSR.
% \JV recompiles the method to generate machine code that conforms to the new
% version of the application and switches execution of an active method to
% this new code.
% 
% \paragraph{Stack extraction} In order to support arbitrary changes to
% active methods, a DSU system must extract an active method's stack state,
% transform it to satisfy the new version of the method, and transfer
% execution to the right instruction in the new method's body.
% Upstare~\cite{upstare}, a DSU system for C server applications, supports
% updates to active methods. Programmers write a stack transformer function
% that takes takes the old version's stack and returns a new one. Upstare's
% \emph{stack extraction} support takes the old version's stack, applies the
% programmer specified transformer and resumes execution after the update.
% This model relies heavily on developer expertise and testing.
