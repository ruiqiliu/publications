% * Update timing [Not yet written]
%    programmer specified update points
%    immediate updates
%    synchronization
%    quiescence points
%    multithreaded updates
%    loop extraction

\section{Update Timing}
The previous section dealt with safety properties at each program point for
a specific update.  In this section, we discuss update timing, or how a
dynamic updating system ensures that an application reaches a program point
that is safe for the update.

\paragraph{Update points in a program}
\USD systems provide API calls that check for and perform an update, and
typically allow either the programmer or the compiler to instrument program
update points. Note that these points are potential update points, and it is
not necessarily safe to perform an update at these points.  \JV uses method
entry points, method exit points, and loop backedges as potential update points.
These points are the same as {\em safe points} in a Virtual Machine. At
these safe points, the VM can safely switch threads and perform garbage
collection. Upstare, a \USD system
for C, has the same update points as \JV. In \JV, the compiler already
instruments these safe points for normal program execution, whereas in
Upstare, the \USD system has to instrument the program to make it
updateable.

The developers of Ginseng observed that programs that benefit most from
dynamic updating are typically structured as long-running event processing
loops. Each loop iteration is usually independent of each other and
processes a particular transaction. The start of each loop iteration serves
as a {\em quiescent point} where there are no partially-completed
transactions, and all global state is consistent. The common use case
scenario in Ginseng is for the developer to mark update points at the start
of outer loop iterations. Ginseng performs a static analysis to ensure that
it is safe to update at a marked program point, and updates the application
during runtime.

\paragraph{Return barriers}
In \JV, a user can trigger an update at any time during program execution.
While \JV will suspend the application at the earliest VM safe point it
encounters, this point is not necessarily safe for the update. For instance, a
modified method might be active on stack, violating activeness safety
discussed in Section~\ref{sec:safety}. In such situations, \JV
installs return barriers that trigger an update after all unsafe methods
have returned. Return barriers are most useful for long-running or
frequently-invoked unsafe methods, which are likely to be on stack almost
all the time. Return barriers are not sufficient if an unsafe method
contains an infinite loop, and would never return. There are at least two
good solutions which deal with infinite loops ---
stack reconstruction, used first in Upstare, discussed in
Section~\ref{subsec:updates-active-methods}, and
loop extraction, used
first in Ginseng.

\paragraph{Loop extraction}
In loop extraction, the programmer can mark potentially unsafe long running
loops, and the compiler will extract the loop out into its own function,
that is called on each loop iteration. If an update changes the loop body,
the extracted function will be unsafe for the update, but the update can
happen after it returns and later loop iterations call the new version's
function. Because of the code change, state used by the loop across
iterations might have to be updated as well. Ginseng automatically
generates state transformers for this loop state.

\subsection{Updates to active methods}
\label{subsec:updates-active-methods}
A dynamic updating system can improve flexibility by performing updates to
methods that are active on stack. Supporting updates to active methods in
their full generality makes it impossible to guarantee even non-semantic
safety properties such as type-safety or correct execution that respects
the language specification. However, actual changes to real applications
drive DSU systems to support some changes to active methods. Systems lie on
various points in the spectrum from no support at all to more general
support for arbitrary code changes.

Systems that enforce activeness safety, do not support updates to active
methods. As mentioned in Section~\ref{sec:safety}, these systems are
type-safe but very restrictive. They do support a simple
update that changes a version number string printed by the main method of
the application, since the main method is always active. Systems that
enforce con-freeness safety, for instance Ginseng~\cite{neamtiu06dsu}, support
such updates by allowing existing methods to run the old version, while
executing newer versions for future invocations.

When considering changes to methods, it is important to mention the level
of abstraction at which we compare method versions. In systems that compile
source code to machine code, either statically in the case of C or
dynamically in the case of Java, the source code may be the same across
versions, while the compiled machine codes is different, as explained in
more detail in
Section~\ref{sec:dsu-view-of-changes}. When DSU systems refer to changes,
they usually do so at the compiled code level. \JV enforces activeness
safety at the bytecode level, i.e., only methods with unchanged bytecodes
can be active at update time. At the machine code level, \JV performs
\acl{OSR}.
\JV recompiles the method to generate machine code that conforms to the new
version of the application and switches execution of an active method to
this new code.

\paragraph{Stack extraction} In order to support arbitrary changes to
active methods, a DSU system must extract an active method's stack state,
transform it to satisfy the new version of the method, and transfer
execution to the right instruction in the new method's body.
Upstare is the only dynamic updating system we are aware of, that supports
updates to active methods~\cite{upstare}. Programmers write a stack transformer function that takes takes
the old version's stack and returns a new one. Upstare's \emph{stack
extraction} support takes the old version's stack, applies the programmer
specified transformer and resumes execution after the update.  This model
relies heavily on developer expertise and testing.

% \paragraph{Stack reconstruction}
% Upstare is the only \USD system we are aware of, that supports changes to
% active methods on stack. When a long running method is changed between
% versions, Upstare maps program points between the old and the new method
% bodies, and requires the programmer to specify a stack transformer
% function. Such a function that takes the old function's stack state and
% returns the new functions' stack. As part of the update, Upstare extracts
% stack state, invokes the stack transformer to reconstruct the new version's
% stack, and resumes execution at the right point in the new method body.

\subsection{Multithreaded applications}
The above discussion on update points focussed mainly on single-threa\-d\-ed
applications. Other than stack reconstruction, which can update unsafe
methods in more than one thread, all other mechanisms fail for
multithreaded programs. Unsurprisingly, the challenge of synchronizing
multiple threads to all simultaneously be at a safe update point is hard. A
\USD system has to suspend a thread at a safe point, while waiting for
other threads to also reach respective safe points. This waiting can
adversely affects application throughput, and in the worst case may
deadlock the application. Neamtiu \EA address this problem in their work on
Stump~\cite{neamtiu09stump}. Stump allows the developer to specify a few
program points (in each thread) to be update safe. The system then uses
static analysis and runtime support to expand this list to other program
points with safe behavior. The runtime system synchronizes across threads,
resuming them if they have waited for too long.
