Because software systems are imperfect, developers are forced to fix bugs
and add new features. The common way of applying changes to a running
system is to stop the application or machine and restart with the new
version. Stopping and restarting causes a disruption in service that is at
best inconvenient and at worst causes revenue loss and compromises safety.
Dynamic software updating (DSU) addresses these problems by updating
programs while they execute. Prior DSU systems for managed languages like
Java and C\# lack necessary functionality: they are inefficient and do not
support updates that occur commonly in practice.

This dissertation presents the design and implementation of \JV, a DSU
system for Java. \JV's combination of \emph{flexibility}, \emph{safety},
and \emph{efficiency} is a significant advance over prior approaches.  Our
key contribution is the extension and integration of existing Virtual
Machine services with safe, flexible, and efficient dynamic updating
functionality.  Our approach is flexible enough to support a large class of
updates, guarantees type-safety, and imposes no space or time overheads on
steady-state execution.

\JV supports many common updates. Users can add, delete, and change
existing classes.  Changes may add or remove fields and methods, replace
existing ones, and change type signatures.  Changes may occur at any level
of the class hierarchy.  To initialize new fields and update existing ones,
\JV applies \emph{class} and \emph{object transformer} functions, the
former for static fields and the latter for object instance fields.  These
features cover many updates seen in practice.  \JV supports 20
of 22 updates to three open-source programs---Jetty web server,
JavaEmailServer, and CrossFTP server---based on actual releases occurring
over a one to two year period. This support is substantially more flexible
than prior systems.

\JV is safe. It relies on bytecode verification to statically type-check
updated classes.  To avoid dynamic type errors due to the timing of an
update, \JV stops the executing threads at a \emph{DSU safe point} and then
applies the update. DSU safe points are a subset of VM \emph{safe points},
where it is safe to perform garbage collection and thread scheduling.  DSU
safe points further restrict the methods that may be on each thread's
stack, depending on the update.  Restricted methods include updated methods
for code consistency and safety, and user-specified methods for semantic
safety. \JV installs return barriers and uses on-stack replacement to speed
up reaching a safe point when necessary.  While \JV does not guarantee that
it will reach a DSU safe point, in our multithreaded benchmarks it almost
always does.

% \todo{Find out a way to nicely include this sentence, or drop it totally:
% Much prior work does not support updating multithreaded programs.}

\JV includes a tool that automatically generates default object
transformers which initialize new and changed fields to default values and
retain values of unchanged fields in heap objects.  If needed, programmers
may customize the default transformers. \JV is the first dynamic updating
system to extend the garbage collector to identify and transform all object
instances of updated types.  This dissertation introduces the concept of
object-specific state transformers to repair application heap state for
certain classes of bugs that corrupt part of the heap, and a novel
methodology that employes dynamic analysis to automatically generate these
transformers. \JV's eager object transformation design and implementation
supports the widest class of updates to date.

Finally, \JV is efficient. It imposes no overhead during steady-state
execution.  During an update, it imposes overheads to classloading and
garbage collection.  After an update, the adaptive compilation system will
incrementally optimize the updated code in its usual fashion. \JV is the
first full-featured dynamic updating system that imposes no steady-state
overhead.

In summary, \JV is the most-featured, most flexible, safest, and
best-performing dynamic updating system for Java and marks a significant
step towards practical support for dynamic updates in managed language
virtual machines.
