
\chapter{Introduction}

Evolving software is a fact of life. Developers constantly fix bugs and add
new features to software. These software changes
make their way to deployed systems. However, many systems ranging
from medical equipment to communication and transportation systems to
financial systems, must be always available. For online stores, brokerages,
and exchanges an hour of downtime can mean significant revenue loss running
into thousands or even millions of dollars~\cite{gartner98downtime, roc1,
eagle-rock, parker}. Mission critical applications such as satellites
cannot tolerate downtime~\cite{french-satellite}. When it comes to
end-user applications and operating systems, downtime is both expensive
and inconvenient~\cite{zorn05}. In spite of these concerns, the most common
update method used today is to stop a running system, install an update, and
restart with the new version. In one study of nearly 6,000 outages of
high-availability applications, 75\% were for planned hardware and software
maintenance \cite{lowell:04}. Another study of the IT industry finds 80\%
of all downtime to have been planned, and 15\% of downtime is attributed to
software updates \cite{vs-downtime}. Moreover, security issues force system
administrators to patch their systems ever more frequently \cite{vista}.
With Internet access being ubiquitous, and application and OS
vulnerabilities in unpatched systems being exploited more and more,
delaying updates poses real security risks \cite{altekar05opus, ksplice,
survival-time}.

As a solution to these problems, researchers have proposed \acf{DSU}.  \USD is a
general-purpose approach to updating running programs. \USD dynamically patches a program running an old version of an
application, updating running code and data to a state consistent with a
newer version.  \USD's appeal is that it can be applied generally without
the need for redundant hardware or explicit special-purpose system designs~\cite{kspliceslashdot08}. An ideal general-purpose \USD system
should be \emph{flexible}, \emph{safe}, and \emph{efficient}
\cite{hicks-thesis, neamtiu-thesis}.
%
\begin{description}
%
\item[Flexible.] A \USD system should be \emph{flexible} enough to handle the type of
changes developers typically make between versions. While it is not
possible to handle every possible software update, \USD desires to support most
software version changes encountered in practice.
%
\item[Safe.] Dynamically updating to the new version should be as {\em safe}
as starting the new version from scratch. The system should guarantee to
developers and users that the update process does not introduce type or
process state errors, and that the updated version executes as intended.
%
\item[Efficient.] \USD should provide an \emph{efficient} upgrade process and efficient
normal execution.  \USD support should add no performance overhead over
normal application execution, and the update process itself should be
quick. Preparing the patch and testing the update itself requires
additional development effort, but this effort should be minimal.
%
\end{description}

Researchers have made significant strides toward making \USD practical for
systems written in C or C++, supporting server feature
upgrades~\cite{neamtiu06dsu, chen:icse07, upstare}, security
patches~\cite{altekar05opus}, and operating systems
upgrades~\cite{K42reconfig, k42usenix, baumann07reboots, baumann04workshop,
dynamos_eurosys_07, chen06vee, ksplice}.  Unfortunately, work on \USD for
managed languages such as Java, C\#, Python and Ruby lags behind work for C
and C++.  However, the use of Java and other dynamic languages is increasingly
common, powering systems both small and large~\cite{gartner-java,
ibm-nyse-java, cell-phones, pizlo09, sdtimes}.
Websites and web-services are increasingly written in dynamic languages such as Python
and Ruby. Java powers both large complex systems running stock exchanges
and millions of small cell phones and other handheld devices.  Even
safety-critical
applications such as those that control aircraft originally written in
Ada, are migrating to Java. Dynamic updating
support for managed languages has lagged behind pervasive use of these
languages. For example, while the HotSpot JVM~\cite{JVMhotswap} and some
.NET languages~\cite{VSEnC} support on-the-fly method body updates, this
support is too inflexible for all but the simplest updates \cite{jvolve}.
Academic approaches~\cite{ritzau00dynamic, Mala00a, orso:java,
bierman08upgradej} offer more flexibility, but have never been demonstrated on
realistic applications, and furthermore, these prior designs impose
substantial space and time overheads on steady-state execution.

In this dissertation, we present \JV, a \JVM that provides general-purpose
dynamic updating of Java applications. \JV's combination of flexibility,
safety, and efficiency is a major advance over prior approaches. Our key
contribution is to show how to extend and integrate existing \VM services
to support dynamic updating that is flexible enough to support the largest class of
updates to date, guarantees type-safety, and imposes no space or time overheads on
steady-state execution.

\paragraph{Flexibility.}
A practical \USD system must support changes to software that developers
typically make between versions. Prior studies on C applications
\cite{neamtiu05evolution}, the Linux kernel \cite{linux-evolution} and Java
applications \cite{java-evolution} show that changes can usually be
categorized into addition of new methods and types, modification of method
definitions, and changes to the type signatures of methods and data fields.
%
\JV supports these and other common updates. Users can add, delete, and change
existing classes.  Changes may add or remove fields and methods, replace
existing ones, and change type signatures.  Changes may occur at any level
of the class hierarchy. Changes can alter data structures, e.g., replace a
list with a hash map.
As a testament to its flexibility, our experience shows that \JV supports
20 of 22
updates to three open-source programs---Jetty web server, JavaEmailServer,
and CrossFTP server---based on actual releases occurring over 1 to 2 years.

% To initialize new fields, update existing ones, and alter data structures
% \JV applies \emph{class} and \emph{object transformer} functions, the
% former for static fields and the latter for object instance fields. The
% system automatically generates default transformers, which initialize new
% and changed fields to default values and retain values of unchanged fields.
% If needed, programmers can customize the default transformers.

% Object transformers considered by dynamic updating systems in the past have
% primarily converted state from the old version to the new
% version. Such per-type transformers uniformly apply the same logic to all
% objects of a type. In this dissertation, we explore state transformers that
% customize actions based on the object being
% considered. We use such transformers to fix faulty application state caused
% by incorrect execution. A subset of objects in the heap might be
% corrupted because of a bug. In this model, a transformer
% examines an object's state and repairs it if necessary.
% We also explore automatically generating such transformers to ease the
% burden on the developer.  We
% perform a dynamic analysis that, informed by the program patch, runs on the
% old version of the program to
% infer heap invariants that distinguish between objects that need repair and
% those that do not, and automatically generates state transformers based on the
% update patch and the discovered invariants. We are the first to consider
% systematic, accurate repair of buggy application state.

\paragraph{Overview of \JV's approach.}
\JV consists of 1) an offline source analysis tool called the \UPT that
uses
jClasslib Java Bytecode Viewer library, and 2) a \JVM with
\USD support built on top of \RVM, a research \VM. \UPT identifies differences
between source versions and generates an update specification that includes
\emph{class} and \emph{object transformer} functions that specify logic to
update application data from the old to the new version.
The \JV \VM takes in this specification and performs the update.  Upon being
notified that an update is available, \JV pauses the application, performs
the update atomically in a single step, and resumes application execution
of the updated version. Only old code runs before the update and only new code
runs after the update. Application data stored in globals, locals, and the
heap always corresponds to the newest version of the application, a property
called {\em representation consistency}\index{representation consistency}.
To maintain representation consistency \JV suspends the running program at
a \DSP and converts all object instances to conform to the newest version
before it restarts the application.

\paragraph{Update Safety.} \JV relies on Java's semantics to ensure that the
old and new application versions are independently consistent and {\em
type-safe}. \JV ensures a type-safe update process by allowing only
unchanged methods to be active on stack when performing the update
\cite{mutatis, neamtiu06dsu, k42usenix}. \JV relies on
the developer and testing process to specify other methods that should not
be executing during an update in
order to maintain correct application semantics with the new version
\cite{gupta-thesis, neamtiu08context}. For example, consider a method {\tt
foo} that makes two consecutive methods invocations to {\tt bar} and {\tt
baz}, and requires for correctness that both calls are of the same version.
If both methods are updated, the developer should not allow an update when
{\tt foo} is active, to prevent an update in between calls to {\tt bar} and
{\tt baz}.

To process an update, \JV pauses
execution at a \DSP.  \USD safe points are a subset of \VM{} {\em safe
points} where it is safe to switch application threads and perform \GC.
\USD safe points have the aforementioned restrictions on what
methods can be active on stack to guarantee a safe update.
%
In situations where restricted methods are active on stack, \JV installs
return barriers~\cite{return-barrier} on these methods that inform the \VM
that an unsafe method has returned and that it may now be safe to perform an update.
\JV's use of return barriers increases the
likelihood of an application reaching a \DSP, but does not guarantee
reaching one in multithreaded programs~\cite{neamtiu09stump}.
% More sophisticated thread
% scheduling support would be need to attain greater
% flexibility~\cite{neamtiu09stump}.
\JV utilizes \OSR to recompile restricted methods with unchanged method
bodies that are active on stack. In this case, \JV promptly performs the
update. In our experience, if a safe point can ever be reached, \JV's
support is sufficient to reach it.

\paragraph{Updating code.}
\JV makes use of \JIT compilation to efficiently update the code.
\JV invalidates existing compiled code and installs new
bytecode for all changed method implementations.  When the application
resumes execution these methods are \JIT-compiled when they are next
invoked.  The adaptive compilation system naturally optimizes updated
methods further if they execute frequently.

\paragraph{Updating data.} To update live object instances of changed
  classes, \JV makes use of \acf{GC}, and is the first dynamic updating system to
  do so.  \JV initiates a whole-heap \GC, which identifies existing object instances
  of updated classes. As the last phase of \GC, \JV initializes and
  transforms all updated objects to their new versions using default or
  user provided object transformers for each updated type.
Because \JV identifies all objects it will update before actually
  invoking object transformers, the system can control the order in which
  objects are transformed. To our knowledge, \JV is the only system with
  this functionality. In the current implementation, \JV relies on the
  developer to specify the required order.

In \JV's natural and intuitive object transformation model, the
  transformer function receives two objects: an object of the old type
  corresponding to the state before the update, and an object of the new type
  corresponding to the state after the update.
For global variables,
\JV invokes class transformers where the
  programmer has access to any object reachable from global variables, i.e.,
  static fields of classes.  The \UPT generates default object and class
  transformers that are simple, and the developer can write a more
  sophisticated function if required.

\paragraph{Specializing and automating transformers.}
In this
  dissertation, we also explore state transformers with logic that depends
  on the state of the object being transformed, rather than only performing a
  uniform action for all objects. We use such transformers to repair
  erroneous application state that results from executing a buggy program
  version. As part of the update, the dynamic updating system runs state transformers
  on a subset of objects with corrupted state. We use these transformers to
  fix memory leaks in real applications. In this dissertation we also
  present a methodology that starts with a bugfix and instruments the
  application to mark objects that would be corrupted in a run of the old
  version of the program. We then perform a novel dynamic analysis that
  infers a predicate that distinguishes between marked and unmarked objects
  in the application and automatically generates object transformers that
  repair the state of objects identified by the predicate. We are the first
  to consider systematic, accurate repair of buggy application state during
  dynamic updating.

\paragraph{Efficiency.}
\JV imposes no overhead on steady-state execution. During an update,
\JV employs
classloading and garbage collection. After an
update, the adaptive compilation system will incrementally optimize the
updated code in its usual fashion.  Eventually, the code is fully optimized
and running with no additional overhead.  The zero overhead in steady-state
execution for a VM-based approach is in contrast to \USD techniques for C,
C++, and previous proposals for managed languages. These prior approaches use a compiler or dynamic rewriter to insert
levels of indirection~\cite{neamtiu06dsu, orso:java} or
trampolines~\cite{chen06vee,chen:icse07,altekar05opus,ksplice}, which add
overhead during normal execution.

We assessed \JV by applying it to updates corresponding to one to two years'
worth of releases for three open-source multithreaded applications: Jetty
web server, JavaEmailServer (an SMTP and POP server), and CrossFTP server.
\JV successfully applies 20 of the 22 updates---the two
updates it does not support change a method within an infinite loop that is
always on the stack.  Microbenchmark results show that the pause time due
to an update depends on the size of the heap and fraction of transformed
objects.  Experiments with Jetty show that applications updated by \JV
execute correctly and
enjoy the same steady-state performance as if started from scratch.

In summary, the main contributions of this dissertation are:
\begin{enumerate}
\item New techniques that extend and integrate standard virtual machine services
for managed languages to support a flexible, safe, and efficient dynamic
software updating service.
\item The first implementation to employ garbage collection to update objects in the heap and a
flexible model that allows object transformers to enforce execution order.
\item The first systematic methodology for developing state transformers to
repair application state, employing a novel
dynamic analysis to automatically generate the transformers.
\item The design, implementation, and evaluation, using real-world
applications, of \JV, a Java \VM with
fully-featured support for dynamic software updating, that is distinguished
from prior work in its realism, flexibility, technical novelty, and high
performance.  
\end{enumerate}
This work demonstrates a significant step towards supporting flexible,
efficient, and safe updates in managed code virtual machines.  We believe
that our design, implementation, and results show that this technology,
together with a rigorous testing regime, is ready to be adopted and be a
part of future virtual machines.
% To take
% dynamic updating support to the real world, researchers should build
% dynamic updating systems that work in various environments --- from desktop
% applications, to operating systems, to web applications; address the
% challenges of safety guarantees and build update safety analysis into
% Integrated Development Environments; develop efficient dynamic updating
% testing mechanisms that can be included in the software release process;
% and integrate dynamic updating into the edit-compile-debug cycle.

% Visa transaction processing system is routinely updated as many as 20,000
% times per year, yet tolerates less than 0.5% down-time \cite{visa}
