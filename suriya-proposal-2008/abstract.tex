% $Id: abstract.tex 8063 2008-05-09 01:15:48Z mwh657 $

\begin{abstract}
  \noindent
  Software evolves to fix bugs and add features, but stopping and
  restarting existing programs to take advantage of these changes can
  be inconvenient and costly.  Dynamic software updating (DSU)
  addresses these problems by updating programs while they run.
  The challenge is to develop 
  DSU infrastructure that is \emph{flexible}, \emph{safe}, and
  \emph{efficient}---DSU should enable updates that are likely to
  occur in practice, and updated programs should be as reliable and as
  efficient as those started from scratch.
  
  This proposal presents the design and implementation of a JVM we call
  \DSU{} that is enhanced
  with DSU support.  The key insight is that
  flexible, safe, and efficient DSU can be supported by naturally
  extending existing VM services.  By
  piggybacking on classloading and garbage collection, \DSU{} can
  flexibly support additions and replacements of fields and methods anywhere
  within the class hierarchy, and in a manner that may alter class
  signatures.  By utilizing bytecode verification and
  thread synchronization support, \DSU{} can ensure that an applied
  update will never violate type-safety. By building on top of \acf{OSR}
  support, \DSU{} can support updates to active methods on stack.
  Finally, by employing \acs{JIT}
  compilation, there is no DSU-related 
  overhead before or after an update.
  Our work demonstrates that the VM is well-suited to support practical DSU
  services and DSU support should be a standard feature of VMs in the
  future.
%   Using \DSU{}, we successfully applied dynamic continuous updates
%   corresponding to 20 
%   of the 22 releases that occurred over nearly two years' time, one
%   update per release, for three open-source programs, Jetty web
%   server,  JavaEmailServer, and CrossFTP server.  Our results indicate
%   that the VM is well-suited to support practical DSU services.
\end{abstract} 
