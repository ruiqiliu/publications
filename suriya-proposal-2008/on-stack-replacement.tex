
\paragraph{\acf{OSR}.}
\label{paragraph:osr}

By utilizing \acs{OSR}, we can push the envelope of active methods on stack
that \DSU{} allows during an update. \JikesRVM{}'s adaptive compiler
employs \acs{OSR} \cite{Fink:CGO:03} to transition from a long running
base-compiled version of a method to an optimized version. OSR
transformation involves the following three steps. 1) The VM extracts
compiler-independent state from the activation record of a suspended
thread. 2) The VM then generates specialized code for the suspended
activation. This code consists of a \emph{specialized prologue} that saves
values into locals and loads these values on the stack. 3) This specialized
code also transfers execution in the suspended thread to the optimized
version of the method at the current program counter.

\DSU{} can piggyback on OSR support to allow \emph{indirectly updated
methods} to be active on stack. The source code for these methods is
unchanged, but these methods must be recompiled because they refer to
offsets of fields or methods in updated classes. OSR needs no additional
information to recompile the method and jump to its new body.

To allow methods whose bodies are updated to be active on stack during the
update, \DSU{} must be able the compare their implementations to find a
correspondence between the old and updated versions, and identify the
method location to resume execution at after the update. This problem is in
general undecidable. We propose to implement an approach that we believe
will work in many cases. We propose to have \acs{UPT}, the offline tool
compare method implementations and provide \DSU{} a list of
methods and equivalent program execution points between the two versions,
that would let \DSU{} support updates to these methods.
% \DSU{} can use this list to support updates to these methods.
Also, a
method body change might add or remove local variables, and \DSU{} must
maintain a consistent stack mapping between the old and updated versions.
Moreover, these local variables might refer to objects whose classes are
updated and need to be transformed before the updated method can resume
execution.

\acs{OSR} in \JikesRVM{} and the extensions describes above only support
changes to the topmost method on the activation stack. We propose to use
stack barriers to support changes to other methods on stack. In order to
replace other methods on stack, we propose to overwrite the return address
of their activation records to jump to special barrier code. This code
examines the stack, uses \acs{OSR} to update the current method and returns
control back to the application. By maintaining the invariant that the
topmost method will always be of the new version, \DSU{} can update all
active methods on stack.
