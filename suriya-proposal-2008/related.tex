\section{Related Work}

We discuss related work on supporting DSU in managed
languages and in C and C++. DSU for managed
languages falls into two categories: special-purpose VMs and libraries
and/or classloader-inserted or compiler-inserted code modifications.
Support for DSU in C and C++ often combines compiler and runtime
system support.  Existing approaches widely vary in their update
flexibility, safety guarantees, and run-time overhead.

% Broadly speaking, \DSU{} is one of the most flexible systems proposed
% to date, and % when compared to systems with similar flexibility,
% demonstrates superior performance and a realistic and thorough
% evaluation. 

\paragraph{Edit and Continue Development}

Debuggers have long provided \emph{edit and continue} (EnC)
functionality that permits
limited modifications to program state to avoid stopping and
restarting during debugging. For example, Sun's HotSwap
VM~\cite{JVMhotswap,Dmit01a}, .NET Visual Studio for C\# and
C++~\cite{VSEnC}, and library-based support~\cite{eaddy05enc} for .NET
applications all provide EnC.  These systems are all less flexible than
\DSU, typically supporting only code changes within method bodies.
This limitation reduces safety concerns, and programmers need not write
class or object transformers, but as discussed in
Section~\ref{sec:experience}, more than half of the updates we saw in
practice would be disallowed. 

%% As a vehicle for ``fix \& continue'' this is really popular!  Just Google
%% for edit and continue .net.  There was a big uproar when the feature was
%% taken away.

% \paragraph{DSU in Managed Languages}

%Approaches that implement DSU for managed languages can be divided into two
%categories: those that employ a special-purpose VM, and those based on
%compiler and/or library support.

% \paragraph*{Approaches with special VM support.}

\paragraph{Special VM support for DSU}

JDrums~\cite{ritzau00dynamic} and Dynamic Virtual Machine
(DVM)~\cite{Mala00a} both implement DSU support for Java within the
VM, providing a programming
interface similar to \DSU.  However, their implementations impose
overheads during normal execution, whereas \DSU{} has zero overhead
and a richer evaluation.
Both prior VMs update \emph{lazily}.  For example, JDrums traps object
pointer dereferences to check whether a new version of the object's
class is available. If so, the VM runs the object transformer
function(s) to upgrade the object.  By contrast \DSU{} performs
updates eagerly, as part of a full heap GC\@. DVM performs updates
lazily as JDrums, but does some eager conversion incrementally.  Lazy
updating has the advantage that the pause due to an update can be
amortized over subsequent execution.

The main drawback is that the overhead persists during normal execution
even though updates are relatively rare. \ksm{TODO: say why so much.}

Both JDrums and DVM are in the Sun JDK 1.2 VM, which uses an extra
level of indirection (the \emph{handle space}) to support heap
compaction.  Indirection conveniently supports object updates, but
adds additional space and time overhead.  DVM only works with the interpreter.
Relative to the stock bytecode interpreter, which is already slow, the
extra traps result in roughly 10\% overhead.  By contrast, \DSU{}
imposes no overhead before or after an update.  Neither JDrums nor
DVM has been evaluated on updates derived from realistic
applications. % ---only a handful of toy updates have been considered.
% Papers on 
% JDrums present only a toy phone book example and the DVM paper reports
% no experience at all.

%\suriya{is this a bit too strong?} \ksm{nope}

%% JDRUMS: implements the conversion lazily.  They have a similar interface
%% (object and class transformers). The drawback here is that there is overhead
%% in the general case of execution---we do not know when the update is
%% complete.  Implemented in Sun's JDK 1.2.  Adds a level of indirection to the
%% new object.  Thus overhead builds up over time.  It also appears they have a
%% more limited interface to what can be referenced in a conversion function.
%% For example, there is no way to refer to fields other than those of the
%% object's class (i.e., no super-class fields) and there is no way to call new
%% methods, like constructors.  Not clear if there are restrictions on how
%% methods can be changed.

%% DVM: use an incremental mark-sweep collector, where mark phase marks objects
%% to be updated and the sweep phase incrementally updates them (prior to being
%% accessed by the mutator).  Like JDRUMS, all accesses to marked objects are
%% trapped.  Imposes a stock 10\% overhead, even only using bytecode.

%% Both of these: no significant experience with real applications, according
%% to how they change in practice.  They also can't handle native methods
%% because they can't trap access to modified objects.  Doing the full GC
%% solves this problem.

% More recently, Nicoara et al. developed PROSE, a system for run-time
PROSE performs run-time
code patching with an API in the style of aspect-oriented
programming~\cite{nicoara:eurosys08}.  PROSE aims to support
short-term, run-time patches to code for logging, introspection, or
performance adaptation, rather than for DSU. % in support of run-time software
As such, PROSE only supports updates to method bodies,
with no support for signature or state changes. % This flexibility is
% similar to the EnC implementations discussed above; 
% indeed, PROSE builds on the HotSwap method replacement support in its
% Sun JDK implementation~\cite{JVMhotswap}.

Gilmore et al.~\cite{GilmoreKW97} propose DSU support for modules in
ML programs. They use a programming interface that is similar to ours, but
more restrictive.  They also propose using copying GC to perform the
update, as we do.  They formalized an abstract machine for
implementing updates using a copying garbage collector, but did not
implement it.

 
Boyapati et al.~\cite{boyapati03lazy} support lazily upgrading objects
in a \emph{persistent object store} (POS).  Though in a different
domain, their programming interface is quite similar to \DSU{} and the
other Java-based systems: programmers provide object transformer
functions for each class whose signature has changed.  In their system, object
transformers % are somewhat different than ours.  Their system
% allows the object transformer of some class $A$ to access the state of
of some class $A$ may access the state of
old objects pointed to by $A$'s fields, assuming these objects are
fully encapsulated; i.e., they are only reachable through $A$.
Encapsulation is ensured via extensions to the type system.  By
contrast, our transformers may dereference fields via old objects, but
if these fields point to objects whose classes have been updated, they
will see the \emph{new} versions. % (a semantics which is more
% typical).
We plan to explore the costs/benefit trade-off of these
semantics in future work.
%% To avoid programmer annotations we could use a tool like Uno


% \paragraph*{Approaches using a standard VM.}
\paragraph{Standard VM, with support for DSU}
To avoid changing the VM, researchers have developed special-purpose
classloaders, compiler support, or both for DSU.  The main drawbacks
of these approaches are less flexibility and greater
overhead.  Eisenbach and Barr~\cite{BarrE03} and Milazzo et
al.~\cite{Milazzo05updates} use custom classloaders to allow
binary-compatible changes and component-level changes, respectively.
The former targets libraries and the latter is part of the design of a
special-purpose software architecture.

%% Eisenbach and Barr: safe upgrading without restarting.  They support
%% upgrades that satisfy binary compatibility.  Uses a custom classloader and
%% JMX to replace the code of existing objects.  No way to modify the state of
%% the objects.

%% Milazzo use a modified class loader to load individual replacements to
%% classes in a special-purpose architecture.  The class loader may modify the
%% bytecode of the loaded class to deal with type version namespace issues.
%% Basically this is more limited in scope than our approach.

Orso et al.~\cite{orso:java} support DSU via source-to-source translation
by introducing a proxy class that indirects accesses
to objects that could change.  
%% For each class C that might change in the future they produce a proxy for
%% that class.  All calls from clients of C are redirected to call the wrapper
%% instead.  When C is updated by some new class C', a new C' object is created
%% and initialized using the old state of C and the wrapper is redirected to
%% point to C'.
This approach requires updated classes to export the same public
interface---it forbids new non-private methods and fields. % can be added to an
A more general limitation of non VM-based approaches
is that they are not \emph{transparent}---they make changes to the
class hierarchy, insert or rename classes, etc.  This approach makes
it essentially impossible to be robust in the face of code using
reflection or native methods.  Moreover, the extra runtime support
imposes both time and space overheads.  By contrast, modifying the VM
is much simpler, given its existing services.  Our VM approach handles
native methods and reflection,
and is more expressive, e.g., supporting signature changes.

%% Run-time mod of class hierarchy in a live ...
%% -  Also a classloader-based approach.  Classloaders are used to manage the 
%% *** skip this

\paragraph{Dynamic Software Updating for C/C++}
\suriya{DSU or expansion in headings?}

Recently several substantial systems for dynamically updating C and
C++ programs have emerged that target server
applications~\cite{HjalmtyssonG98,altekar05opus,neamtiu06dsu,chen:icse07}
and operating systems
components~\cite{K42reconfig,k42usenix,chen06vee,lee06linuxmod,dynamos_eurosys_07}.
% These systems are more mature than most of the systems described above, in
% some cases with substantial updating experience.  The flexibility

Although these systems are mature, the flexibility
afforded by \DSU{} is comparable or superior to most of them.
Only Ginseng~\cite{neamtiu06dsu} imposes fewer restrictions
on update timing than \DSU, but all these systems
cannot update multi-threaded programs or handle object-oriented language
features.

The lack of a VM is a significant disadvantage % in the implementation
for C and C++ DSU.  For example, because a VM-based \acs{JIT} can compile
and recompile replacement classes, it can update them with no
persistent overhead.  By contrast, C and C++ implementations must use
either statically-inserted
indirections~\cite{HjalmtyssonG98,neamtiu06dsu,K42reconfig,k42usenix}
or dynamically-inserted trampolines to redirect function
calls~\cite{altekar05opus,chen06vee,chen:icse07,Ksplice}.  Both cases impose
persistent overhead on normal execution, % in and of themselves, and can
and inhibit optimization.  Likewise, because these systems lack a garbage
collector, they either do not update object instances at
all~\cite{Ksplice}, update them
lazily~\cite{neamtiu06dsu,chen:icse07} or perform extra allocation and
allocator bookkeeping to be able to locate the objects at
update-time~\cite{k42usenix}.  Both of the latter two approaches impose time and space
overheads on normal execution, whereas \DSU{}'s VM-based approach has
no a priori overheads.  Finally, the fact that C and C++ are not
type-safe greatly complicates efforts to ensure that updates behave
correctly.

%% basic swapping for single modules.


